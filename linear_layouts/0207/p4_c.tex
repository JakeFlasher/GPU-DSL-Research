% ============================================================
%  BUTA — Blackwell Unified Tile Algebra
%  Part 4: Evaluation Strategy, Benchmark Suite,
%          Reproducibility, Threats, Limitations,
%          and Risk Assessment
% ============================================================
\documentclass[10pt,twocolumn]{article}

% ---- Geometry & Typography ----
\usepackage[margin=0.85in,columnsep=0.25in]{geometry}
\usepackage[T1]{fontenc}
\usepackage{lmodern}
\usepackage{microtype}

% ---- Mathematics ----
\usepackage{amsmath,amssymb,amsthm,mathtools}

% ---- Figures & Tables ----
\usepackage{graphicx,booktabs,array,multirow}
\usepackage{fancyvrb}
\usepackage{float}

% ---- Cross-references & Links ----
\usepackage[colorlinks,citecolor=blue!70!black,
            linkcolor=blue!60!black,urlcolor=blue!50!black]{hyperref}
\usepackage[capitalise,noabbrev]{cleveref}

% ---- Misc ----
\usepackage{xcolor}
\usepackage{enumitem}
\setlist{nosep,leftmargin=*}

% ---- Theorem-like environments ----
\theoremstyle{definition}
\newtheorem{definition}{Definition}[section]
\newtheorem{invariant}[definition]{Invariant}
\newtheorem{remark}[definition]{Remark}
\newtheorem{procedure}[definition]{Procedure}
\newtheorem{criterion}[definition]{Criterion}
\theoremstyle{plain}
\newtheorem{proposition}[definition]{Proposition}
\newtheorem{obligation}[definition]{Proof Obligation}

% ---- Convenience macros (matching Parts 1--3) ----
\newcommand{\BUTA}{\textsc{Buta}}
\newcommand{\Ftwo}{\mathbb{F}_2}
\newcommand{\ZZ}{\mathbb{Z}}
\newcommand{\NN}{\mathbb{N}}
\newcommand{\RR}{\mathbb{R}}
\newcommand{\ISL}{\textsc{ISL}}
\newcommand{\CuTe}{\textsc{CuTe}}
\newcommand{\TileIR}{Tile\,IR}
\newcommand{\TMEM}{\textsc{TMEM}}
\newcommand{\SMEM}{\textsc{SMEM}}
\newcommand{\tcgen}{\texttt{tcgen05}}
\newcommand{\tma}{\textsc{TMA}}
\newcommand{\II}{\mathit{II}}
\newcommand{\reuse}{d_{\mathrm{reuse}}}
\newcommand{\unverified}{\textsc{[unverified]}}
\newcommand{\inference}{\textsc{[inference]}}

% ---- Continue section numbering from Part 3 ----
\setcounter{section}{16}

% ---- Title ----
\title{%
  \BUTA{}: Blackwell Unified Tile Algebra\\[4pt]
  \large Part~4 --- Evaluation Strategy, Benchmark Suite,\\
         Reproducibility, Threats to Validity,\\
         Known Limitations, and Risk Assessment}

\author{%
  \textit{[Authors redacted for review]}\\
  Target architecture: NVIDIA Blackwell (B200 cc\,10.0;
  Grace-Blackwell GB10 cc\,12.0)\\
  Toolchain: CUDA ${>}\,13.0$ (prefer 13.1+),
  PTX ${>}\,9.0$ (strict)}

\date{February 2026}

\begin{document}
\maketitle

% ====================================================================
%  OVERVIEW
% ====================================================================
\noindent
Parts~1--3 have, respectively, framed the Blackwell
optimisation challenge and its formal preliminaries
(Part~1), defined the seven-object \BUTA{} model with
semantic layers, invariants, and proof obligations
(Part~2), and specified the solver encodings, layout
tooling, data-movement pipeline, cache analysis, and
calibration microbenchmark plan required to
operationalise the model (Part~3).
This concluding part establishes how \BUTA{} will be
\emph{evaluated} against ground-truth baselines
(\cref{sec:eval}); defines the concrete benchmark
suite and measurement methodology (\cref{sec:bench});
specifies the hardware, toolchain, and release plan
for reproducibility (\cref{sec:repro}); analyses
threats to validity (\cref{sec:threats}); enumerates
known theoretical and practical limitations
(\cref{sec:limitations}); and presents a severity-ranked
risk assessment for the nine identified gaps~G1--G9
with a phased mitigation timeline
(\cref{sec:risk}).


% ====================================================================
%  SECTION 17 — EVALUATION STRATEGY
% ====================================================================
\section{Evaluation Strategy}
\label{sec:eval}

The \BUTA{} evaluation must establish three claims:
(i)~the solver produces schedules competitive with or
superior to expert-designed implementations;
(ii)~the analytic cache model predicts hardware
behaviour with bounded error; and
(iii)~the layout verification pipeline correctly
classifies layouts at all three representation layers.
Each claim requires a distinct validation methodology,
defined below.

\subsection{Solver Output vs.\ Expert Schedules}
\label{sec:eval:solver}

\begin{definition}[Schedule Quality
  Metric~\cite{OPT_PIPE}]
\label{def:sched-metric}
For a given tile-level dependence graph $G = (V, E)$
and cost vector $\mathbf{c}(x)$~\cite{ARCH_BW},
the primary quality metric is the \emph{achieved
initiation interval} $\II_{\mathrm{solver}}$ produced
by the SMT formulation (Part~3,
\cref{def:smt-clauses}).
The baseline comparison is the \emph{expert initiation
interval} $\II_{\mathrm{expert}}$ extracted from
published Flash Attention implementations (FA3, FA4)
on Blackwell~\cite{OPT_PIPE}.
The evaluation reports:
\begin{equation}
  \rho_{\II} =
  \frac{\II_{\mathrm{expert}}}
       {\II_{\mathrm{solver}}}
  \label{eq:ii-ratio}
\end{equation}
where $\rho_{\II} \leq 1$ indicates the solver
matches or improves upon the expert, and
$\rho_{\II} > 1$ indicates the solver discovers a
superior schedule.
\end{definition}

\begin{definition}[End-to-End Throughput
  Comparison~\cite{OPT_PIPE, ARCH_BW}]
\label{def:throughput-metric}
The secondary metric is \emph{achieved throughput}
$\Theta$ in TFLOPS, measured on hardware:
\begin{equation}
  \Theta =
  \frac{2 \cdot M \cdot N \cdot K}
       {t_{\mathrm{kernel}} \cdot 10^{12}}
  \label{eq:tflops}
\end{equation}
for an $M \times N \times K$ problem executed in
$t_{\mathrm{kernel}}$ seconds.
Solver-generated schedules are compiled to PTX
($> 9.0$) and executed on both B200 (cc\,10.0,
148~SMs) and GB10 (cc\,12.0)~\cite{ARCH_BW}.
Comparison baselines include:
\begin{enumerate}[label=(\alph*)]
  \item Expert FA3/FA4 implementations for Flash
    Attention~\cite{OPT_PIPE}.
  \item cuBLAS default schedule for dense
    GEMM~\cite{ARCH_BW}.
  \item Triton-to-\TileIR{} auto-generated
    schedules~\cite{NV_BLOG_TILE}.
\end{enumerate}
\end{definition}

\begin{definition}[Cross-Generation Schedule
  Distinction~\cite{OPT_PIPE}]
\label{def:cross-gen}
To confirm that the Blackwell cost vector
$\mathbf{c}(10.0)$ captures genuine architectural
differences, the evaluation verifies that solver-optimal
schedules for Blackwell \emph{differ structurally} from
Hopper-optimal schedules produced by the same
solver with Hopper cost
parameterisation~\cite{OPT_PIPE}.
Structural difference is measured by:
\begin{equation}
  d_{\mathrm{struct}} =
  \bigl|\{v \in V :
    W_{\mathrm{BW}}(v) \neq W_{\mathrm{HP}}(v)
    \;\lor\;
    \theta_{\mathrm{BW}}(v) \neq \theta_{\mathrm{HP}}(v)
  \}\bigr| / |V|
  \label{eq:struct-diff}
\end{equation}
where $W(\cdot)$ is the warp assignment and
$\theta(\cdot)$ is the pipeline stage assignment.
A nonzero $d_{\mathrm{struct}}$ confirms the cost
vector's discriminative
power~\cite{OPT_PIPE, ARCH_BW}.
\end{definition}

\begin{figure}[t]
\centering
\begin{BVerbatim}[fontsize=\scriptsize]
+===========================================================+
|  Evaluation Flow: Solver vs Expert                        |
+===========================================================+
|                                                           |
|  INPUT:                                                   |
|    Dependence graph G = (V, E)  [per benchmark kernel]    |
|    Cost vector c(x) [calibrated, Part 3, ARCH_BW]         |
|    Expert schedule (M_exp, W_exp, II_exp) [OPT_PIPE]      |
|                                                           |
|  PHASE A: Solver Execution                                |
|  +-----------------------------------------------------+  |
|  | Run ILP -> II_lb  (Phase 1, Part 3)                  |  |
|  | Run SMT -> (M_sol, W_sol, II_sol)  (Phase 2)        |  |
|  | IF num_ctas=2: include Phi_2cta  [NV_BLOG_TILE]      |  |
|  | Record: solve_time, II_sol                           |  |
|  +--+--------------------------------------------------+  |
|     |                                                     |
|  PHASE B: Schedule Comparison                             |
|  +-----------------------------------------------------+  |
|  | Compute rho_II = II_exp / II_sol                     |  |
|  | Compute d_struct (warp + stage diff)                 |  |
|  | Verify INV-S1..S5 on solver output [Part 2]          |  |
|  +--+--------------------------------------------------+  |
|     |                                                     |
|  PHASE C: Hardware Execution                              |
|  +-----------------------------------------------------+  |
|  | Compile solver schedule -> PTX > 9.0                 |  |
|  | Execute on B200 (cc 10.0) and GB10 (cc 12.0)        |  |
|  | Measure Theta (TFLOPS) via Nsight Compute            |  |
|  | Compare: Theta_sol vs Theta_exp vs Theta_triton      |  |
|  +-----------------------------------------------------+  |
|                                                           |
|  OUTPUT:                                                  |
|    Table: kernel x platform x {II, Theta, rho_II}        |
|    Structural diff d_struct (Blackwell vs Hopper)         |
|    Invariant verification report                          |
+===========================================================+
\end{BVerbatim}
\caption{Three-phase evaluation flow for solver
  output comparison.
  Phase~A invokes the ILP+SMT pipeline
  (Part~3)~\cite{OPT_PIPE};
  Phase~B compares against expert baselines;
  Phase~C validates on Blackwell
  hardware~\cite{ARCH_BW} with Tile~IR
  compilation~\cite{NV_BLOG_TILE}.}
\label{fig:eval-solver}
\end{figure}

\begin{remark}
Soi et al.~\cite{OPT_PIPE} report that their Twill
system rediscovers expert Flash Attention schedules on
both Hopper and Blackwell.
The \BUTA{} evaluation extends this by additionally
testing the 2CTA extension
$\Phi_{\mathrm{2cta}}$~\cite{NV_BLOG_TILE} and the
joint layout-schedule formulation (gap~G4), which are
absent from the Twill baseline.
Solve time is expected to remain below 60~seconds for
Flash Attention-scale
problems~\cite{OPT_PIPE}.
\end{remark}


\subsection{Cache Model Validation}
\label{sec:eval:cache}

\begin{definition}[Miss Prediction Error
  Metric~\cite{SEED_3, ARCH_BW}]
\label{def:cache-metric}
For a given tile program, tile ordering~$\sigma$, and
platform with parameters
$(N_{\mathrm{SM}}, C_{\mathrm{L2}})$, the
\emph{relative prediction error} is:
\begin{equation}
  \varepsilon_{\mathrm{miss}}(\sigma) =
  \frac{|\mathcal{C}(\sigma, N_{\mathrm{SM}},
    C_{\mathrm{L2}}) -
    M_{\mathrm{hw}}(\sigma)|}
    {M_{\mathrm{hw}}(\sigma)}
  \label{eq:miss-error}
\end{equation}
where $\mathcal{C}$ is the predicted miss count
(Part~2, \cref{def:cache-func}) and
$M_{\mathrm{hw}}$ is the measured miss count obtained
from Nsight Compute \texttt{l2\_sector\_misses}
hardware performance counters.
\end{definition}

\begin{definition}[Validation Protocol~\cite{SEED_3}]
\label{def:cache-valid-protocol}
The cache model is validated in three tiers of
increasing generality:

\smallskip\noindent
\textbf{Tier~1 (reproduction).}
Reproduce the SEED\_3 result on GB10
(cc\,12.0, 128\,KB \SMEM{}): sawtooth ordering
reduces L2 misses by $67\% \pm 5\%$ (370M $\to$ 120M
sectors) for Flash Attention causal
mask~\cite{SEED_3}.
The model prediction
$\mathcal{C}(\sigma_{\mathrm{saw}})$ must satisfy
$\varepsilon_{\mathrm{miss}} < 0.15$
(15\% relative error)~\cite{SEED_3}.

\smallskip\noindent
\textbf{Tier~2 (cross-platform).}
Execute the same Flash Attention configuration on B200
(cc\,10.0, 228\,KB \SMEM{}, 148~SMs) with platform
parameters from~\cite{ARCH_BW}.
Verify that the model adapts to the larger \SMEM{}
capacity and different SM count, with
$\varepsilon_{\mathrm{miss}} < 0.20$ \inference{}.
The 58\% cache-miss latency
reduction~\cite{ARCH_BW} may alter the
performance-per-miss relationship but should not
affect miss count prediction.

\smallskip\noindent
\textbf{Tier~3 (generalisation).}
Extend to dense GEMM and mixed-precision GEMM
(gap~G5).
For non-streaming access patterns where tile reuse is
significant, the model must either maintain
$\varepsilon_{\mathrm{miss}} < 0.20$ or correctly
identify when its streaming assumption is violated
and report degraded confidence \inference{}.
\end{definition}

\begin{figure}[t]
\centering
\begin{BVerbatim}[fontsize=\scriptsize]
+===========================================================+
|  Cache Model Validation Tiers                             |
+===========================================================+
|                                                           |
|  TIER 1: Reproduction (GB10)                [SEED_3]      |
|  +-----------------------------------------------------+  |
|  | Workload: Flash Attention causal, N=8192, d=128      |  |
|  | Platform: GB10 cc 12.0, SMEM 128 KB                  |  |
|  | Orderings: raster, sawtooth                          |  |
|  | Target:    eps_miss < 0.15                           |  |
|  | Baseline:  370M -> 120M sectors (67% reduction)      |  |
|  +--+--------------------------------------------------+  |
|     | PASS => proceed                                     |
|     v                                                     |
|  TIER 2: Cross-Platform (B200)              [ARCH_BW]     |
|  +-----------------------------------------------------+  |
|  | Workload: Flash Attention (same config)              |  |
|  | Platform: B200 cc 10.0, SMEM 228 KB, 148 SMs        |  |
|  | Adjust:   tile size for 228 KB SMEM                  |  |
|  | Target:   eps_miss < 0.20 [INFERENCE]                |  |
|  | Verify:   model adapts to different N_SM, C_SMEM     |  |
|  +--+--------------------------------------------------+  |
|     | PASS => proceed                                     |
|     v                                                     |
|  TIER 3: Generalisation                     [gap G5]      |
|  +-----------------------------------------------------+  |
|  | Workloads: dense GEMM, mixed-prec GEMM, conv         |  |
|  | Platforms: B200 + GB10                                |  |
|  | Target:    eps_miss < 0.20 for streaming patterns    |  |
|  |            OR correct degraded-confidence flag        |  |
|  | Classify:  per NV_workloads taxonomy [NV_workloads]  |  |
|  +-----------------------------------------------------+  |
|                                                           |
|  FAILURE PROTOCOL:                                        |
|    If Tier 1 fails => model is unsound; revise d_reuse   |
|    If Tier 2 fails => cross-platform params incorrect     |
|    If Tier 3 fails => streaming assumption insufficient;  |
|                       extend model (gap G5)               |
+===========================================================+
\end{BVerbatim}
\caption{Three-tier cache model validation.
  Tier~1 reproduces~\cite{SEED_3};
  Tier~2 tests cross-platform transfer to
  B200~\cite{ARCH_BW};
  Tier~3 generalises beyond Flash Attention,
  classified by the HARP
  taxonomy~\cite{NV_workloads}.
  Failure at each tier triggers a specific
  remediation.}
\label{fig:cache-valid}
\end{figure}


\subsection{Layout Correctness Testing}
\label{sec:eval:layout}

\begin{definition}[Layout Verification
  Acceptance Criteria~\cite{SEED_1, SEED_2, SEED_4}]
\label{def:layout-criteria}
The three-layer layout verification pipeline
(Part~3, \cref{fig:layout-pipeline}) is tested
against ground-truth layout libraries from the
golden sources.
Acceptance requires:
\begin{enumerate}[label=(\alph*)]
  \item \textbf{$\Ftwo$ layer}: for every
    linear layout $L$ in the Triton test suite
    of~\cite{SEED_1}, the pipeline correctly
    computes $\mathrm{rank}(L)$, verifies
    injectivity, and produces correct composition
    $L_2 \cdot L_1$ for all tested pairs.
  \item \textbf{\ISL{} layer}: for every \CuTe{} layout
    $H$ convertible by Algorithm~1
    of~\cite{SEED_2}, the pipeline produces an
    \ISL{} relation $R_H$ satisfying
    $|R_H| = |H|$ (layout size) and
    $\|R_H\| = \|H\|$ (co-size).
    Swizzle patterns are correctly represented as
    bit-level \ISL{} operations~\cite{SEED_2}.
  \item \textbf{Categorical layer}: for every layout
    in the CUTLASS test suite of~\cite{SEED_4},
    the pipeline correctly identifies whether the
    layout is \emph{tractable} (arises from
    $\mathbf{Nest}$ categorical construction) and
    produces a matching descriptor:
    $L_{\mathrm{computed}} = L_{\mathrm{CUTLASS}}$
    element-wise over the tile
    domain~\cite{SEED_4}.
  \item \textbf{Cross-layer consistency}: for every
    layout representable at multiple layers, the
    embedding $\phi$
    (Part~2, \cref{def:phi-embed}) and functor~$F$
    (Part~2, \cref{def:F-functor}) produce
    consistent results: $\phi(L) = R_L$ at the
    $\Ftwo$--\ISL{} boundary, and
    $F(f) = R_f$ at the
    $\mathbf{Nest}$--\ISL{} boundary.
    This validates PO-1 and PO-2
    (Part~2)~\cite{SEED_1, SEED_2, SEED_4}.
\end{enumerate}
\end{definition}

\begin{remark}
Layouts for FP4 and FP6 precisions have not been
characterised in the golden sources \unverified{}
(gap~G8).
The evaluation will include FP4/FP6 layouts
\emph{only} if empirical \tcgen{} microbenchmarks
(Part~3, \cref{def:mma-calib}) reveal their structure;
otherwise, FP4/FP6 layout testing is deferred and
labelled accordingly~\cite{ARCH_BW, SEED_1}.
\end{remark}


% ====================================================================
%  SECTION 18 — BENCHMARK SUITE
% ====================================================================
\section{Benchmark Suite}
\label{sec:bench}

\subsection{Kernel Selection and Justification}
\label{sec:bench:kernels}

The benchmark suite is designed to exercise all four
\BUTA{} semantic layers across multiple workload types.
Kernel selection is governed by two criteria:
(i)~coverage of the HARP taxonomy axes
(mixed-reuse, compute-bound, memory-bound)%
~\cite{NV_workloads}; and
(ii)~availability of expert baselines for
comparative evaluation~\cite{OPT_PIPE, ARCH_BW}.

\begin{definition}[Benchmark Kernel
  Set]
\label{def:kernel-set}
The benchmark suite $\mathcal{B}$ consists of five
kernel families:

\smallskip\noindent
\textbf{B1: Flash Attention Forward.}
$Q, K, V \in \mathbb{R}^{N \times d}$;
split-$Q$ dataflow with persistent CTAs;
$N \in \{2048, 4096, 8192, 16384\}$,
$d \in \{64, 128, 256\}$;
causal and non-causal variants.
This is the primary evaluation kernel:
expert baselines exist~\cite{OPT_PIPE},
cache model is validated~\cite{SEED_3},
2CTA mode is critical~\cite{NV_BLOG_TILE},
and the workload is mixed-reuse
(MMA + softmax epilogue)~\cite{NV_workloads}.

\smallskip\noindent
\textbf{B2: Flash Attention Backward.}
Gradient computation with $dQ$, $dK$, $dV$ accumulation;
more complex dependence graph than forward;
higher memory pressure (three gradient matrices
co-resident).
Tests scheduling constraint system scalability.
Expert baselines from~\cite{OPT_PIPE}.

\smallskip\noindent
\textbf{B3: Dense GEMM.}
$C = A \cdot B$ with
$M, N, K \in \{1024, 2048, 4096, 8192\}$;
$\{$FP16, BF16, FP8, FP4$\}$ input types;
FP16 and FP32 accumulation.
Tests precision lattice $\mathcal{P}$
conditioning~\cite{ARCH_BW},
2CTA mode ($\texttt{num\_ctas} = 2$
mandatory)~\cite{NV_BLOG_TILE}, and
layout enumeration for the joint
formulation~\cite{SEED_1}.

\smallskip\noindent
\textbf{B4: Mixed-Precision GEMM.}
Input operands at different precisions
(e.g., $A$ in FP8, $B$ in FP16);
block scaling formats (MXFP4/MXFP8)
\unverified{}.
Tests precision-dependent layout space
$\mathcal{L}(p)$~\cite{ARCH_BW, SEED_1}
and heterogeneous cost vectors
(gap~G8)~\cite{NV_workloads}.

\smallskip\noindent
\textbf{B5: Implicit GEMM Convolution.}
2D convolution lowered to GEMM via im2col
transformation;
$C_{\mathrm{in}} \in \{64, 128, 256\}$,
$C_{\mathrm{out}} \in \{128, 256, 512\}$,
kernel sizes $\{1{\times}1, 3{\times}3\}$.
Tests \tma{} im2col descriptor type
(\texttt{idesc})~\cite{NV_BLOG_TILE},
non-linear swizzle in \ISL{}
representation~\cite{SEED_2}, and
cache model generalisation (gap~G5).
\end{definition}

\begin{figure}[t]
\centering
\begin{BVerbatim}[fontsize=\scriptsize]
+===========================================================+
|  Benchmark Suite Coverage Matrix                          |
+===========================================================+
|                                                           |
|  Kernel    | BUTA Layers Exercised  | HARP Class   | 2CTA |
|  ----------+------------------------+--------------+------+
|  B1: FA Fwd| L + S + D + C          | mixed-reuse  | YES  |
|  B2: FA Bwd| L + S + D + C          | mixed-reuse  | YES  |
|  B3: GEMM  | L + S + D + (C)        | compute-bnd  | YES  |
|  B4: MxGEMM| L + S + D              | mixed-reuse  | YES  |
|  B5: Conv  | L + S + D + (C)        | mixed-reuse  | opt  |
|  ----------+------------------------+--------------+------+
|                                                           |
|  Legend:                                                   |
|    L = Layout Morphism Space [SEED_1, SEED_2, SEED_4]     |
|    S = Schedule Constraint System [OPT_PIPE, ARCH_BW]     |
|    D = Data Movement Relation [NV_BLOG_TILE, ARCH_BW]     |
|    C = Cache Traffic Function [SEED_3, ARCH_BW]           |
|    (C) = cache model applied but INFERENCE (gap G5)       |
|    opt = optional; depends on kernel configuration        |
|                                                           |
|  Precision coverage:                                      |
|    B1, B2: FP16, BF16  (primary: FP16)                    |
|    B3:     FP16, BF16, FP8, FP4  [ARCH_BW]               |
|    B4:     mixed FP8+FP16, MXFP4/MXFP8 [UNVERIFIED]     |
|    B5:     FP16, FP8                                      |
|                                                           |
|  All benchmarks cite: ARCH_BW (hw params), OPT_PIPE       |
|    (solver), NV_BLOG_TILE (Tile IR target)                |
+===========================================================+
\end{BVerbatim}
\caption{Benchmark suite coverage matrix.
  Every kernel exercises the layout and scheduling
  layers; cache traffic analysis is primary for
  Flash Attention~\cite{SEED_3} and \inference{}
  for GEMM/convolution (gap~G5).
  The HARP taxonomy~\cite{NV_workloads} classifies
  each kernel by workload type.
  2CTA mode is required for all dense MMA
  kernels~\cite{NV_BLOG_TILE}.}
\label{fig:bench-coverage}
\end{figure}


\subsection{Platform Configuration}
\label{sec:bench:platform}

\begin{definition}[Hardware
  Configurations~\cite{ARCH_BW, SEED_3}]
\label{def:hw-config}
All benchmarks execute on two platforms:

\smallskip\noindent
\textbf{Platform~A: NVIDIA B200} (cc\,10.0).
148~SMs across 8~GPCs (dual-die NV-HBI);
\SMEM{} 228\,KB/SM; \TMEM{} 256\,KB/SM;
L2 $\approx 65$\,MB unified;
HBM3e 192\,GB at 8\,TB/s;
up to 64 concurrent warps per SM;
4~warp schedulers per SM~\cite{ARCH_BW}.

\smallskip\noindent
\textbf{Platform~B: NVIDIA Grace-Blackwell GB10}
(cc\,12.0).
\SMEM{} 128\,KB/SM; \TMEM{} 256\,KB/SM;
L2 $\approx 65$\,MB;
up to 48 concurrent warps per SM~\cite{ARCH_BW}.
SM count differs from B200; measured at
runtime~\cite{SEED_3}.
\end{definition}

\begin{definition}[Software
  Configuration~\cite{NV_BLOG_TILE}]
\label{def:sw-config}
Toolchain requirements:
\begin{enumerate}[label=(\roman*)]
  \item CUDA $\geq 13.1$~\cite{NV_BLOG_TILE}.
  \item PTX $> 9.0$ (strict)~\cite{NV_BLOG_TILE}.
  \item NVIDIA Nsight Compute for hardware performance
    counters (L2 sector misses, SM active cycles,
    tensor core utilisation).
  \item SMT solver: Z3 $\geq 4.12$ or cvc5 $\geq 1.0$
    for the constraint
    system~\cite{OPT_PIPE}.
  \item ILP solver: HiGHS, Gurobi, or CPLEX for
    the initiation interval lower
    bound~\cite{OPT_PIPE}.
  \item \ISL{} library $\geq 0.26$ for layout
    analysis~\cite{SEED_2}.
\end{enumerate}
Clock frequencies are locked to avoid DVFS artefacts.
Each measurement is repeated $\geq 100$ times with
warm-up; reported as median $\pm$ IQR
(Part~3, \cref{def:meas-protocol}).
\end{definition}


\subsection{Metrics and Measurement Methodology}
\label{sec:bench:metrics}

\begin{definition}[Evaluation Metrics]
\label{def:all-metrics}
The evaluation reports six primary metrics per
benchmark-platform pair:

\smallskip\noindent
(M1)~\textbf{Initiation interval} $\II_{\mathrm{sol}}$
and ratio $\rho_{\II}$
(\cref{def:sched-metric})~\cite{OPT_PIPE}.

\smallskip\noindent
(M2)~\textbf{Throughput} $\Theta$ in TFLOPS
(\cref{def:throughput-metric})~\cite{ARCH_BW}.

\smallskip\noindent
(M3)~\textbf{L2 miss prediction error}
$\varepsilon_{\mathrm{miss}}$
(\cref{def:cache-metric})~\cite{SEED_3}.

\smallskip\noindent
(M4)~\textbf{Layout verification pass rate}:
fraction of test layouts correctly verified at each
layer~\cite{SEED_1, SEED_2, SEED_4}.

\smallskip\noindent
(M5)~\textbf{Solve time} $t_{\mathrm{solve}}$ in
seconds for the ILP+SMT
pipeline~\cite{OPT_PIPE}.

\smallskip\noindent
(M6)~\textbf{Data-race freedom}: count of
well-formed token DAGs verified DRF by the procedure
of Part~3, \cref{def:drf-verify}~\cite{NV_BLOG_TILE}.

\smallskip\noindent
Metrics M1--M3 require hardware execution; M4--M6 are
computed offline from solver and verification
outputs.
\end{definition}


% ====================================================================
%  SECTION 19 — REPRODUCIBILITY PLAN
% ====================================================================
\section{Reproducibility Plan}
\label{sec:repro}

\subsection{Hardware and Toolchain Requirements}
\label{sec:repro:hw}

\begin{definition}[Minimum Reproduction
  Configuration]
\label{def:min-repro}
Full reproduction of \BUTA{} evaluation results
requires:
\begin{enumerate}[label=(\roman*)]
  \item At least one NVIDIA Blackwell GPU
    (B200 or GB10)~\cite{ARCH_BW}.
  \item CUDA Toolkit $\geq 13.1$ with
    \TileIR{} support~\cite{NV_BLOG_TILE}.
  \item PTX compiler targeting
    $> 9.0$~\cite{NV_BLOG_TILE}.
  \item Root or equivalent access for clock frequency
    locking and Nsight Compute profiling.
\end{enumerate}
Solver-only evaluation (M1, M4--M6) requires no GPU:
the ILP/SMT formulation can be validated on any
machine with Z3 and the \ISL{}
library~\cite{OPT_PIPE, SEED_2}.
\end{definition}

\begin{remark}
Blackwell hardware availability may be constrained
during the review period.
We provide a \emph{solver-only} reproduction path
that validates the constraint encodings, layout
pipeline, and token DAG verification without hardware,
and a \emph{full} reproduction path that additionally
validates throughput and cache predictions on
hardware.
\end{remark}


\subsection{Open-Source Release Plan}
\label{sec:repro:release}

\begin{definition}[Artifact
  Components]
\label{def:artifacts}
The release comprises five components:

\smallskip\noindent
\textbf{A1: Solver implementation.}
Python/C++ implementation of the ILP+SMT pipeline
(Part~3, \cref{sec:solver}) with the 2CTA extension
$\Phi_{\mathrm{2cta}}$~\cite{NV_BLOG_TILE, OPT_PIPE}
and ZLP normalisation~\cite{OPT_PIPE}.
Input: dependence graph $G$ and cost vector
$\mathbf{c}(x)$.
Output: schedule $(M, W, \II, \mathrm{cta})$.

\smallskip\noindent
\textbf{A2: Layout verification pipeline.}
Three-layer pipeline implementation with $\Ftwo$
library~\cite{SEED_1}, \ISL{} analysis
module~\cite{SEED_2}, and categorical
validation module extending the SEED\_4
Python codebase~\cite{SEED_4}.
Input: layout descriptor.
Output: per-layer verification report.

\smallskip\noindent
\textbf{A3: Cache traffic analyser.}
Reuse-distance computation engine and parameterised
miss predictor~\cite{SEED_3} with
Blackwell-specific platform parameters
from~\cite{ARCH_BW}.
Input: tile loop nest and ordering $\sigma$.
Output: predicted miss count $\mathcal{C}(\sigma)$.

\smallskip\noindent
\textbf{A4: Calibration microbenchmark suite.}
PTX-level microbenchmarks for \tcgen{}\texttt{.mma}
latency, \TMEM{} bandwidth, \tma{} overhead, and L2
characterisation (Part~3,
\cref{sec:calibration})~\cite{ARCH_BW}.

\smallskip\noindent
\textbf{A5: Benchmark harness.}
Automated execution of the benchmark suite
(\cref{def:kernel-set}) with result collection,
metric computation, and
comparison~\cite{ARCH_BW, OPT_PIPE}.
\end{definition}

\begin{figure}[t]
\centering
\begin{BVerbatim}[fontsize=\scriptsize]
+===========================================================+
|  Artifact Package Structure                               |
+===========================================================+
|                                                           |
|  buta/                                                    |
|  |-- solver/                                              |
|  |   |-- ilp_encoder.py       # ILP Phase 1 [OPT_PIPE]   |
|  |   |-- smt_encoder.py       # SMT Phase 2 [OPT_PIPE]   |
|  |   |-- cta2_extension.py    # 2CTA [NV_BLOG_TILE]      |
|  |   |-- zlp_normalise.py     # ZLP [OPT_PIPE]           |
|  |   +-- cost_vectors/                                    |
|  |       |-- blackwell_b200.json   # c(10.0) [ARCH_BW]   |
|  |       +-- blackwell_gb10.json   # c(12.0) [ARCH_BW]   |
|  |                                                        |
|  |-- layout/                                              |
|  |   |-- f2_library.py        # F2 ops [SEED_1]          |
|  |   |-- isl_analysis.py      # ISL layer [SEED_2]       |
|  |   |-- cat_validation.py    # Nest/Tuple [SEED_4]      |
|  |   +-- pipeline.py          # 3-layer verifier          |
|  |                                                        |
|  |-- cache/                                               |
|  |   |-- reuse_distance.py    # LRU stack [SEED_3]       |
|  |   |-- miss_predictor.py    # C(sigma) [SEED_3]        |
|  |   +-- ordering_opt.py      # sigma search [SEED_3]    |
|  |                                                        |
|  |-- calibration/                                         |
|  |   |-- tcgen05_latency.cu   # MMA bench [ARCH_BW]      |
|  |   |-- tmem_bandwidth.cu    # TMEM bench [ARCH_BW]     |
|  |   |-- tma_overhead.cu      # TMA bench [NV_BLOG_TILE] |
|  |   +-- l2_probe.cu          # L2 bench [ARCH_BW]       |
|  |                                                        |
|  |-- benchmarks/                                          |
|  |   |-- flash_attn_fwd.py    # B1                        |
|  |   |-- flash_attn_bwd.py    # B2                        |
|  |   |-- dense_gemm.py        # B3                        |
|  |   |-- mixed_gemm.py        # B4                        |
|  |   +-- conv_igemm.py        # B5                        |
|  |                                                        |
|  +-- tests/                                               |
|      |-- test_layouts.py      # CUTLASS test suite        |
|      |-- test_invariants.py   # INV-* checks              |
|      +-- test_token_dag.py    # DRF verification          |
+===========================================================+
\end{BVerbatim}
\caption{Artifact package structure.
  Each module is anchored to its defining golden
  source.
  The solver-only path (no GPU required) exercises
  \texttt{solver/}, \texttt{layout/},
  \texttt{cache/}, and \texttt{tests/}.
  Full evaluation additionally requires
  \texttt{calibration/} and \texttt{benchmarks/}
  on Blackwell hardware~\cite{ARCH_BW}.}
\label{fig:artifact-structure}
\end{figure}


\subsection{Artifact Evaluation Criteria}
\label{sec:repro:ae}

\begin{criterion}[Artifact Badges]
\label{crit:badges}
The release targets three standard artifact evaluation
badges:
\begin{enumerate}[label=(\roman*)]
  \item \textbf{Available}: all five components (A1--A5)
    released under an open-source licence on a public
    repository.
  \item \textbf{Functional}: solver-only path executes
    to completion on all benchmark dependence graphs,
    produces valid schedules (invariants verified), and
    layout pipeline passes all CUTLASS test
    layouts~\cite{SEED_4}.
  \item \textbf{Reproduced}: on Blackwell hardware, the
    benchmark harness reproduces key results within
    stated tolerances: $\rho_{\II} \leq 1.05$
    (within 5\% of expert
    $\II$)~\cite{OPT_PIPE};
    $\varepsilon_{\mathrm{miss}} < 0.15$ for Tier~1
    cache validation~\cite{SEED_3};
    throughput $\Theta_{\mathrm{sol}} \geq 0.90 \cdot
    \Theta_{\mathrm{expert}}$ for Flash
    Attention~\cite{OPT_PIPE, ARCH_BW}.
\end{enumerate}
\end{criterion}


% ====================================================================
%  SECTION 20 — THREATS TO VALIDITY
% ====================================================================
\section{Threats to Validity}
\label{sec:threats}

\subsection{Internal Threats}
\label{sec:threats:internal}

\begin{definition}[Internal Threat
  Enumeration]
\label{def:internal-threats}
The following factors may compromise the
internal validity of \BUTA{} results:

\smallskip\noindent
\textbf{IT-1: Compiler maturity.}
Jarmusch and Chandrasekaran~\cite{ARCH_BW} observe
that Blackwell dense GEMM throughput lags Hopper by
up to $4\times$, despite $1.56\times$ mixed-precision
throughput gains.
This is attributed to compiler maturity rather than
architectural limitation~\cite{ARCH_BW}.
If the baseline compiler improves substantially
during the evaluation period, reported relative
improvements may shrink.

\smallskip\noindent
\textbf{IT-2: Triton code generation.}
Soi et al.~\cite{OPT_PIPE} identify instances where
Triton makes incorrect code generation decisions.
If solver-generated schedules are compiled via the
Triton-to-\TileIR{} path~\cite{NV_BLOG_TILE},
Triton-level errors may mask solver-level optimality.

\smallskip\noindent
\textbf{IT-3: Undocumented microarchitecture.}
Certain Blackwell parameters are partially documented
or derived from secondary analysis~\cite{ARCH_BW}:
\TMEM{} bandwidth ($\approx$16/8~TB/s read/write per
SM) is \unverified{};
pipeline depth and cache interaction details are
partially unknown~\cite{ARCH_BW}.
Cost vector errors propagate to solver output.

\smallskip\noindent
\textbf{IT-4: \SMEM{} capacity divergence.}
The \SMEM{} capacity differs between B200 (228\,KB,
cc\,10.0) and GB10 (128\,KB,
cc\,12.0)~\cite{ARCH_BW}.
Benchmarks that saturate \SMEM{} on one platform may
behave differently on the other, complicating
cross-platform comparison.

\smallskip\noindent
\textbf{IT-5: ZLP rounding sensitivity.}
The ZLP normalisation procedure
(Part~3)~\cite{OPT_PIPE} introduces
$|\hat{\II} - \II^{*}| \leq 1$ error.
For kernels with very small $\II$, this rounding may
dominate the schedule quality metric.
\end{definition}


\subsection{External Threats}
\label{sec:threats:external}

\begin{definition}[External Threat
  Enumeration]
\label{def:external-threats}
Factors that may limit the generalisability of
\BUTA{} results:

\smallskip\noindent
\textbf{ET-1: Workload diversity.}
The benchmark suite (\cref{def:kernel-set}) covers
five kernel families.
Workloads with fundamentally different access
patterns (e.g., sparse operators, sequence-to-sequence
with variable-length batching) are not represented.
The HARP taxonomy~\cite{NV_workloads} classifies
additional workload types (cross-depth heterogeneous)
that \BUTA{} does not target.

\smallskip\noindent
\textbf{ET-2: Architecture specificity.}
\BUTA{} is defined exclusively for Blackwell
(cc\,10.0, cc\,12.0)~\cite{ARCH_BW}.
The model's applicability to future NVIDIA architectures
or non-NVIDIA accelerators is not established.
The \TileIR{} backend targets Blackwell
only~\cite{NV_BLOG_TILE}.

\smallskip\noindent
\textbf{ET-3: Problem scale.}
The ILP+SMT solver is validated on singly-nested
tile loops~\cite{OPT_PIPE}.
Workloads with deeply nested control flow, dynamic
shapes, or conditional branching exceed the modelling
scope.
\end{definition}


\subsection{Construct Threats}
\label{sec:threats:construct}

\begin{definition}[Construct Threat
  Enumeration]
\label{def:construct-threats}
Threats arising from the modelling abstractions:

\smallskip\noindent
\textbf{CT-1: Singly-nested loop restriction.}
The OPT\_PIPE formulation~\cite{OPT_PIPE} is
restricted to singly-nested loops without additional
control flow.
Flash Attention's inner loop satisfies this
constraint, but backward passes and fused kernels may
violate it.

\smallskip\noindent
\textbf{CT-2: Token abstraction fidelity.}
The \TileIR{} token model~\cite{NV_BLOG_TILE} is an
abstraction of hardware memory ordering.
If the abstraction is overly conservative (ordering
more than necessary), the solver may produce
sub-optimal schedules.
If insufficiently conservative, data races may occur
despite passing the DRF verifier.

\smallskip\noindent
\textbf{CT-3: Cache model simplification.}
The reuse-distance model~\cite{SEED_3} assumes LRU
replacement.
Blackwell's L2 may use a different eviction policy;
if so, the prediction error
$\varepsilon_{\mathrm{miss}}$ may exceed stated
bounds~\cite{ARCH_BW, SEED_3}.

\smallskip\noindent
\textbf{CT-4: $\texttt{num\_warps}$ abstraction.}
\TileIR{} does not expose $\texttt{num\_warps}$ in
CUDA~13.1~\cite{NV_BLOG_TILE};
only an occupancy hint is available.
The warp assignment $W(v)$ in $\mathcal{S}$ is
therefore a logical mapping whose fidelity to physical
scheduling depends on compiler
behaviour \inference{}.
\end{definition}

\begin{figure}[t]
\centering
\begin{BVerbatim}[fontsize=\scriptsize]
+===========================================================+
|  Threat-to-Validity Summary Matrix                        |
+===========================================================+
|                                                           |
|  ID   | Category | Severity | Mitigation              |   |
|  ------+----------+----------+-------------------------+  |
|  IT-1 | Internal | HIGH     | Report compiler version; |  |
|       |          |          | normalize against cuBLAS |  |
|  IT-2 | Internal | MEDIUM   | Use PTX direct path;     |  |
|       |          |          | bypass Triton where poss.|  |
|  IT-3 | Internal | HIGH     | Calibrate; label UNVER.; |  |
|       |          |          | sensitivity analysis     |  |
|  IT-4 | Internal | MEDIUM   | Separate configs per cc  |  |
|  IT-5 | Internal | LOW      | Report eps; use eps=0    |  |
|       |          |          | for small-II kernels     |  |
|  ------+----------+----------+-------------------------+  |
|  ET-1 | External | MEDIUM   | Extend suite post-publ.  |  |
|  ET-2 | External | LOW      | Scope clearly stated     |  |
|  ET-3 | External | MEDIUM   | Characterize solver fail |  |
|  ------+----------+----------+-------------------------+  |
|  CT-1 | Constr.  | HIGH     | Mark multi-nest INFER.   |  |
|  CT-2 | Constr.  | MEDIUM   | Validate DRF on hardware |  |
|  CT-3 | Constr.  | MEDIUM   | Measure eviction policy  |  |
|  CT-4 | Constr.  | LOW      | Occupancy hint sweep     |  |
|  ------+----------+----------+-------------------------+  |
|                                                           |
|  Sources anchoring threat identification:                 |
|    IT-1, IT-3, IT-4: ARCH_BW                              |
|    IT-2, CT-1:       OPT_PIPE                             |
|    CT-2, CT-4:       NV_BLOG_TILE                         |
|    ET-1:             NV_workloads                          |
|    CT-3:             SEED_3, ARCH_BW                       |
+===========================================================+
\end{BVerbatim}
\caption{Threat-to-validity summary with severity
  ranking and primary mitigation strategy.
  HIGH-severity threats (IT-1, IT-3, CT-1)
  require active mitigation in the evaluation
  design.}
\label{fig:threat-matrix}
\end{figure}


% ====================================================================
%  SECTION 21 — KNOWN LIMITATIONS
% ====================================================================
\section{Known Limitations}
\label{sec:limitations}

This section enumerates the theoretical and practical
boundaries of the \BUTA{} model as defined in
Parts~1--3.
Each limitation is traced to a specific golden source
and, where applicable, linked to a gap (G1--G9).

\subsection{Scheduling Scope Limitations}
\label{sec:limitations:sched}

\begin{definition}[Scheduling
  Limitations~\cite{OPT_PIPE}]
\label{def:sched-limits}
The schedule constraint system $\mathcal{S}$
(Part~2) inherits the following restrictions from the
OPT\_PIPE formulation:
\begin{enumerate}[label=(SL-\arabic*)]
  \item \textbf{Singly-nested loops only.}
    The ILP+SMT formulation models one software-pipelined
    loop body with a single initiation
    interval~$\II$~\cite{OPT_PIPE}.
    Multiply-nested tile loops (e.g., outer tiling
    over batch/head dimensions in multi-head attention)
    require hierarchical scheduling not supported by
    the current model.
  \item \textbf{Orthogonal optimisations excluded.}
    Memory layout selection and register allocation are
    not handled within $\mathcal{S}$;
    they are treated as independent pre-decisions or
    addressed by the joint formulation (gap~G4, Part~3,
    \cref{sec:solver:joint}), which is
    \inference{} and not yet
    validated~\cite{OPT_PIPE}.
  \item \textbf{Fixed dependence graph.}
    The graph $G = (V, E)$ is constructed statically
    from the tile loop body.
    Data-dependent control flow (e.g., conditional
    masking, early exit) cannot be modelled as
    static dependencies~\cite{OPT_PIPE}.
\end{enumerate}
\end{definition}


\subsection{Layout Representation Scope Limitations}
\label{sec:limitations:layout}

\begin{definition}[Layout
  Limitations~\cite{SEED_1, SEED_2, SEED_4}]
\label{def:layout-limits}
The layout morphism space $\mathcal{L}$ (Part~2)
has three inherent boundaries:
\begin{enumerate}[label=(LL-\arabic*)]
  \item \textbf{$\Ftwo$ linearity restriction.}
    Non-linear swizzle patterns (XOR-based bank-conflict
    avoidance) are not representable as $\Ftwo$
    matrices~\cite{SEED_1}.
    The efficient computational path
    $\mathcal{L}^{\Ftwo}$ is unavailable for such
    patterns; the \ISL{} layer is
    required~\cite{SEED_2}.
  \item \textbf{Tractable layout subset.}
    The categorical characterisation~\cite{SEED_4}
    defines a strict subset of all possible layouts
    as ``tractable.''
    Layouts that are not decomposable into
    $\mathbf{Nest}$ morphisms cannot be verified at
    Layer~3 of the pipeline; their correctness
    relies on the \ISL{} layer alone.
  \item \textbf{Unestablished embeddings.}
    The embedding $\phi$ (PO-1, gap~G1) and functor
    $F$ (PO-2, gap~G2) are \inference{} constructs
    whose correctness has not been formally
    proved~\cite{SEED_1, SEED_2, SEED_4}.
    Until established, cross-layer layout verification
    carries residual uncertainty.
  \item \textbf{FP4/FP6 layouts uncharacterised.}
    Precision-specific layout spaces for FP4 and FP6
    are not documented in the golden sources;
    the valid tile shapes and swizzle patterns for
    these precisions are \unverified{}
    (gap~G8)~\cite{ARCH_BW, SEED_1}.
\end{enumerate}
\end{definition}


\subsection{Cache Model Scope Limitations}
\label{sec:limitations:cache}

\begin{definition}[Cache Model
  Limitations~\cite{SEED_3}]
\label{def:cache-limits}
The cache traffic function $\mathcal{C}$
(Part~2) has the following scope boundaries:
\begin{enumerate}[label=(CL-\arabic*)]
  \item \textbf{Flash Attention-only validation.}
    The reuse-distance model has been validated
    exclusively for Flash Attention on GB10
    (cc\,12.0, 128\,KB \SMEM{})~\cite{SEED_3}.
    All claims for GEMM, convolution, or B200 are
    \inference{} (gap~G5).
  \item \textbf{L1 excluded for streaming.}
    The model treats L1 as negligible for
    streaming tile
    patterns~\cite{SEED_3}.
    For non-streaming patterns where tile reuse is
    significant (e.g., GEMM with large tiles), L1 may
    contribute meaningfully; the model does not
    capture this.
  \item \textbf{LRU assumption.}
    The reuse-distance formalism~\cite{SEED_3}
    assumes LRU replacement.
    Actual Blackwell L2 eviction policy is not fully
    documented~\cite{ARCH_BW};
    deviation from LRU degrades prediction accuracy.
  \item \textbf{Tropical semiring \inference{}.}
    The $(\min, +)$ semiring structure on
    $\mathcal{C}$ (Part~2) is not established in the
    golden sources.
    Its validity depends on the assumption of
    independent data footprints between composed tile
    programs; shared footprints break additive
    decomposition.
\end{enumerate}
\end{definition}


\subsection{Platform and Toolchain Scope}
\label{sec:limitations:platform}

\begin{definition}[Platform
  Limitations~\cite{NV_BLOG_TILE, ARCH_BW}]
\label{def:platform-limits}
\begin{enumerate}[label=(PL-\arabic*)]
  \item \textbf{Blackwell-only.}
    \TileIR{} targets only compute capability
    10.x and 12.x~\cite{NV_BLOG_TILE}.
    The entire \BUTA{} model---seven objects and all
    invariants---is parameterised by Blackwell-specific
    constants.
    Extension to other architectures (Hopper, future
    generations) requires re-parameterisation at minimum
    and potentially structural changes.
  \item \textbf{$\texttt{num\_warps}$ not exposed.}
    The \TileIR{} backend in CUDA~13.1 does not expose
    $\texttt{num\_warps}$~\cite{NV_BLOG_TILE}.
    The warp assignment $W(v)$ is a logical construct
    within $\mathcal{S}$; its mapping to physical
    resources is compiler-mediated.
  \item \textbf{Register spilling.}
    Large reduction dimensions may cause register
    spilling in the \TileIR{}
    backend~\cite{NV_BLOG_TILE}.
    The scheduling model does not account for spill
    costs, which may degrade actual performance
    relative to solver-predicted $\II$.
  \item \textbf{Small-GEMM performance.}
    Small GEMM performance on the \TileIR{} backend
    is reported as poor~\cite{NV_BLOG_TILE};
    not all Triton operations are implemented.
    Benchmarks at small problem sizes may not reflect
    the model's capabilities.
\end{enumerate}
\end{definition}


% ====================================================================
%  SECTION 22 — RISK ASSESSMENT AND MITIGATION
% ====================================================================
\section{Risk Assessment and Mitigation}
\label{sec:risk}

This section provides a severity-ranked assessment of
the nine identified gaps (G1--G9) and defines a phased
mitigation timeline.

\subsection{Gap Severity Ranking}
\label{sec:risk:ranking}

\begin{definition}[Severity
  Classification]
\label{def:severity}
Each gap is assigned a severity level based on two
dimensions: (i)~\emph{impact on model soundness}
(whether the gap invalidates a proof obligation or
invariant) and (ii)~\emph{feasibility of resolution}
(whether the gap is resolvable with available tools
and hardware).
The classification uses three levels:
\begin{enumerate}[label=(\roman*)]
  \item \textbf{Critical}: gap blocks one or more
    proof obligations;
    model soundness is conditional on resolution.
  \item \textbf{Major}: gap limits model applicability
    or accuracy for specific workloads/configurations;
    core model remains sound.
  \item \textbf{Minor}: gap affects peripheral model
    features or calibration precision; core results
    are not sensitive to resolution.
\end{enumerate}
\end{definition}

\begin{figure}[t]
\centering
\begin{BVerbatim}[fontsize=\scriptsize]
+===========================================================+
|  Gap Severity Ranking                                     |
+===========================================================+
|                                                           |
|  Gap  | Severity  | Blocks PO | Sources            | Ph  |
|  ------+-----------+-----------+--------------------+-----+
|  G1   | CRITICAL  | PO-1,PO-6 | SEED_1, SEED_2     | P1  |
|  G2   | CRITICAL  | PO-2,PO-6 | SEED_4, SEED_2     | P1  |
|  G7   | CRITICAL  | PO-5,PO-6 | NV_BLOG_TILE,      | P1  |
|       |           |           | OPT_PIPE           |     |
|  ------+-----------+-----------+--------------------+-----+
|  G3   | MAJOR     | (PO-3)    | ARCH_BW,           | P2  |
|       |           |           | NV_BLOG_TILE       |     |
|  G4   | MAJOR     | ---       | OPT_PIPE, SEED_1,  | P2  |
|       |           |           | SEED_2, ARCH_BW    |     |
|  G5   | MAJOR     | PO-4      | SEED_3, ARCH_BW,   | P2  |
|       |           |           | NV_workloads       |     |
|  G6   | MAJOR     | PO-3      | NV_BLOG_TILE,      | P2  |
|       |           |           | OPT_PIPE, ARCH_BW  |     |
|  ------+-----------+-----------+--------------------+-----+
|  G8   | MINOR     | ---       | ARCH_BW, SEED_1,   | P3  |
|       |           |           | NV_BLOG_TILE       |     |
|  G9   | MINOR     | ---       | OPT_PIPE, ARCH_BW  | P3  |
|  ------+-----------+-----------+--------------------+-----+
|                                                           |
|  PO dependency:                                           |
|    PO-6 (end-to-end) requires PO-1..PO-5                  |
|    G1 -> PO-1 -> PO-6    [layout foundation]              |
|    G2 -> PO-2 -> PO-6    [layout foundation]              |
|    G7 -> PO-5 -> PO-6    [memory ordering]                |
|    G5 -> PO-4             [cache fidelity]                |
|    G6 -> PO-3             [schedule soundness]            |
|                                                           |
|  Ph = Phase assignment (P1 = months 1-3, P2 = 4-8,       |
|                         P3 = 9-12)                        |
+===========================================================+
\end{BVerbatim}
\caption{Gap severity ranking with proof obligation
  dependencies.
  Critical gaps (G1, G2, G7) block the capstone
  proof obligation PO-6 and must be resolved in
  Phase~1.
  Major gaps (G3--G6) affect model applicability.
  Minor gaps (G8, G9) affect calibration precision.}
\label{fig:gap-ranking}
\end{figure}


\subsection{Mitigation Timeline}
\label{sec:risk:timeline}

\begin{definition}[Phased Mitigation Plan]
\label{def:mitigation-plan}
The nine gaps are resolved in three phases aligned
with proof obligation dependencies:

\smallskip\noindent
\textbf{Phase~1 (Months 1--3): Foundation.}
Resolve the three critical gaps that block PO-6:
\begin{enumerate}[label=(P1.\arabic*)]
  \item \textbf{G1} ($\Ftwo \to \ISL{}$ embedding):
    constructive proof of $\phi$ injectivity and
    composition preservation using \ISL{} library
    automated verification on all Triton test
    layouts~\cite{SEED_1, SEED_2}.
    Deliverable: formal proof document and
    \ISL{}-based checker (artifact~A2).
  \item \textbf{G2} ($\mathbf{Nest} \to \ISL{}$
    functor):
    define $F$ on objects and morphisms;
    extend SEED\_4 Python codebase~\cite{SEED_4} to
    emit \ISL{} relations;
    validate against CUTLASS test suite and Algorithm~1
    of~\cite{SEED_2}.
    Deliverable: functor implementation and
    compatibility test suite.
  \item \textbf{G7} (token ordering integration):
    formalise token DAG as additional edges in
    $G$~\cite{NV_BLOG_TILE, OPT_PIPE};
    implement DRF verifier (Part~3,
    \cref{def:drf-verify});
    encode in SMT formulation.
    Deliverable: token-augmented solver (artifact~A1).
\end{enumerate}

\smallskip\noindent
\textbf{Phase~2 (Months 4--8): Breadth.}
Resolve the four major gaps that affect model
applicability:
\begin{enumerate}[label=(P2.\arabic*)]
  \item \textbf{G3} (\TMEM{} lifecycle model):
    formalise state machine
    (Part~2, \cref{def:tmem-lifecycle}) as resource
    constraint in solver;
    calibrate allocation latency via
    microbenchmarks~\cite{ARCH_BW}.
    Deliverable: \TMEM{}-aware solver extension.
  \item \textbf{G4} (joint layout-schedule):
    implement enumeration strategy (Part~3,
    \cref{def:joint-formulation}) for $\Ftwo$ linear
    layouts~\cite{SEED_1};
    benchmark against separate optimisation.
    Deliverable: joint solver module (artifact~A1).
  \item \textbf{G5} (generalised cache model):
    extend reuse-distance model~\cite{SEED_3} to
    dense GEMM with workload
    classification~\cite{NV_workloads};
    calibrate on B200 ($C_{\mathrm{L2}} \approx 65$\,MB,
    228\,KB \SMEM{})~\cite{ARCH_BW}.
    Deliverable: parameterised cache analyser
    (artifact~A3).
  \item \textbf{G6} (2CTA scheduling):
    implement $\Phi_{\mathrm{2cta}}$ extension
    (Part~3, \cref{def:2cta-clauses})~\cite{NV_BLOG_TILE};
    microbenchmark TPC-internal sync
    latency~\cite{ARCH_BW};
    validate on Flash Attention with 2CTA.
    Deliverable: 2CTA-capable solver.
\end{enumerate}

\smallskip\noindent
\textbf{Phase~3 (Months 9--12): Precision.}
Resolve the two minor gaps that refine calibration:
\begin{enumerate}[label=(P3.\arabic*)]
  \item \textbf{G8} (precision-dependent layouts):
    empirical characterisation of FP4/FP6 layout
    requirements via \tcgen{} microbenchmarks on
    B200~\cite{ARCH_BW};
    extend $\mathcal{L}(p)$ with validated
    constraints~\cite{SEED_1}.
    Deliverable: precision-parameterised layout library.
  \item \textbf{G9} (calibrated cost vector):
    full cost table
    $\mathbf{c}(\mathrm{op}, p, T)$ for all
    precisions and tile sizes via PTX microbenchmark
    suite~\cite{ARCH_BW};
    ZLP normalisation validated~\cite{OPT_PIPE}.
    Deliverable: calibrated cost vectors
    (artifact~A4 data).
\end{enumerate}
\end{definition}


\subsection{Contingency Plans}
\label{sec:risk:contingency}

\begin{definition}[Contingency Strategies]
\label{def:contingency}
For each severity class, a fallback strategy is
defined:

\smallskip\noindent
\textbf{Critical gap failure.}
If any of G1, G2, or G7 proves intractable:
\begin{enumerate}[label=(\alph*)]
  \item For G1/G2: reduce the layout hierarchy to a
    two-layer system ($\Ftwo$ + \ISL{} without
    categorical verification).
    The model loses the semantic foundation
    of~\cite{SEED_4} but retains operational
    correctness via the \ISL{} layer~\cite{SEED_2}.
    PO-6 is weakened to a two-layer simulation.
  \item For G7: treat token ordering as
    \emph{pre-validated} by the \TileIR{} compiler
    rather than verified within $\mathcal{S}$.
    PO-5 is replaced by a trust assumption on the
    \TileIR{} memory model~\cite{NV_BLOG_TILE}.
\end{enumerate}

\smallskip\noindent
\textbf{Major gap failure.}
If G5 (cache generalisation) fails: restrict
$\mathcal{C}$ applicability to Flash Attention
(per~\cite{SEED_3}); report GEMM/convolution
cache predictions as \unverified{}.
If G6 (2CTA) fails: restrict solver to single-CTA
mode; report 2CTA results using expert baselines
only~\cite{NV_BLOG_TILE}.

\smallskip\noindent
\textbf{Minor gap failure.}
If G8/G9 (FP4/FP6 layouts, full cost table) fail:
restrict benchmarks to FP16/BF16/FP8 precisions for
which calibration data exists~\cite{ARCH_BW};
FP4/FP6 results labelled \unverified{}.
\end{definition}

\begin{figure}[t]
\centering
\begin{BVerbatim}[fontsize=\scriptsize]
+===========================================================+
|  Mitigation Timeline (12-month plan)                      |
+===========================================================+
|                                                           |
|  Month: 1   2   3   4   5   6   7   8   9  10  11  12    |
|         |----PHASE 1----|----PHASE 2--------|--PHASE 3-|  |
|                                                           |
|  G1 (F2->ISL) :  [=====]                                  |
|    PO-1 proof      ^                                      |
|                                                           |
|  G2 (Cat->ISL):  [=====]                                  |
|    PO-2 proof      ^                                      |
|                                                           |
|  G7 (Tokens)  :  [=====]                                  |
|    PO-5 proof      ^                                      |
|                                                           |
|  G3 (TMEM)    :           [========]                      |
|    Calibrate               ^                              |
|                                                           |
|  G4 (Joint)   :           [========]                      |
|    Solver ext              ^                              |
|                                                           |
|  G5 (Cache)   :           [===========]                   |
|    PO-4 valid              ^                              |
|                                                           |
|  G6 (2CTA)    :           [========]                      |
|    PO-3 valid              ^                              |
|                                                           |
|  G8 (Prec.)   :                          [======]         |
|    FP4/FP6 char.                          ^               |
|                                                           |
|  G9 (Costs)   :                          [======]         |
|    Full cost tbl                          ^               |
|                                                           |
|  PO-6 (E2E)   :                                [===]     |
|    Capstone                                      ^        |
|                                                           |
|  Milestones:                                              |
|    M3:  All critical gaps resolved; PO-1,2,5 proved       |
|    M8:  All major gaps resolved; PO-3,4 validated         |
|    M12: Full model validated; PO-6 established            |
|         Artifact release with all benchmarks              |
+===========================================================+
\end{BVerbatim}
\caption{Twelve-month mitigation timeline.
  Critical gaps (G1, G2, G7) are resolved in
  Phase~1 (months~1--3) to establish the
  foundational proof obligations.
  Major gaps (G3--G6) are resolved in Phase~2
  (months~4--8).
  Minor gaps (G8, G9) and the capstone PO-6 are
  completed in Phase~3 (months~9--12).}
\label{fig:timeline}
\end{figure}


\subsection{Risk-Adjusted Confidence Assessment}
\label{sec:risk:confidence}

\begin{proposition}[Model Viability
  Assessment \inference{}]
\label{prop:viability}
The \BUTA{} model is viable under the following
conditions, each linked to gap resolution and
evidence from the golden sources:
\begin{enumerate}[label=(\roman*)]
  \item \emph{Layout foundation} (G1, G2):
    the constructive nature of both $\phi$ and $F$
    (domain-restricted \ISL{} relations over binary
    or natural-number domains) makes formal proofs
    tractable using existing \ISL{} library
    capabilities~\cite{SEED_1, SEED_2, SEED_4}.
    Risk: \textbf{LOW}.
  \item \emph{Scheduling optimality} (G6, G9):
    the Twill system~\cite{OPT_PIPE} demonstrates
    that ILP+SMT solvers rediscover expert schedules
    on Blackwell, establishing the approach's
    feasibility.
    The 2CTA extension adds complexity but preserves
    the core SAT+ILP structure.
    MMA latency constancy~\cite{ARCH_BW} simplifies
    cost calibration.
    Risk: \textbf{LOW-MEDIUM}.
  \item \emph{Cache model generalisation} (G5):
    the SEED\_3 model is validated for one workload
    on one platform~\cite{SEED_3}.
    Generalisation is the highest-uncertainty
    component: non-streaming patterns may require
    fundamentally different modelling.
    The HARP taxonomy~\cite{NV_workloads} provides
    structural guidance for pattern-specific
    sub-models.
    Risk: \textbf{MEDIUM-HIGH}.
  \item \emph{Token ordering} (G7):
    the \TileIR{} memory model~\cite{NV_BLOG_TILE}
    is well-defined as a partial order;
    encoding as dependency edges in the SMT
    formulation is a direct extension of
    existing techniques~\cite{OPT_PIPE}.
    Risk: \textbf{LOW}.
  \item \emph{TMEM lifecycle} (G3):
    partially documented~\cite{ARCH_BW};
    microbenchmark calibration is required.
    Risk: \textbf{MEDIUM} (documentation-dependent).
\end{enumerate}
\end{proposition}


% ====================================================================
%  SECTION 23 — CONCLUSION
% ====================================================================
\section{Conclusion}
\label{sec:conclusion}

This four-part proposal has defined the Blackwell
Unified Tile Algebra (\BUTA{}): a compositional formal
model that jointly captures layout transformations as
algebraic morphisms (grounded in the $\Ftwo$ binary
matrix formalism of~\cite{SEED_1}, the \ISL{}
quasi-affine relations of~\cite{SEED_2}, and the
categorical foundations of~\cite{SEED_4}); pipeline
scheduling as constraint optimisation (extending the
ILP+SMT formulation of~\cite{OPT_PIPE} with
Blackwell-specific parameterisation
from~\cite{ARCH_BW} and 2CTA constraints
from~\cite{NV_BLOG_TILE}); data movement as
token-ordered descriptor relations (formalising the
\TileIR{} memory model of~\cite{NV_BLOG_TILE} over
the Blackwell memory hierarchy
of~\cite{ARCH_BW}); and cache traffic as analytic
reuse models (building on the reuse-distance formalism
of~\cite{SEED_3} with workload classification
from~\cite{NV_workloads}).

The model comprises seven mathematical objects whose
composition is governed by 18~invariants
(INV-L1--L4, INV-D1--D4, INV-S1--S5, INV-T1--T4),
three objective functions (OBJ-1--OBJ-3), and six
proof obligations (PO-1--PO-6).
Nine research gaps (G1--G9) have been identified,
severity-ranked, and assigned to a three-phase
mitigation timeline.
The evaluation strategy validates the model against
expert baselines from~\cite{OPT_PIPE}, hardware
measurements on B200 and GB10~\cite{ARCH_BW, SEED_3},
and ground-truth layout libraries
from~\cite{SEED_1, SEED_2, SEED_4}.

The core question posed in Part~1---whether a
compositional formal model can jointly capture layout
algebra, scheduling, data movement, and cache traffic
for Blackwell while preserving semantic
correctness---has been answered affirmatively at the
specification level.
The constructive components (solver encodings, layout
pipeline, cache analyser, calibration plan) have been
defined with sufficient precision for implementation.
The proof obligations establish the formal agenda that
must be discharged for soundness; the contingency
plans ensure that partial resolution still yields a
useful, if reduced, model.


% ====================================================================
%  SUMMARY
% ====================================================================
\section*{Part~4 Summary}

This final part has defined the evaluation strategy
(\cref{sec:eval}) with three validation axes:
solver output vs.\ expert schedules~\cite{OPT_PIPE},
cache model fidelity~\cite{SEED_3, ARCH_BW}, and
layout correctness~\cite{SEED_1, SEED_2, SEED_4}.
The benchmark suite (\cref{sec:bench}) comprises five
kernel families covering mixed-reuse, compute-bound,
and memory-bound workloads~\cite{NV_workloads} on
both B200 and GB10~\cite{ARCH_BW}.
The reproducibility plan (\cref{sec:repro}) defines
five artifact components, a solver-only reproduction
path, and artifact evaluation
criteria~\cite{OPT_PIPE, NV_BLOG_TILE}.
Threats to validity (\cref{sec:threats}) enumerate
eleven factors across internal, external, and
construct categories, with mitigations anchored to
specific golden sources~\cite{ARCH_BW, OPT_PIPE,
NV_BLOG_TILE, SEED_3, NV_workloads}.
Known limitations (\cref{sec:limitations}) trace
twelve model boundaries to the golden
sources~\cite{OPT_PIPE, SEED_1, SEED_2, SEED_3,
SEED_4, NV_BLOG_TILE, ARCH_BW}.
The risk assessment (\cref{sec:risk}) severity-ranks
nine gaps, assigns them to a twelve-month phased
timeline, and defines contingency strategies for
each severity class.

Across all four parts, every non-trivial claim has been
cited to one of the eight golden sources; claims
derived by logical composition of source-grounded
constructs are marked \inference{}; claims not
supported by the sources are marked \unverified{}.


% ====================================================================
%  SOURCE AUDIT
% ====================================================================
\section*{Source Audit (Part~4)}
\label{sec:audit-p4}

\noindent
All eight golden sources are cited in this part.
\Cref{tab:audit-p4} records usage.

\begin{table}[H]
\centering
\caption{Part~4 source audit: usage and anchors.}
\label{tab:audit-p4}
\footnotesize
\begin{tabular}{@{}p{1.3cm}p{6.0cm}@{}}
\toprule
\textbf{ID} & \textbf{Used for (anchor)} \\
\midrule
ARCH\_BW &
  Platform configuration (B200 148~SMs, 228\,KB \SMEM{},
  \TMEM{} 256\,KB, L2 ${\sim}$65\,MB;
  GB10 128\,KB \SMEM{}, 48 warps);
  throughput metric baseline; dense GEMM lag ($4\times$)
  as threat IT-1; 58\% cache-miss latency reduction;
  undocumented microarchitecture (IT-3);
  precision lattice FP4/FP6 (gap G8);
  cost vector calibration (gap G9);
  cross-platform parameterisation;
  benchmark hardware anchor. \\
\addlinespace
OPT\_PIPE &
  Schedule quality metric ($\II$ ratio); Twill
  rediscovery as baseline; cross-generation schedule
  distinction; solve time prediction ($< 60$\,s);
  singly-nested loop limitation (SL-1, CT-1);
  orthogonal optimisation exclusion (SL-2);
  Triton code generation issues (IT-2);
  ILP+SMT solver requirements;
  ZLP rounding sensitivity (IT-5);
  solver artifact specification;
  gap G4 joint layout-schedule;
  gap G6 2CTA extension anchor;
  gap G9 cost calibration anchor. \\
\addlinespace
NV\_BLOG\_ TILE &
  \TileIR{} compilation target; 2CTA
  (\texttt{num\_ctas=2}) in evaluation;
  Triton-to-\TileIR{} baseline comparison;
  $\texttt{num\_warps}$ limitation (CT-4, PL-2);
  register spilling (PL-3);
  small-GEMM performance (PL-4);
  CUDA $\geq 13.1$ / PTX $> 9.0$ requirements;
  token ordering (CT-2); \tma{} descriptor types
  (im2col for convolution benchmark B5);
  Blackwell-only scope (ET-2, PL-1);
  gap G7 token integration anchor. \\
\addlinespace
NV\_work\-loads &
  Benchmark kernel HARP classification (mixed-reuse,
  compute-bound);
  external threat ET-1 (workload diversity);
  cache model Tier~3 classification;
  mixed-precision workload characterisation (B4);
  gap G5 generalisation guidance;
  workload descriptor conditioning. \\
\addlinespace
SEED\_1 &
  Layout verification criterion (a): $\Ftwo$ layer
  testing against Triton test suite;
  linearity restriction (LL-1);
  FP4/FP6 layout gap (LL-4, G8);
  layout enumeration for joint solver (G4);
  artifact A2 $\Ftwo$ library;
  gap G1 mitigation (Phase~1). \\
\addlinespace
SEED\_2 &
  Layout verification criterion (b): \ISL{} layer
  testing via Algorithm~1;
  swizzle representation for convolution (B5);
  cross-layer consistency ($\phi$ validation);
  artifact A2 \ISL{} module;
  gap G1 target representation;
  gap G2 cross-validation source. \\
\addlinespace
SEED\_3 &
  Cache validation Tiers~1--3 (67\% miss reduction
  reproduction, cross-platform, generalisation);
  miss prediction error metric $\varepsilon_{\mathrm{miss}}$;
  Flash Attention-only limitation (CL-1);
  L1 negligible for streaming (CL-2);
  LRU assumption (CL-3);
  artifact A3 cache analyser;
  gap G5 anchor;
  GB10 platform anchor. \\
\addlinespace
SEED\_4 &
  Layout verification criterion (c): categorical
  tractability check against CUTLASS test suite;
  tractable layout subset limitation (LL-2);
  functor $F$ gap G2 (Phase~1 mitigation);
  artifact A2 categorical module;
  cross-layer consistency ($F$ validation). \\
\bottomrule
\end{tabular}
\end{table}


% ====================================================================
%  BIBLIOGRAPHY
% ====================================================================
\begin{thebibliography}{8}

\bibitem{ARCH_BW}
A.~Jarmusch and S.~Chandrasekaran,
``Microbenchmarking NVIDIA's Blackwell Architecture:
An In-Depth Architectural Analysis,''
\emph{arXiv:2512.02189v1}, Dec.~2025.
\url{https://arxiv.org/html/2512.02189v1}

\bibitem{OPT_PIPE}
K.~Soi, A.~Venkat, T.~Nowicki, and
G.~Biros,
``Optimal Software Pipelining and Warp
Specialization for Tensor Core GPUs,''
\emph{arXiv:2512.18134v1}, Dec.~2025.
\url{https://arxiv.org/html/2512.18134v1}

\bibitem{NV_BLOG_TILE}
NVIDIA,
``Advancing GPU Programming with the CUDA Tile IR
Backend for OpenAI Triton,''
\emph{NVIDIA Developer Blog}, Jan.~2026.
\url{https://developer.nvidia.com/blog/advancing-gpu-programming-with-the-cuda-tile-ir-backend-for-openai-triton/}

\bibitem{NV_workloads}
S.~Garg, M.~Pellauer, and T.~Krishna,
``HARP: A Taxonomy for Heterogeneous and
Hierarchical Processors for Mixed-Reuse Workloads,''
\emph{arXiv:2502.13113}, Feb.~2025.
\url{https://arxiv.org/html/2502.13113}

\bibitem{SEED_1}
J.~Zhou, M.~Lezcano, A.~Goucher, et~al.,
``Linear Layouts: Robust Code Generation of Efficient
Tensor Computation Using $\mathbb{F}_2$,''
\emph{arXiv:2505.23819v3}, Oct.~2025.
\url{https://arxiv.org/html/2505.23819v3}

\bibitem{SEED_2}
S.~Bhaskaracharya, N.~Acharya, M.~Hagedorn,
and S.~Grover,
``Modeling Layout Abstractions Using Integer Set
Relations,''
\emph{arXiv:2511.10374v1}, Nov.~2025.
\url{https://arxiv.org/html/2511.10374v1}

\bibitem{SEED_3}
Z.~Zhu, J.~Pan, and Y.~Ding,
``Sawtooth Wavefront Reordering: Enhanced CuTile
FlashAttention on NVIDIA GB10,''
\emph{arXiv:2601.16032v2}, Jan.~2026.
\url{https://arxiv.org/html/2601.16032v2}

\bibitem{SEED_4}
R.~Carlisle, A.~Shah, H.~Stern, and
D.~VanKoughnett,
``Categorical Foundations for CuTe Layouts,''
\emph{arXiv:2601.05972v1}, Jan.~2026.
\url{https://arxiv.org/pdf/2601.05972v1}

\end{thebibliography}

\end{document}
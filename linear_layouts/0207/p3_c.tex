% ============================================================
%  BUTA — Blackwell Unified Tile Algebra
%  Part 3: Solver Encodings, Layout Tooling,
%          Data-Movement Pipeline, Cache Analysis,
%          and Calibration Plan
% ============================================================
\documentclass[10pt,twocolumn]{article}

% ---- Geometry & Typography ----
\usepackage[margin=0.85in,columnsep=0.25in]{geometry}
\usepackage[T1]{fontenc}
\usepackage{lmodern}
\usepackage{microtype}

% ---- Mathematics ----
\usepackage{amsmath,amssymb,amsthm,mathtools}

% ---- Figures & Tables ----
\usepackage{graphicx,booktabs,array,multirow}
\usepackage{fancyvrb}
\usepackage{float}

% ---- Cross-references & Links ----
\usepackage[colorlinks,citecolor=blue!70!black,
            linkcolor=blue!60!black,urlcolor=blue!50!black]{hyperref}
\usepackage[capitalise,noabbrev]{cleveref}

% ---- Misc ----
\usepackage{xcolor}
\usepackage{enumitem}
\setlist{nosep,leftmargin=*}

% ---- Theorem-like environments ----
\theoremstyle{definition}
\newtheorem{definition}{Definition}[section]
\newtheorem{invariant}[definition]{Invariant}
\newtheorem{remark}[definition]{Remark}
\newtheorem{procedure}[definition]{Procedure}
\theoremstyle{plain}
\newtheorem{proposition}[definition]{Proposition}
\newtheorem{obligation}[definition]{Proof Obligation}

% ---- Convenience macros (matching Parts 1--2) ----
\newcommand{\BUTA}{\textsc{Buta}}
\newcommand{\Ftwo}{\mathbb{F}_2}
\newcommand{\ZZ}{\mathbb{Z}}
\newcommand{\NN}{\mathbb{N}}
\newcommand{\RR}{\mathbb{R}}
\newcommand{\ISL}{\textsc{ISL}}
\newcommand{\CuTe}{\textsc{CuTe}}
\newcommand{\TileIR}{Tile\,IR}
\newcommand{\TMEM}{\textsc{TMEM}}
\newcommand{\SMEM}{\textsc{SMEM}}
\newcommand{\tcgen}{\texttt{tcgen05}}
\newcommand{\tma}{\textsc{TMA}}
\newcommand{\II}{\mathit{II}}
\newcommand{\reuse}{d_{\mathrm{reuse}}}
\newcommand{\unverified}{\textsc{[unverified]}}
\newcommand{\inference}{\textsc{[inference]}}

% ---- Continue section numbering from Part 2 ----
\setcounter{section}{10}

% ---- Title ----
\title{%
  \BUTA{}: Blackwell Unified Tile Algebra\\[4pt]
  \large Part~3 --- Solver Encodings, Layout Tooling,\\
         Data-Movement Pipeline, Cache Analysis,\\
         and Calibration Plan}

\author{%
  \textit{[Authors redacted for review]}\\
  Target architecture: NVIDIA Blackwell (B200 cc\,10.0;
  Grace-Blackwell GB10 cc\,12.0)\\
  Toolchain: CUDA ${>}\,13.0$ (prefer 13.1+),
  PTX ${>}\,9.0$ (strict)}

\date{February 2026}

\begin{document}
\maketitle

% ====================================================================
%  OVERVIEW
% ====================================================================
\noindent
Part~2 defined the seven mathematical objects of
\BUTA{}, their semantic layers, invariants, and proof
obligations.
This part develops the \emph{computational methods}
required to work with \BUTA{}: solver encodings that
translate the abstract constraint system into
executable ILP and SMT formulations
(\cref{sec:solver}); layout tooling that realises the
three-layer layout hierarchy as verifiable computations
(\cref{sec:layout-tool}); data-movement and
token-ordering pipelines that operationalise the
\TileIR{} memory model (\cref{sec:dm-pipeline});
cache traffic analysis procedures that compute
reuse-distance predictions (\cref{sec:cache-pipe});
calibration microbenchmark protocols that fit the
hardware parameters (\cref{sec:calibration}); and
workload characterisation methods that classify
operators for cost-vector conditioning
(\cref{sec:workload-char}).


% ====================================================================
%  SECTION 11 — SOLVER ENCODING METHODS
% ====================================================================
\section{Solver Encoding Methods}
\label{sec:solver}

The schedule constraint system $\mathcal{S}$ (Part~2,
Definition~4.4) is an abstract optimisation problem.
This section specifies its concrete encoding into ILP
and SMT solver inputs, parameterised by the Blackwell
cost vector~\cite{ARCH_BW} and extended with the 2CTA
constraints~\cite{NV_BLOG_TILE}.

\subsection{ILP Encoding for Initiation Interval
            Lower Bound}
\label{sec:solver:ilp}

The ILP formulation determines a lower bound on the
initiation interval $\II$ by relaxing the boolean
schedule variables to continuous
surrogates~\cite{OPT_PIPE}.

\begin{definition}[ILP Variable
  Set~\cite{OPT_PIPE}]
\label{def:ilp-vars}
For dependence graph $G = (V, E)$, define:
\begin{align}
  &\II \in \NN^{+}
    &&\text{(initiation interval)} \notag\\
  &M(v) \in \{0, \dots, \II{-}1\}
    &&\forall\, v \in V
    \text{ (schedule time)} \notag\\
  &\theta(v) \in \{0, \dots, |V|{-}1\}
    &&\forall\, v \in V
    \text{ (stage index)} \notag
\end{align}
The stage index satisfies
$M(v) = M(v) \bmod \II$ with
$\theta(v) = \lfloor M(v) / \II \rfloor$,
decomposing the absolute schedule time into within-period
offset and pipeline
stage~\cite{OPT_PIPE}.
\end{definition}

\begin{definition}[ILP Constraint
  Encoding~\cite{OPT_PIPE, ARCH_BW}]
\label{def:ilp-constraints}
The ILP over $G = (V, E)$ with Blackwell cost
vector $\mathbf{c}(x)$ (Part~2, Definition~7.1) is:
\begin{equation}
  \min\; \II
  \label{eq:ilp-obj}
\end{equation}
subject to, for every $(u, v, d, \delta) \in E$:
\begin{equation}
  M(v) - M(u) + \II \cdot \delta
  \;\geq\;
  d(u,v)
  \label{eq:ilp-dep}
\end{equation}
where $d(u,v) = \mathbf{c}(x)[\mathrm{type}(u)]$ is
the operation latency drawn from the
Blackwell-parameterised cost
vector~\cite{ARCH_BW}.
Additionally, the \emph{resource lower bound}
contributes:
\begin{equation}
  \II \;\geq\;
  \left\lceil
    \frac{\sum_{v \in V_c} d(v)}{R_c}
  \right\rceil
  \quad \forall\, c \in
  \{\text{TC},\, \tma{},\, \text{ALU}\}
  \label{eq:ilp-resource}
\end{equation}
where $V_c \subseteq V$ is the set of operations
requiring functional unit class~$c$ and
$R_c$ is the number of such units per
SM~\cite{OPT_PIPE, ARCH_BW}.
For Blackwell: $R_{\text{TC}} = 1$
(one 5th-generation tensor core per
SM)~\cite{ARCH_BW}.
\end{definition}

\begin{remark}
The ILP provides only a \emph{lower bound} on the
achievable $\II$.
The true optimum may be higher due to resource
conflicts not captured by the aggregate bound in
\cref{eq:ilp-resource}.
The tight solution is obtained from the full SMT
encoding below~\cite{OPT_PIPE}.
\end{remark}


\subsection{SMT Encoding for Joint SWP+WS}
\label{sec:solver:smt}

The SMT formulation encodes the \emph{joint}
software-pipelining and warp-specialisation problem
as a satisfiability query~\cite{OPT_PIPE}.

\begin{definition}[SMT Variable
  Encoding~\cite{OPT_PIPE}]
\label{def:smt-vars}
Given ILP lower bound $\II_{\mathrm{lb}}$, the
SMT solver iterates over candidate values
$\II = \II_{\mathrm{lb}}, \II_{\mathrm{lb}}{+}1,
\dots$ until satisfiable.
For each candidate $\II$, define:
\begin{align}
  &\mathrm{op}[v, i, t] \in \{0, 1\}
    && \forall\, v \in V,\; i \in [I],\; t \in [T]
    \notag\\
  &W(v) \in \{0, \dots, N_W{-}1\}
    && \forall\, v \in V
    \notag
\end{align}
where $I$ is the maximum number of in-flight
iterations (bounded by pipeline depth),
$T = I \cdot \II$ is the total schedule horizon,
and $N_W$ is the warp count~\cite{OPT_PIPE}.
The boolean $\mathrm{op}[v, i, t] = 1$ encodes that
operation~$v$ from iteration~$i$ executes at absolute
time~$t$~\cite{OPT_PIPE}.
\end{definition}

\begin{definition}[SMT Clause
  Set~\cite{OPT_PIPE}]
\label{def:smt-clauses}
The constraint set
$\Phi = \Phi_{\mathrm{uniq}} \wedge
\Phi_{\mathrm{dep}} \wedge
\Phi_{\mathrm{fu}} \wedge
\Phi_{\mathrm{res}}$ consists of:

\smallskip\noindent
\textbf{Uniqueness} $\Phi_{\mathrm{uniq}}$:
each operation executes \emph{exactly once} per
iteration:
\begin{equation}
  \forall\, v, i:\quad
  \sum_{t=0}^{T-1} \mathrm{op}[v, i, t] = 1
  \label{eq:smt-uniq}
\end{equation}

\smallskip\noindent
\textbf{Dependency} $\Phi_{\mathrm{dep}}$:
for every $(u, v, d, \delta) \in E$ and iteration~$i$,
if $\mathrm{op}[u, i, t_u] = 1$ and
$\mathrm{op}[v, i{+}\delta, t_v] = 1$ then
$t_v - t_u \geq d(u,v)$~\cite{OPT_PIPE}.

\smallskip\noindent
\textbf{Functional unit exclusivity}
$\Phi_{\mathrm{fu}}$:
for each unit class $c$ and time~$t$:
\begin{equation}
  \sum_{\substack{v \in V_c \\ i \in [I]}}
  \mathrm{op}[v, i, t] \;\leq\; R_c
  \label{eq:smt-fu}
\end{equation}

\smallskip\noindent
\textbf{Resource capacity} $\Phi_{\mathrm{res}}$:
active execution contexts $\leq 4$ per SM at every
time step~\cite{ARCH_BW, OPT_PIPE}, and concurrent
warps $\leq N_{\mathrm{WS}}(x)$
($64$ for cc\,10.0, $48$ for
cc\,12.0)~\cite{ARCH_BW}.
\end{definition}


\subsection{2CTA Extension of the SMT Formulation}
\label{sec:solver:2cta}

Dense MMA on Blackwell requires
$\texttt{num\_ctas} = 2$~\cite{NV_BLOG_TILE}.
The base OPT\_PIPE formulation~\cite{OPT_PIPE}
models single-CTA scheduling; extending it to CTA
pairs is a formal contribution of \BUTA{} (gap~G6).

\begin{definition}[2CTA Variable
  Extension~\cite{NV_BLOG_TILE}]
\label{def:2cta-vars}
Augment the SMT variable set with:
\begin{align}
  &\mathrm{cta}(v) \in \{0, 1\}
    && \forall\, v \in V
    \notag\\
  &\mathrm{shared}(a) \in \{0, 1\}
    && \forall\, a \in \mathrm{Allocs}(\SMEM{})
    \notag
\end{align}
where $\mathrm{cta}(v)$ assigns each operation to a
CTA within the TPC pair and $\mathrm{shared}(a)$
indicates whether \SMEM{} allocation~$a$ is in the
shared operand region accessible by both
CTAs~\cite{NV_BLOG_TILE, ARCH_BW}.
\end{definition}

\begin{definition}[2CTA Constraint
  Clauses~\cite{NV_BLOG_TILE, ARCH_BW}]
\label{def:2cta-clauses}
The 2CTA constraint set $\Phi_{\mathrm{2cta}}$
comprises:

\smallskip\noindent
\textbf{(a) MMA operand spanning.}
For every MMA operation $v_{\mathrm{mma}} \in V$ with
operand-$A$ load $v_A$ and operand-$B$ load $v_B$:
\begin{equation}
  \mathrm{cta}(v_A) + \mathrm{cta}(v_B) = 1
  \label{eq:2cta-span}
\end{equation}
This ensures the two loads originate from different
CTAs, leveraging TPC-internal bandwidth for
operand sharing~\cite{NV_BLOG_TILE, ARCH_BW}.

\smallskip\noindent
\textbf{(b) \SMEM{} capacity with sharing.}
Total \SMEM{} consumption per TPC respects the
aggregate capacity minus the shared region counted
once:
\begin{equation}
  \sum_{\substack{a \in \mathrm{Allocs} \\
    \mathrm{shared}(a)=0}}
  \!\!\mathrm{size}(a)
  \;+\;
  \sum_{\substack{a \in \mathrm{Allocs} \\
    \mathrm{shared}(a)=1}}
  \!\!\mathrm{size}(a)
  \;\leq\;
  2\, C_{\SMEM{}}(x)
  \label{eq:2cta-smem}
\end{equation}
where $C_{\SMEM{}}(10.0) = 228$\,KB and
$C_{\SMEM{}}(12.0) = 128$\,KB per
SM~\cite{ARCH_BW}.
Shared operands are loaded once and read by both
CTAs via the MMA.2SM
path~\cite{ARCH_BW}.

\smallskip\noindent
\textbf{(c) Symmetry breaking.}
Fix $\mathrm{cta}(v_0) = 0$ for one
distinguished operation $v_0 \in V$, pruning
redundant symmetric solutions.

\smallskip\noindent
\textbf{(d) Inter-CTA synchronisation.}
For operations $(u, v)$ assigned to different CTAs
that share a data dependency:
\begin{equation}
  \mathrm{cta}(u) \neq \mathrm{cta}(v)
  \;\Rightarrow\;
  t_v - t_u \;\geq\; d(u,v) + \tau_{\mathrm{sync}}
  \label{eq:2cta-sync}
\end{equation}
where $\tau_{\mathrm{sync}}$ is the TPC-internal
synchronisation latency \unverified{}.
\end{definition}

\begin{remark}
The value of $\tau_{\mathrm{sync}}$ is not
characterised in the golden sources.
Microbenchmark calibration (see \cref{sec:calib:tma})
is required; until measured, solver results using
$\Phi_{\mathrm{2cta}}$ carry an \unverified{}
parameter.
\end{remark}

\begin{figure}[t]
\centering
\begin{BVerbatim}[fontsize=\scriptsize]
+===========================================================+
|  Solver Pipeline: ILP -> SMT -> Schedule                  |
+===========================================================+
|                                                           |
|  INPUT: Dependence graph G = (V, E)                       |
|         Cost vector c(x) [ARCH_BW]                        |
|         Precision p in P  [ARCH_BW]                       |
|         num_ctas in {1, 2} [NV_BLOG_TILE]                 |
|                                                           |
|  PHASE 1: ILP Lower Bound                                 |
|  +--------------------------------------------------+     |
|  | min II  s.t.  M(v)-M(u)+II*delta >= d(u,v)       |     |
|  |               II >= ceil(sum d(v)/R_c) per unit   |     |
|  | Solver: Gurobi / CPLEX / HiGHS                   |     |
|  | Output: II_lb (lower bound)                       |     |
|  +----+---------------------------------------------+     |
|       |                                                   |
|       v  II = II_lb                                       |
|                                                           |
|  PHASE 2: SMT Search                                      |
|  +--------------------------------------------------+     |
|  | LOOP: II = II_lb, II_lb+1, ...                    |     |
|  |   Encode Phi_uniq AND Phi_dep AND Phi_fu          |     |
|  |          AND Phi_res                              |     |
|  |   IF num_ctas=2: AND Phi_2cta [gap G6]            |     |
|  |   IF tokens:     AND Phi_token [gap G7]            |     |
|  |                                                   |     |
|  |   Query: Z3 / cvc5                                |     |
|  |   IF SAT: extract (M, W, II, cta)                 |     |
|  |   IF UNSAT: II := II + 1; continue                |     |
|  +----+---------------------------------------------+     |
|       |                                                   |
|       v                                                   |
|                                                           |
|  PHASE 3: Schedule Extraction                             |
|  +--------------------------------------------------+     |
|  | From satisfying assignment:                       |     |
|  |   M(v)  : time slot per operation                 |     |
|  |   W(v)  : warp assignment per operation            |     |
|  |   II    : initiation interval (optimal)            |     |
|  |   cta(v): CTA assignment (if 2CTA)                |     |
|  |                                                   |     |
|  | Validate against INV-S1..S5 [Part 2]               |     |
|  +--------------------------------------------------+     |
|                                                           |
|  OUTPUT: Feasible Blackwell schedule (M, W, II, cta)      |
+===========================================================+
\end{BVerbatim}
\caption{Three-phase solver pipeline.
  Phase~1 (ILP) lower-bounds
  $\II$~\cite{OPT_PIPE};
  Phase~2 (SMT) searches for a satisfying joint
  SWP+WS assignment with optional 2CTA
  extension~\cite{NV_BLOG_TILE};
  Phase~3 extracts and validates the schedule against
  Blackwell resource bounds~\cite{ARCH_BW}.}
\label{fig:solver-pipeline}
\end{figure}


\subsection{ZLP Normalisation Procedure}
\label{sec:solver:zlp}

The Blackwell cost vector $\mathbf{c}(x)$ contains
real-valued latencies (e.g.,
$c_{\mathrm{MMA}} \approx 11.2$ cycles).
SMT solvers operate most efficiently on integer
constraints.
The Zero-One Linear Program (ZLP) normalisation
of~\cite{OPT_PIPE} converts cost ratios to minimal
positive integers.

\begin{definition}[ZLP
  Normalisation~\cite{OPT_PIPE}]
\label{def:zlp}
Given raw cost vector
$\mathbf{c} = (c_1, \dots, c_k) \in \RR_{>0}^k$:
\begin{enumerate}[label=(\roman*)]
  \item Compute ratios:
    $r_j = c_j / \min_{j'} c_{j'}$
    for all $j$.
  \item Find the smallest $N \in \NN^{+}$ such that
    $|\lfloor N \cdot r_j \rceil - N \cdot r_j|
    < \varepsilon$
    for all~$j$, where $\varepsilon$ is a rounding
    tolerance (default $10^{-3}$).
  \item Set normalised costs:
    $\hat{c}_j = \lfloor N \cdot r_j \rceil$ for
    all~$j$.
\end{enumerate}
The normalised cost vector
$\hat{\mathbf{c}} = (\hat{c}_1, \dots, \hat{c}_k)$
replaces $\mathbf{c}$ in all SMT constraints.
\end{definition}

\begin{proposition}[ZLP Optimality
  Preservation~\cite{OPT_PIPE}]
\label{prop:zlp}
If $\varepsilon = 0$ (exact rationalisation), the ZLP
normalisation preserves optimality: a schedule
$(\hat{M}, \hat{W}, \hat{\II})$ that minimises $\II$
under $\hat{\mathbf{c}}$ also minimises $\II$ under
$\mathbf{c}$.
For $\varepsilon > 0$, the normalised optimum satisfies
$|\hat{\II} - \II^{*}| \leq 1$, where $\II^{*}$ is
the true optimum.
\end{proposition}

\begin{remark}
For Blackwell, the cost vector anchor is
$c_{\mathrm{MMA}} \approx 11.0$--$11.4$ cycles
(approximately constant across FP16 tile
sizes)~\cite{ARCH_BW}.
The constancy of MMA latency across precisions beyond
FP16 is \unverified{} for FP4, FP6, and
FP64 (gap~G9).
\end{remark}


\subsection{Joint Layout-Schedule Encoding}
\label{sec:solver:joint}

Gaps~G4 and~G8 identify the absence of a joint
layout-schedule constraint encoding in which layout
selection variables appear as parameters in scheduling
constraints.

\begin{definition}[Layout Selection
  Variables \inference{} (Gap~G4)]
\label{def:layout-select}
Introduce a layout selection variable
$\ell \in \{1, \dots, K\}$ indexing a finite candidate
set
$\mathcal{L}_{\mathrm{cand}}(p, T) \subseteq
\mathrm{Mor}_{\mathcal{L}}$
of valid layouts for precision~$p$ and tile
shape~$T$~\cite{SEED_1, ARCH_BW}.
For each candidate $\ell$, the cost vector becomes
layout-dependent:
\begin{equation}
  \mathbf{c}_{\ell}(x) =
  \bigl(
    c_{\mathrm{MMA}}(p, T),\;
    c_{\tma{}}^{\mathrm{ld}}(T, \ell),\;
    c_{\tma{}}^{\mathrm{st}}(T, \ell),\;
    \dots
  \bigr)
  \label{eq:layout-cost}
\end{equation}
where $c_{\tma{}}^{\mathrm{ld}}(T, \ell)$ depends on
the alignment properties of layout~$\ell$,
which determine whether the \tma{} descriptor achieves
coalesced transfer or incurs penalty
cycles~\cite{NV_BLOG_TILE, SEED_2}.
\end{definition}

\begin{definition}[Joint Formulation
  \inference{}]
\label{def:joint-formulation}
The joint layout-schedule problem is:
\begin{equation}
  \min_{\ell \in [K]}\;
  \min_{\II}\;
  \II
  \quad\text{s.t.}\quad
  \Phi(\mathbf{c}_\ell)
  \label{eq:joint-obj}
\end{equation}
\end{definition}

\begin{remark}
Two solution strategies are available.
\emph{Enumeration}: solve $K$ independent SMT problems
(one per candidate layout) and select the minimum
$\II$.
This is tractable when $K$ is small, as it is for
$\Ftwo$ linear layouts at fixed bit-width: the
candidate set is enumerable from the layout library
of~\cite{SEED_1}.
\emph{Integrated}: encode $\ell$ as an integer variable
in the SMT formulation, using conditional cost
assignments.
This may increase solve time but finds the joint
optimum in a single query \inference{}.
Tractability of the integrated approach is an open
question (gap~G4).
\end{remark}

\begin{figure}[t]
\centering
\begin{BVerbatim}[fontsize=\scriptsize]
+===========================================================+
|  ZLP Normalisation Procedure                              |
+===========================================================+
|                                                           |
|  INPUT:  Raw cost vector c = (c_1, ..., c_k)             |
|          Tolerance eps (default 1e-3)                     |
|                                                           |
|  STEP 1: Compute c_min = min(c_j)                        |
|  STEP 2: Compute ratios r_j = c_j / c_min for all j      |
|                                                           |
|  STEP 3: Search for minimal N in {1, 2, 3, ...}          |
|          such that for all j:                             |
|            |round(N * r_j) - N * r_j| < eps               |
|                                                           |
|  STEP 4: Set c_hat_j = round(N * r_j) for all j          |
|                                                           |
|  OUTPUT: Normalised integer cost vector c_hat             |
|                                                           |
|  EXAMPLE (Blackwell FP16):                                |
|    c = (c_MMA=11.2, c_TMA_ld=~80, c_ALU=4, c_cp=~8)     |
|    c_min = 4                                              |
|    r = (2.8, ~20, 1, ~2)                                  |
|    N = 5: c_hat = (14, ~100, 5, ~10)                      |
|                                                           |
|  PROPERTY: If eps=0, ZLP preserves II optimality          |
|            If eps>0, |II_hat - II*| <= 1                  |
|                                                           |
|  NOTE: TMA latencies are UNVERIFIED for specific          |
|        tile sizes; values above are illustrative          |
+===========================================================+
\end{BVerbatim}
\caption{ZLP normalisation procedure
  (after~\cite{OPT_PIPE}).
  Raw Blackwell cycle counts from~\cite{ARCH_BW}
  are converted to minimal positive integers for SMT
  solver efficiency.
  TMA load/store latencies for specific tile sizes
  are not published and require calibration
  (\cref{sec:calib:tma}).}
\label{fig:zlp}
\end{figure}


% ====================================================================
%  SECTION 12 — LAYOUT TOOLING
% ====================================================================
\section{Layout Tooling}
\label{sec:layout-tool}

The layout morphism space $\mathcal{L}$ (Part~2,
Definition~4.2) has three representation layers.
This section specifies the concrete computational
tooling for each layer and the cross-layer
verification pipeline that connects them.

\subsection{$\Ftwo$ Binary Matrix Library}
\label{sec:layout-tool:f2}

\begin{definition}[$\Ftwo$ Library
  Interface~\cite{SEED_1}]
\label{def:f2-library}
The $\Ftwo$ library exposes the following operations on
binary matrices $L \in \Ftwo^{m \times n}$:
\begin{enumerate}[label=(\alph*)]
  \item $\textsc{Compose}(L_1, L_2) = L_2 \cdot L_1$:
    matrix multiplication over $\Ftwo$;
    cost $O(m \cdot n \cdot p)$ for
    $L_1 \in \Ftwo^{m \times n}$,
    $L_2 \in \Ftwo^{n \times p}$~\cite{SEED_1}.
  \item $\textsc{Convert}(L_1, L_2) = L_2 \cdot
    L_1^{-1}$:
    layout-to-layout conversion via pseudoinverse;
    eliminates quadratic code
    explosion~\cite{SEED_1}.
  \item $\textsc{Rank}(L)$:
    rank over $\Ftwo$ via Gaussian elimination;
    cost $O(\min(m,n)^2 \cdot \max(m,n))$.
  \item $\textsc{Enumerate}(m, n, p)$:
    enumerate all valid $\Ftwo^{m(p) \times n}$ layout
    matrices for precision~$p \in \mathcal{P}$
    satisfying alignment constraints
    $\mathrm{alignment}(p)$~\cite{ARCH_BW, SEED_1}.
\end{enumerate}
The dimension $m(p)$ depends on element bit-width:
$m(\texttt{FP16}) = \lceil\log_2(16)\rceil = 4$
address bits per element; for \texttt{FP4},
$m(\texttt{FP4}) = 2$~\cite{SEED_1}.
\end{definition}

\begin{remark}
The \textsc{Enumerate} operation is tractable because
the candidate space grows as
$O(2^{m(p) \cdot n})$, which for typical tile
sizes ($n \leq 16$ address bits) and low-precision
types ($m \leq 4$) yields at most $2^{64}$
candidates before rank and alignment pruning.
In practice, the Triton compiler
of~\cite{SEED_1} uses a curated library rather than
exhaustive enumeration.
Layouts for FP4 and FP6 have not been characterised in
the golden sources \unverified{} (gap~G8).
\end{remark}


\subsection{\ISL{}-Based Relational Layout Analysis}
\label{sec:layout-tool:isl}

\begin{definition}[\ISL{} Relation Construction
  from \CuTe{}~\cite{SEED_2}]
\label{def:isl-construct}
Given a \CuTe{} layout
$H = (\mathrm{shape}, \mathrm{stride})$ with
$\mathrm{shape} = (s_0, \dots, s_{k-1})$ and
$\mathrm{stride} = (d_0, \dots, d_{k-1})$,
Algorithm~1 of~\cite{SEED_2} constructs the \ISL{}
relation:
\begin{equation}
  R_H = \bigl\{c \to
    \textstyle\sum_{i=0}^{k-1}
    \bigl(\lfloor c / \sigma_i \rfloor \bmod s_i\bigr)
    \cdot d_i
  \bigr\}
  \label{eq:isl-cute}
\end{equation}
where $\sigma_i = \prod_{j<i} s_j$ is the cumulative
stride.
The \ISL{} library represents this as a quasi-affine
relation over integer
sets~\cite{SEED_2}.
\end{definition}

\begin{definition}[\ISL{} Swizzle
  Analysis~\cite{SEED_2}]
\label{def:isl-swizzle}
Non-linear swizzle patterns (e.g., XOR-based \SMEM{}
bank-conflict avoidance) are modelled as bit-level
integer set operations within \ISL{}:
\begin{equation}
  S_w(a) = a \oplus (\lfloor a / 2^b \rfloor
  \bmod 2^w) \cdot 2^s
  \label{eq:isl-swizzle}
\end{equation}
where $b, w, s$ parametrise the swizzle
pattern~\cite{SEED_2}.
The \ISL{} representation captures this via
existentially quantified binary decomposition of the
address bits.
Swizzles that are non-linear lie
\emph{outside} $\mathcal{L}^{\Ftwo}$ and require
$\mathcal{L}^{\ISL{}}$~\cite{SEED_1, SEED_2}.
\end{definition}

\begin{definition}[\ISL{} Composition
  Verification~\cite{SEED_2}]
\label{def:isl-compose-verify}
To verify that a chain of layout transformations
$\ell_1, \ell_2, \dots, \ell_k$ is consistent
across layers, compute:
\begin{equation}
  R_{\mathrm{chain}} =
  R_{\ell_k} \circ_{\ISL{}} \cdots
  \circ_{\ISL{}} R_{\ell_1}
  \label{eq:isl-chain}
\end{equation}
using \ISL{} relation composition, and verify that
$R_{\mathrm{chain}}$ is injective on its domain (per
INV-L1, Part~2) by checking:
\[
  |\operatorname{dom}(R_{\mathrm{chain}})|
  = |\operatorname{ran}(R_{\mathrm{chain}})|
\]
using \ISL{} cardinality counting~\cite{SEED_2}.
Layout size $|H|$ and co-size $\|H\|$ (Part~1,
Definition~2.5) provide diagnostic bounds.
\end{definition}


\subsection{Categorical Construction Validation}
\label{sec:layout-tool:cat}

\begin{definition}[Tractability
  Check~\cite{SEED_4}]
\label{def:cat-check}
Given a layout relation $R$, determine whether it
arises from the $\mathbf{Tuple}$/$\mathbf{Nest}$
categorical construction by:
\begin{enumerate}[label=(\roman*)]
  \item Decompose $R$ into shape-stride pairs via the
    inverse of Algorithm~1~\cite{SEED_2}.
  \item Verify that the shape-stride structure satisfies
    the tractability criterion
    of~\cite{SEED_4}: the layout is tractable iff
    it can be expressed as a composition of categorical
    morphisms in $\mathbf{Nest}$.
  \item Cross-validate against the CUTLASS library
    of known tractable layouts~\cite{SEED_4}.
\end{enumerate}
\end{definition}

\begin{definition}[CUTLASS Alignment
  Test~\cite{SEED_4}]
\label{def:cutlass-align}
For each candidate layout, the validation pipeline
compares the computed $\mathbf{Nest}$ morphism output
against CUTLASS ground-truth layout descriptors.
The Python implementation of~\cite{SEED_4}
provides automated comparison: for layout~$L$,
emit the \CuTe{} layout descriptor and verify
$L_{\mathrm{computed}} = L_{\mathrm{CUTLASS}}$
element-wise over the tile domain.
Mismatches indicate either a non-tractable layout or
an error in the embedding functor
$F$~\cite{SEED_4}.
\end{definition}


\subsection{Three-Layer Verification Pipeline}
\label{sec:layout-tool:pipeline}

\begin{figure}[t]
\centering
\begin{BVerbatim}[fontsize=\scriptsize]
+===========================================================+
|  Layout Verification Pipeline                             |
+===========================================================+
|                                                           |
|  INPUT: Candidate layout L for tile T, precision p        |
|                                                           |
|  LAYER 1: F_2 Fragment Check                              |
|  +-----------------------------------------------------+  |
|  | Is L linear?                                        |  |
|  |   YES: L_f2 = F2 matrix [SEED_1]                    |  |
|  |         Check rank(L_f2) = min(m,n) -> injective    |  |
|  |         Check alignment(p) constraints [ARCH_BW]    |  |
|  |         => PASS/FAIL for linear fragment             |  |
|  |   NO:  proceed to ISL layer                          |  |
|  +--+-----------+--------------------------------------+  |
|     |           |                                         |
|     | linear    | non-linear                              |
|     v           v                                         |
|  LAYER 2: ISL Analysis                                    |
|  +-----------------------------------------------------+  |
|  | Construct R_L via Algorithm 1 [SEED_2]               |  |
|  | IF linear: verify phi(L_f2) = R_L  [PO-1, gap G1]   |  |
|  | Compute |dom(R_L)|, |ran(R_L)| for injectivity      |  |
|  | Model swizzle as ISL bit-ops [SEED_2]                |  |
|  | Verify composition: R_chain injective                |  |
|  | => PASS/FAIL for ISL representation                  |  |
|  +--+--------------------------------------------------+  |
|     |                                                     |
|     v                                                     |
|  LAYER 3: Categorical Validation                          |
|  +-----------------------------------------------------+  |
|  | Decompose R_L into Nest morphism [SEED_4]            |  |
|  | Check tractability criterion [SEED_4]                |  |
|  | Verify F(nest_morph) = R_L  [PO-2, gap G2]          |  |
|  | Cross-validate against CUTLASS [SEED_4]              |  |
|  | => PASS/FAIL for categorical foundation              |  |
|  +-----------------------------------------------------+  |
|                                                           |
|  OUTPUT: Verified layout with representation at each      |
|          applicable layer; flagged proof obligation        |
|          status (PO-1 / PO-2)                             |
+===========================================================+
\end{BVerbatim}
\caption{Three-layer layout verification pipeline.
  Each candidate layout is checked at the
  $\Ftwo$~\cite{SEED_1}, \ISL{}~\cite{SEED_2},
  and categorical~\cite{SEED_4} levels.
  Cross-layer consistency (embedding $\phi$ and
  functor $F$) is a proof obligation for \BUTA{}
  (gaps~G1, G2).}
\label{fig:layout-pipeline}
\end{figure}

\begin{proposition}[Pipeline
  Completeness \inference{}]
\label{prop:pipeline-complete}
Every layout representable in $\mathcal{L}^{\Ftwo}$ is
also representable in $\mathcal{L}^{\ISL{}}$ (by
definition of the embedding $\phi$).
Every tractable layout (per~\cite{SEED_4}) is
representable in $\mathcal{L}^{\ISL{}}$ (by definition
of the functor $F$).
The pipeline therefore provides layered verification:
passing at a lower layer implies representability at
higher layers, subject to PO-1 and PO-2.
\end{proposition}

\begin{remark}
Layouts that are representable in
$\mathcal{L}^{\ISL{}}$ but not in
$\mathcal{L}^{\Ftwo}$ (i.e., non-linear swizzles)
must bypass Layer~1 and be verified at Layers~2--3.
Layouts that are representable in
$\mathcal{L}^{\ISL{}}$ but not tractable in the
categorical sense bypass Layer~3.
The pipeline records which layers each layout passes,
enabling the solver to prefer layouts with the
strongest formal guarantees.
\end{remark}


% ====================================================================
%  SECTION 13 — DATA MOVEMENT AND TOKEN ORDERING
% ====================================================================
\section{Data Movement and Token Ordering Pipeline}
\label{sec:dm-pipeline}

This section operationalises the data movement relation
$\mathcal{D}$ (Part~2, Definition~4.5) and the token
ordering framework (Part~2,
Definition~6.5) into constructive procedures.

\subsection{\tma{} Descriptor Generation from Layout
            Relations}
\label{sec:dm-pipeline:tma-gen}

\begin{definition}[Layout-to-Descriptor
  Mapping~\cite{NV_BLOG_TILE, SEED_2}]
\label{def:layout-to-desc}
Given a layout morphism
$\ell \in \mathrm{Mor}_{\mathcal{L}}$ governing a tile
transfer between tiers $\mu_s$ and $\mu_d$ in
$\mathcal{H}$, the \tma{} descriptor generation
proceeds as:
\begin{enumerate}[label=(\roman*)]
  \item \emph{Contiguity analysis}: extract the
    \ISL{} relation $R_\ell$~\cite{SEED_2} and compute
    whether the range is contiguous in the destination
    tier's address space.
  \item \emph{Descriptor type selection}: if the
    transfer is a structured bulk copy
    (HBM $\to$ \SMEM{}),
    use $\mathrm{desc} = \texttt{idesc}$
    (im2col descriptor for multi-dimensional tiles) or
    $\mathrm{desc} = \texttt{sdesc}$
    (simple descriptor for contiguous
    regions)~\cite{NV_BLOG_TILE}.
  \item \emph{Address computation}: the descriptor's
    base address and stride parameters are derived
    from $R_\ell$'s affine coefficients;
    for $\Ftwo$ linear layouts, these are extracted
    from the matrix
    columns~\cite{SEED_1, SEED_2}.
  \item \emph{Validation}: verify that the generated
    descriptor targets an edge $({\mu_s}, {\mu_d})
    \in E(\mathcal{H})$ and that the transfer size
    respects tier capacity
    (INV-D3)~\cite{ARCH_BW}.
\end{enumerate}
\end{definition}

\begin{remark}
Tensor-of-pointer access patterns exhibit poor
performance on Blackwell and are \emph{not}
generated by the descriptor mapping;
\tma{} descriptors are strictly
preferred~\cite{NV_BLOG_TILE}.
\end{remark}

\begin{figure}[t]
\centering
\begin{BVerbatim}[fontsize=\scriptsize]
+===========================================================+
|  TMA Descriptor Generation Flow                           |
+===========================================================+
|                                                           |
|  INPUT:  Layout morphism l in Mor(L)                      |
|          Source tier mu_s, dest tier mu_d                  |
|          Tile shape T, precision p                         |
|                                                           |
|  STEP 1: Contiguity Analysis                              |
|  +-----------------------------------------------------+  |
|  | Compute R_l via ISL [SEED_2]                         |  |
|  | Check: is ran(R_l) a contiguous interval in mu_d?    |  |
|  |   YES -> simple descriptor (sdesc)                   |  |
|  |   NO  -> check structured pattern                    |  |
|  |          YES -> im2col descriptor (idesc)            |  |
|  |          NO  -> REJECT (use explicit loads)          |  |
|  +--+--------------------------------------------------+  |
|     |                                                     |
|  STEP 2: Descriptor Parameter Extraction                  |
|  +-----------------------------------------------------+  |
|  | base_addr = min(ran(R_l))                            |  |
|  | For sdesc:  size = |ran(R_l)| * bits(p) / 8          |  |
|  | For idesc:  dims, strides from R_l affine coeffs     |  |
|  |            via ISL coeff extraction [SEED_2]         |  |
|  +--+--------------------------------------------------+  |
|     |                                                     |
|  STEP 3: Token Assignment                                 |
|  +-----------------------------------------------------+  |
|  | Assign fresh token tau_out to this TMA operation     |  |
|  | Record (mu_s, mu_d, desc, tau_in, tau_out) in D      |  |
|  | Verify: sum(live_allocs(mu_d)) <= cap(mu_d,x)       |  |
|  |         [INV-D3, ARCH_BW]                            |  |
|  +-----------------------------------------------------+  |
|                                                           |
|  OUTPUT: TMA descriptor tuple delta_TMA                   |
|          Token tau_out for consumer registration          |
+===========================================================+
\end{BVerbatim}
\caption{\tma{} descriptor generation from layout
  relations.
  Contiguity analysis uses \ISL{}
  representations~\cite{SEED_2};
  descriptor types follow the \TileIR{}
  specification~\cite{NV_BLOG_TILE};
  capacity validation uses Blackwell
  parameters~\cite{ARCH_BW}.}
\label{fig:tma-gen}
\end{figure}


\subsection{Token DAG Construction}
\label{sec:dm-pipeline:token-dag}

\begin{definition}[Token Insertion
  Rules~\cite{NV_BLOG_TILE}]
\label{def:token-rules}
Given a tile program's operation sequence, the token
dependency DAG
$\mathcal{T} = (\mathrm{Ops}, \mathrm{TokenEdges})$
is constructed by the following rules:
\begin{enumerate}[label=(\alph*)]
  \item \emph{Producer-consumer}: for every \tma{}
    load $u$ producing token $\tau$ and every MMA or
    ALU operation $v$ consuming the loaded data,
    insert edge $(u, v)$ with token
    $\tau$~\cite{NV_BLOG_TILE}.
  \item \emph{Write-after-read}: for every \tma{}
    store $u$ to a region and subsequent \tma{} load $v$
    of the same region, insert edge
    $(u, v)$~\cite{NV_BLOG_TILE}.
  \item \emph{\TMEM{} lifecycle}: for every
    $\tcgen{}\texttt{.mma}$ operation $u$ writing to
    \TMEM{} and subsequent
    $\tcgen{}\texttt{.cp}$ operation $v$ reading the
    same \TMEM{} region, insert edge
    $(u, v)$~\cite{ARCH_BW, NV_BLOG_TILE}.
  \item \emph{Deallocation ordering}: for every
    $\tcgen{}\texttt{.cp}$ operation $u$ consuming
    data from a \TMEM{} region and subsequent
    $\texttt{dealloc}$ operation $v$, insert edge
    $(u, v)$~\cite{ARCH_BW}.
\end{enumerate}
Program-ordered operations (non-token-ordered) are
sequenced by textual position and do not introduce
token edges~\cite{NV_BLOG_TILE}.
\end{definition}

\begin{figure}[t]
\centering
\begin{BVerbatim}[fontsize=\scriptsize]
+===========================================================+
|  Token DAG for Flash Attention Inner Loop (one iter)      |
+===========================================================+
|                                                           |
|  TMA_load(K_j)                TMA_load(V_j)              |
|  tau_K ----+                  tau_V ----+                 |
|            |                            |                 |
|            v                            |                 |
|  MMA_S: S_ij = Q_i * K_j^T             |                 |
|  tau_S ----+                            |                 |
|            |                            |                 |
|            v                            |                 |
|  ALU: P_ij = softmax(S_ij)             |                 |
|  tau_P ----+                            |                 |
|            |                            |                 |
|            +---+     +------------------+                 |
|                |     |                                    |
|                v     v                                    |
|  MMA_O: O_i += P_ij * V_j   (accumulator in TMEM)        |
|  tau_O ----+                                              |
|            |                                              |
|            v                                              |
|  tcgen05.cp: TMEM -> RF   (when epilogue needed)         |
|  tau_cp ---+                                              |
|            |                                              |
|            v                                              |
|  TMA_store(O_i): RF -> SMEM -> HBM                       |
|                                                           |
|  Token edges: tau_K -> MMA_S, tau_V -> MMA_O,            |
|               tau_S -> ALU, tau_P -> MMA_O,               |
|               tau_O -> tcgen05.cp, tau_cp -> TMA_store    |
|                                                           |
|  Properties:                                              |
|    - DAG is acyclic (INV-D2)  [NV_BLOG_TILE]             |
|    - All conflicting pairs ordered (DRF)                  |
|    - TMEM lifecycle: alloc -> mma -> cp -> dealloc        |
+===========================================================+
\end{BVerbatim}
\caption{Token DAG for one iteration of Flash
  Attention inner loop on Blackwell.
  Token edges enforce ordering between \tma{}
  loads~\cite{NV_BLOG_TILE}, MMA
  operations~\cite{ARCH_BW}, and \TMEM{}
  copies~\cite{ARCH_BW}.
  The DAG is acyclic (INV-D2) and data-race free
  (Part~2, Definition~6.6).}
\label{fig:token-dag-fa}
\end{figure}


\subsection{Token DAG Verification}
\label{sec:dm-pipeline:verify}

\begin{definition}[Acyclicity
  Check~\cite{NV_BLOG_TILE}]
\label{def:acyclicity}
The token DAG $\mathcal{T}$ is verified acyclic
(INV-D2, Part~2) by topological sort.
If the sort succeeds (all operations receive a
topological ordering), the DAG is acyclic.
Failure indicates a circular token dependency,
which must be resolved before schedule
generation.
Complexity: $O(|V| + |\mathrm{TokenEdges}|)$.
\end{definition}

\begin{definition}[Data-Race Freedom
  Verification~\cite{NV_BLOG_TILE}]
\label{def:drf-verify}
Given a well-formed token DAG $\mathcal{T}$, verify
data-race freedom (Part~2, Definition~6.6) by:
\begin{enumerate}[label=(\roman*)]
  \item Identify all \emph{conflict pairs}:
    operation pairs $(a, b)$ accessing the same memory
    location with at least one write.
  \item For each conflict pair, verify
    $a \preceq_{\mathcal{T}} b$ or
    $b \preceq_{\mathcal{T}} a$
    using reachability queries on
    $\mathcal{T}$~\cite{NV_BLOG_TILE}.
  \item Any unordered conflict pair constitutes a
    data race; report and require additional token
    edges.
\end{enumerate}
Reachability is preprocessed via transitive closure
in $O(|V|^2)$ time, after which each pair query is
$O(1)$.
\end{definition}

\begin{definition}[Token-to-SMT
  Integration~\cite{OPT_PIPE, NV_BLOG_TILE}
  (Gap~G7)]
\label{def:token-smt}
Token edges are integrated into the SMT formulation
by augmenting the dependence graph:
for each $(u, v) \in \mathrm{TokenEdges}$,
add dependency:
\begin{equation}
  M(v) - M(u) \;\geq\; \mathrm{lat}(u)
  \label{eq:token-dep}
\end{equation}
where $\mathrm{lat}(u)$ is the operation latency of~$u$
drawn from $\mathbf{c}(x)$~\cite{ARCH_BW}.
The augmented graph $G' = (V, E \cup
\mathrm{TokenEdges})$ replaces $G$ in the SMT clause
set $\Phi_{\mathrm{dep}}$~\cite{OPT_PIPE}.
\end{definition}

\begin{proposition}[DRF Decidability]
\label{prop:drf-decide}
Data-race freedom for a tile block with a finite
well-formed token DAG $\mathcal{T}$ is decidable in
$O(|V|^2 + |V| \cdot |\mathrm{TokenEdges}|)$ time.
The procedure is the composition of transitive-closure
computation and conflict-pair enumeration
from~\cref{def:drf-verify}.
\end{proposition}

\begin{remark}
Token ordering may be too fine-grained for solver
scalability on large programs.
Mitigation: merge non-interfering token groups into
\emph{coarsened tokens}, reducing
$|\mathrm{TokenEdges}|$ while preserving DRF
\inference{} (gap~G7).
The \TileIR{} memory model may evolve in future CUDA
releases; the integration should be parameterised by
CUDA version~\cite{NV_BLOG_TILE}.
\end{remark}


% ====================================================================
%  SECTION 14 — CACHE TRAFFIC ANALYSIS PIPELINE
% ====================================================================
\section{Cache Traffic Analysis Pipeline}
\label{sec:cache-pipe}

The cache traffic function $\mathcal{C}$ (Part~2,
Definition~4.6) requires three computational stages:
reuse-distance computation, parameterised miss
prediction, and tile-ordering optimisation.

\subsection{Reuse-Distance Computation}
\label{sec:cache-pipe:reuse}

\begin{definition}[Stack-Based Reuse Distance
  Algorithm~\cite{SEED_3}]
\label{def:reuse-algo}
Given a tile program's memory access trace
$\alpha = (a_1, a_2, \dots, a_n)$ at sector
granularity ($s = 32$\,bytes), the reuse distance
$\reuse(a_i)$ is computed by maintaining an LRU stack
$\mathcal{S}$:
\begin{enumerate}[label=(\roman*)]
  \item For each access $a_i$:
    if $a_i \in \mathcal{S}$, let
    $\reuse(a_i)$ equal the depth of $a_i$ in
    $\mathcal{S}$; move $a_i$ to top.
  \item If $a_i \notin \mathcal{S}$,
    $\reuse(a_i) = \infty$ (cold miss); push $a_i$ to
    top; evict bottom if
    $|\mathcal{S}| > C_{\mathrm{L2}}/s$.
\end{enumerate}
A sector access is an L2 hit iff
$\reuse(a_i) < C_{\mathrm{L2}}/s$~\cite{SEED_3}.
The per-sector total miss count sums the cold and
capacity misses.
\end{definition}

\begin{remark}
The na\"ive stack simulation has complexity
$O(n \cdot C_{\mathrm{L2}}/s)$ per access.
Tree-based implementations (e.g., balanced BST on
stack positions) reduce this to
$O(n \log (C_{\mathrm{L2}}/s))$.
For Blackwell, $C_{\mathrm{L2}} \approx 65$\,MB,
giving $C_{\mathrm{L2}}/s \approx 2 \times 10^6$
sectors~\cite{ARCH_BW}.
For Flash Attention workloads, the
trace length scales with sequence length
$N$~\cite{SEED_3}.
\end{remark}


\subsection{Parameterised Miss Prediction}
\label{sec:cache-pipe:miss}

\begin{definition}[Model Instantiation for
  Blackwell~\cite{SEED_3, ARCH_BW}]
\label{def:miss-model-inst}
The cache traffic function is instantiated per
workload by:
\begin{enumerate}[label=(\roman*)]
  \item \emph{Trace generation}: symbolically expand the
    tile loop nest under ordering $\sigma$, producing a
    sector-level trace template parameterised by tile
    shape $T$, tile count, and access
    pattern~\cite{SEED_3}.
  \item \emph{Platform parameters}: instantiate
    $C_{\mathrm{L2}} \approx 65$\,MB,
    sector size $s = 32$\,B~\cite{ARCH_BW};
    $N_{\mathrm{SM}} = 148$ for B200 (or runtime SM
    count for GB10)~\cite{ARCH_BW, SEED_3}.
  \item \emph{L2 contention scaling}: under streaming
    access, effective per-SM L2 capacity scales as
    $C_{\mathrm{eff}} \approx C_{\mathrm{L2}} /
    N_{\mathrm{SM}}$;
    L2 hit rate correlates with
    $N_{\mathrm{SM}}$~\cite{SEED_3}.
  \item \emph{L1 assessment}: for streaming tile
    patterns, L1 is negligible and excluded from the
    model~\cite{SEED_3}.
    For non-streaming patterns (e.g., tiled GEMM with
    reuse), L1 contribution requires separate
    estimation \inference{} (gap~G5).
\end{enumerate}
\end{definition}

\begin{figure}[t]
\centering
\begin{BVerbatim}[fontsize=\scriptsize]
+===========================================================+
|  Cache Traffic Analysis Pipeline                          |
+===========================================================+
|                                                           |
|  INPUT:  Tile program P                                   |
|          Tile ordering sigma                              |
|          Platform params: C_L2, N_SM, s [ARCH_BW]         |
|          Access pattern type [NV_workloads]                |
|                                                           |
|  STAGE 1: Trace Generation                                |
|  +-----------------------------------------------------+  |
|  | Symbolically expand tile loop under sigma            |  |
|  | Produce sector-level access trace alpha              |  |
|  | Pattern classification:                              |  |
|  |   streaming (FA K,V) -> L1 negligible [SEED_3]       |  |
|  |   reuse (GEMM A,B)   -> L1 may matter [INFERENCE]   |  |
|  |   mixed               -> per-operator [NV_workloads] |  |
|  +--+--------------------------------------------------+  |
|     |                                                     |
|  STAGE 2: Reuse Distance Computation                      |
|  +-----------------------------------------------------+  |
|  | LRU stack simulation at sector granularity           |  |
|  | d_reuse(a, sigma) for each sector a                  |  |
|  | Complexity: O(n * log(C_L2/s))                       |  |
|  +--+--------------------------------------------------+  |
|     |                                                     |
|  STAGE 3: Miss Prediction                                 |
|  +-----------------------------------------------------+  |
|  | C(sigma, N_SM, C_L2) =                               |  |
|  |   sum_{a} 1[d_reuse(a,sigma) > C_L2/s]  [SEED_3]    |  |
|  |                                                     |  |
|  | Validated baseline: sawtooth on GB10 FA              |  |
|  |   67% miss reduction [SEED_3]                        |  |
|  | Generalization: GEMM, conv [INFERENCE, gap G5]       |  |
|  +--+--------------------------------------------------+  |
|     |                                                     |
|  STAGE 4: Ordering Optimization (feed to solver)          |
|  +-----------------------------------------------------+  |
|  | Encode C(sigma) as OBJ-2 term in solver [Part 2]     |  |
|  | Sawtooth sigma_saw as known feasible solution        |  |
|  | Search: permutation space or parameterized family    |  |
|  +-----------------------------------------------------+  |
|                                                           |
|  OUTPUT: Predicted miss count C(sigma, N_SM, C_L2)        |
|          Optimal or near-optimal ordering sigma*           |
+===========================================================+
\end{BVerbatim}
\caption{Cache traffic analysis pipeline.
  Reuse-distance computation
  follows~\cite{SEED_3}; platform parameters
  from~\cite{ARCH_BW}; access pattern
  classification uses~\cite{NV_workloads}
  taxonomy.
  Stages~3--4 operationalise OBJ-2
  (Part~2).}
\label{fig:cache-pipeline}
\end{figure}


\subsection{Tile Ordering as Optimisation Variable}
\label{sec:cache-pipe:ordering}

\begin{definition}[Ordering Constraint
  Encoding~\cite{SEED_3}]
\label{def:ordering-constraint}
The tile iteration ordering $\sigma$ is encoded as an
optimisation variable by restricting the search to a
parametric family of orderings.
Three candidate families are defined:
\begin{enumerate}[label=(\roman*)]
  \item \emph{Raster}: $\sigma_{\mathrm{rast}}$ scans
    tiles in row-major order (baseline).
  \item \emph{Sawtooth}: $\sigma_{\mathrm{saw}}$
    alternates inner-loop direction per outer-loop
    iteration~\cite{SEED_3}.
  \item \emph{Space-filling}: $\sigma_{\mathrm{hilb}}$
    follows a Hilbert or Z-curve over the 2D tile
    grid \inference{}.
\end{enumerate}
For each family member $\sigma_k$, the pipeline
evaluates
$\mathcal{C}(\sigma_k, N_{\mathrm{SM}},
C_{\mathrm{L2}})$ and selects
$\sigma^{*} = \arg\min_k \mathcal{C}(\sigma_k)$.
\end{definition}

\begin{proposition}[Sawtooth Optimality
  Bound~\cite{SEED_3}]
\label{prop:sawtooth-opt}
For streaming split-$Q$ attention dataflows with
persistent CTAs, sawtooth ordering achieves
$\reuse(a, \sigma_{\mathrm{saw}}) < D_{\mathrm{data}}$
for fraction $\geq (1 - 1/N_{\mathrm{CTA}})$ of
sector accesses~\cite{SEED_3}.
No other ordering in the parametric family
$\{\sigma_{\mathrm{rast}},
\sigma_{\mathrm{saw}},
\sigma_{\mathrm{hilb}}\}$
achieves a uniformly better reuse-distance bound
for streaming patterns \inference{}.
\end{proposition}

\begin{remark}
The sawtooth bound is validated only for Flash
Attention on GB10~\cite{SEED_3}; its extension to
GEMM and convolution is \inference{} (gap~G5).
For non-streaming patterns where tile reuse is
significant, space-filling orderings may outperform
sawtooth \inference{}.
The relationship between tile ordering and \SMEM{}
swizzle patterns~\cite{SEED_2} is not yet formalised.
\end{remark}


% ====================================================================
%  SECTION 15 — CALIBRATION MICROBENCHMARK PLAN
% ====================================================================
\section{Calibration Microbenchmark Plan}
\label{sec:calibration}

The \BUTA{} cost vector $\mathbf{c}(x)$ and tier-graph
labels $(\mathrm{lat}, \mathrm{bw}, \mathrm{cap})$
require empirical calibration on Blackwell hardware.
This section defines the measurement protocols,
hardware counter methodology, and falsifiable
predictions that anchor \BUTA{} to physical
measurements.

\subsection{Experimental Infrastructure}
\label{sec:calib:infra}

\begin{definition}[Measurement Protocol]
\label{def:meas-protocol}
All microbenchmarks execute on:
\begin{enumerate}[label=(\roman*)]
  \item NVIDIA B200 (cc\,10.0, 148~SMs, 228\,KB
    \SMEM{}/SM, \TMEM{} 256\,KB/SM)~\cite{ARCH_BW}.
  \item NVIDIA Grace-Blackwell GB10 (cc\,12.0,
    128\,KB \SMEM{}/SM)~\cite{ARCH_BW, SEED_3}.
  \item Toolchain: CUDA $\geq$\,13.1, PTX $>$\,9.0,
    Nsight Compute for hardware
    counters~\cite{NV_BLOG_TILE}.
\end{enumerate}
Each measurement is repeated ${\geq}\,100$ times with
warm-up; results reported as median $\pm$ IQR.
Clock frequencies are locked to avoid DVFS artefacts.
\end{definition}


\subsection{\tcgen{} Latency Suite}
\label{sec:calib:mma}

\begin{definition}[\tcgen{}\texttt{.mma} Latency
  Measurement~\cite{ARCH_BW}]
\label{def:mma-calib}
For each precision $p \in \mathcal{P}$ and tile size
$T \in \{64{\times}64,\; 128{\times}128,\;
256{\times}128\}$:
\begin{enumerate}[label=(\roman*)]
  \item Construct a serialised dependency chain of
    $N_{\mathrm{ops}}$ consecutive
    \tcgen{}\texttt{.mma} operations,
    each consuming the accumulator of the
    previous~\cite{ARCH_BW}.
  \item Measure total elapsed cycles via
    $\texttt{clock64()}$ around the chain.
  \item Compute per-operation latency:
    $c_{\mathrm{MMA}}(p, T) =
    \mathrm{elapsed} / N_{\mathrm{ops}}$.
\end{enumerate}
\end{definition}

\begin{definition}[Falsifiable Predictions for
  MMA Latency]
\label{def:fp-mma}
The following predictions are derived from the golden
sources and must be confirmed or falsified by
calibration:
\begin{enumerate}[label=(FP-\arabic*)]
  \item $c_{\mathrm{MMA}}(\texttt{FP16}, T) \in
    [11.0, 11.4]$ cycles for all $T$, matching
    the approximately constant latency reported
    in~\cite{ARCH_BW}.
  \item FP32 accumulation halves throughput relative
    to FP16 accumulation:
    $c_{\mathrm{MMA}}(p, T)|_{\mathrm{FP32\,acc}}
    \approx 2 \cdot
    c_{\mathrm{MMA}}(p, T)|_{\mathrm{FP16\,acc}}$%
    ~\cite{ARCH_BW}.
  \item \tcgen{}\texttt{.mma} latency is constant
    across precisions
    $\{\texttt{FP4},\texttt{FP6},\texttt{FP8},
    \texttt{FP64}\}$ \unverified{} (gap~G9).
\end{enumerate}
\end{definition}


\subsection{\TMEM{} Bandwidth Characterisation}
\label{sec:calib:tmem}

\begin{definition}[\TMEM{} Bandwidth
  Measurement~\cite{ARCH_BW}]
\label{def:tmem-calib}
Measure \TMEM{} read and write throughput by:
\begin{enumerate}[label=(\roman*)]
  \item \emph{Read bandwidth}: issue a stream of
    $\tcgen{}\texttt{.ld}$ operations from \TMEM{}
    to RF, varying occupancy (1--64 warps);
    measure bytes transferred per cycle per SM.
  \item \emph{Write bandwidth}: issue a stream of
    $\tcgen{}\texttt{.st}$ operations from RF to
    \TMEM{}, same occupancy sweep.
  \item \emph{Accumulator pathway}: measure effective
    $\texttt{mma\_acc}$ write bandwidth by timing
    \tcgen{}\texttt{.mma} with varying accumulator
    sizes in \TMEM{}~\cite{ARCH_BW}.
  \item \emph{Copy bandwidth}: measure
    $\tcgen{}\texttt{.cp}$ latency and throughput for
    varying region sizes.
\end{enumerate}
\end{definition}

\begin{definition}[Falsifiable Predictions for
  TMEM Bandwidth]
\label{def:fp-tmem}
\begin{enumerate}[label=(FP-\arabic*),start=4]
  \item \TMEM{} read bandwidth
    $\approx 16$\,TB/s per SM at saturation
    \unverified{}~\cite{ARCH_BW}.
  \item \TMEM{} write bandwidth
    $\approx 8$\,TB/s per SM at saturation
    \unverified{}~\cite{ARCH_BW}.
  \item \TMEM{} hit rates of 61--82\% in multi-stage
    tensor pipelines~\cite{ARCH_BW}.
\end{enumerate}
These values are from secondary analysis and require
independent confirmation.
\end{definition}


\subsection{\tma{} Overhead Profiling}
\label{sec:calib:tma}

\begin{definition}[\tma{} vs.\ Pointer-Based
  Comparison~\cite{NV_BLOG_TILE, ARCH_BW}]
\label{def:tma-calib}
Construct paired microbenchmarks for each tile size
$T$:
\begin{enumerate}[label=(\roman*)]
  \item \emph{\tma{} bulk load}: issue a \tma{}
    descriptor-based load of tile~$T$ from HBM to
    \SMEM{}; measure latency and throughput via Nsight
    counters~\cite{NV_BLOG_TILE}.
  \item \emph{Pointer-based gather}: issue element-wise
    loads via tensor-of-pointer pattern for the same
    tile~$T$; measure latency and
    throughput~\cite{NV_BLOG_TILE}.
  \item Compute speedup ratio:
    $\rho(T) = \mathrm{lat}_{\mathrm{ptr}}(T) /
    \mathrm{lat}_{\tma{}}(T)$.
\end{enumerate}
\end{definition}

\begin{definition}[Falsifiable Predictions for TMA]
\label{def:fp-tma}
\begin{enumerate}[label=(FP-\arabic*),start=7]
  \item \tma{} strictly faster than pointer-based loads
    for tiles $\geq 128$ elements:
    $\rho(T) > 1$ for
    $|T| \geq 128$~\cite{NV_BLOG_TILE}.
  \item Tensor-of-pointer patterns exhibit
    ``poor performance''
    per~\cite{NV_BLOG_TILE}; quantify as
    $\rho(T) \geq 2$ for typical tile sizes
    \unverified{}.
\end{enumerate}
\end{definition}


\subsection{L2 Cache Characterisation}
\label{sec:calib:l2}

\begin{definition}[L2 Capacity and Associativity
  Probing~\cite{ARCH_BW, SEED_3}]
\label{def:l2-calib}
Determine effective L2 cache parameters by:
\begin{enumerate}[label=(\roman*)]
  \item \emph{Capacity sweep}: allocate working sets of
    size $W \in \{1, 2, 4, \dots, 128\}$\,MB;
    stream through with single-SM access; measure miss
    rate via \texttt{l2\_hit\_rate} Nsight
    counter~\cite{ARCH_BW}.
    Identify $C_{\mathrm{L2}}$ as the $W$ at which
    miss rate transitions from ${\sim}0$ to
    ${\sim}1$.
  \item \emph{Multi-SM contention}: repeat with
    $N_{\mathrm{SM}} \in \{1, 16, 74, 148\}$ active
    SMs; measure L2 hit rate as a function of
    $N_{\mathrm{SM}}$~\cite{SEED_3}.
  \item \emph{Sector-level validation}: compare
    $\mathcal{C}(\sigma, N_{\mathrm{SM}},
    C_{\mathrm{L2}})$ predicted by the reuse-distance
    model (\cref{def:reuse-algo}) against hardware
    sector miss counters~\cite{SEED_3}.
\end{enumerate}
\end{definition}

\begin{definition}[Falsifiable Predictions for L2]
\label{def:fp-l2}
\begin{enumerate}[label=(FP-\arabic*),start=9]
  \item $C_{\mathrm{L2}} \approx 65$\,MB effective
    capacity on B200~\cite{ARCH_BW}.
  \item Miss rate exhibits a step function at the
    capacity boundary: for $W < C_{\mathrm{L2}}$,
    miss rate $< 5\%$; for
    $W > 1.5 \cdot C_{\mathrm{L2}}$,
    miss rate $> 80\%$~\cite{ARCH_BW}.
  \item Sawtooth ordering reduces L2 misses by
    $67\% \pm 5\%$ on GB10 Flash Attention
    (reproducing~\cite{SEED_3}: 370M $\to$ 120M
    sectors).
  \item 58\% cache-miss latency reduction vs.\
    H200~\cite{ARCH_BW}: measured as
    $\tau_{\mathrm{miss}}(\text{B200}) \approx
    420$ cycles vs.\
    $\tau_{\mathrm{miss}}(\text{H200}) \approx
    1000$ cycles.
\end{enumerate}
\end{definition}


\subsection{Cross-Platform Calibration}
\label{sec:calib:cross}

\begin{figure}[t]
\centering
\begin{BVerbatim}[fontsize=\scriptsize]
+===========================================================+
|  Calibration Experiment Matrix                            |
+===========================================================+
|                                                           |
|  Parameter         | B200 cc10.0  | GB10 cc12.0  | Gap   |
|  ------------------+--------------+--------------+-------+
|  tcgen05.mma       |              |              |       |
|    FP16  latency   | 11.0-11.4 cy | 11.0-11.4 cy | G9   |
|    FP4   latency   | UNVERIFIED   | UNVERIFIED   | G9   |
|    FP6   latency   | UNVERIFIED   | UNVERIFIED   | G9   |
|    FP8   latency   | UNVERIFIED   | UNVERIFIED   | G9   |
|    FP64  latency   | UNVERIFIED   | UNVERIFIED   | G9   |
|  ------------------+--------------+--------------+-------+
|  TMEM bandwidth    |              |              |       |
|    read  BW        | ~16 TB/s UNV | ~16 TB/s UNV | G3   |
|    write BW        | ~8 TB/s UNV  | ~8 TB/s UNV  | G3   |
|    hit rate        | 61-82%       | 61-82%       | G3   |
|  ------------------+--------------+--------------+-------+
|  TMA overhead      |              |              |       |
|    bulk lat.       | measure      | measure      | G9   |
|    ptr-vs-TMA      | rho > 1      | rho > 1      | ---  |
|  ------------------+--------------+--------------+-------+
|  L2 cache          |              |              |       |
|    capacity        | ~65 MB       | ~65 MB       | G5   |
|    miss latency    | ~420 cy      | measure      | G5   |
|    multi-SM rate   | f(N_SM)      | f(N_SM)      | G5   |
|  ------------------+--------------+--------------+-------+
|  SMEM capacity     | 228 KB       | 128 KB       | ---  |
|  Max warps         | 64           | 48           | ---  |
|  SMs               | 148          | measure      | ---  |
|  ------------------+--------------+--------------+-------+
|  2CTA sync latency | UNVERIFIED   | UNVERIFIED   | G6   |
|  ------------------+--------------+--------------+-------+
|                                                           |
|  Sources: ARCH_BW (all hardware params)                   |
|           SEED_3  (L2 model on GB10)                      |
|           NV_BLOG_TILE (TMA, Tile IR, CUDA 13.1)          |
|  UNV = UNVERIFIED (requires measurement)                  |
+===========================================================+
\end{BVerbatim}
\caption{Calibration experiment matrix for B200 and
  GB10.
  Parameters anchored to~\cite{ARCH_BW} are listed
  with expected values; \unverified{} entries require
  independent measurement.
  Gap references identify which \BUTA{} gaps each
  measurement addresses.
  \SMEM{} capacity differs between
  platforms~\cite{ARCH_BW}.}
\label{fig:calib-matrix}
\end{figure}

\begin{definition}[Cross-Platform Parameterisation]
\label{def:cross-plat}
The cost vector and tier-graph labels are instantiated
separately for each platform:
\begin{align}
  \mathbf{c}(10.0) &: \text{B200-specific costs}
    \notag\\
  \mathbf{c}(12.0) &: \text{GB10-specific costs}
    \notag
\end{align}
Key differences between platforms~\cite{ARCH_BW}:
\begin{enumerate}[label=(\roman*)]
  \item $C_{\SMEM{}}(10.0) = 228$\,KB vs.\
    $C_{\SMEM{}}(12.0) = 128$\,KB.
  \item $N_{\mathrm{WS}}(10.0) = 64$ warps vs.\
    $N_{\mathrm{WS}}(12.0) = 48$ warps.
  \item B200 has 148~SMs (8~GPCs, dual-die
    NV-HBI)~\cite{ARCH_BW}; GB10 SM count differs.
\end{enumerate}
The solver must be run separately for each platform,
as optimal schedules may differ due to resource
bound differences~\cite{OPT_PIPE, ARCH_BW}.
\end{definition}

\begin{remark}
SEED\_3~\cite{SEED_3} results (67\% L2 miss
reduction, 60\% throughput increase) are validated on
GB10 (cc\,12.0, 128\,KB \SMEM{}).
Reproducing these results on B200 (228\,KB \SMEM{})
requires adjusting the tile size parameters to respect
the different \SMEM{} capacity; the larger \SMEM{} may
permit deeper pipelining and larger tiles, potentially
improving L2 hit rates \inference{}.
\end{remark}


\subsection{Solver Calibration Validation}
\label{sec:calib:solver}

\begin{definition}[Solver Output
  Validation~\cite{OPT_PIPE, ARCH_BW}]
\label{def:solver-valid}
The calibrated solver is validated by:
\begin{enumerate}[label=(\roman*)]
  \item \emph{Schedule rediscovery}: run the SMT solver
    on Flash Attention dependence graphs with
    calibrated Blackwell cost vector; verify that the
    solver rediscovers schedules matching or exceeding
    expert FA3/FA4 implementations~\cite{OPT_PIPE}.
  \item \emph{Performance parity}: compile
    solver-generated schedules to PTX ($> 9.0$) and
    execute on B200/GB10; compare TFLOPS against expert
    baselines~\cite{OPT_PIPE, ARCH_BW}.
  \item \emph{Cross-generation distinction}: verify that
    Blackwell-parameterised optimal schedules differ
    structurally from Hopper-parameterised
    schedules~\cite{OPT_PIPE}, confirming that the
    cost vector captures architectural differences.
\end{enumerate}
\end{definition}

\begin{definition}[Falsifiable Predictions for Solver]
\label{def:fp-solver}
\begin{enumerate}[label=(FP-\arabic*),start=13]
  \item Solve time $< 60$ seconds for Flash
    Attention-scale problems, consistent
    with~\cite{OPT_PIPE} evaluation.
  \item Calibrated solver rediscovers FA3/FA4-equivalent
    schedules on Blackwell~\cite{OPT_PIPE}.
  \item Blackwell cost parameterisation yields distinct
    optimal schedule structure vs.\
    Hopper~\cite{OPT_PIPE}.
  \item With 2CTA extension ($\Phi_{\mathrm{2cta}}$),
    solver achieves lower $\II$ on dense MMA kernels
    than single-CTA
    formulation~\cite{NV_BLOG_TILE} \inference{}.
\end{enumerate}
\end{definition}


% ====================================================================
%  SECTION 16 — WORKLOAD CHARACTERISATION
% ====================================================================
\section{Workload Characterisation}
\label{sec:workload-char}

This section specifies the methods for classifying
workloads within \BUTA{}'s workload descriptor
$\mathcal{W}$ (Part~2, Definition~4.7) and conditioning
the solver cost vector accordingly.

\subsection{Per-Operator Arithmetic Intensity Profiling}
\label{sec:workload-char:ai}

\begin{definition}[Arithmetic Intensity
  Computation~\cite{ARCH_BW, NV_workloads}]
\label{def:ai-compute}
For each tile-level operation $\mathrm{op}_i$ in a
workload, compute:
\begin{equation}
  \mathrm{AI}_i =
  \frac{\mathrm{FLOP}(\mathrm{op}_i, p_i)}
       {\mathrm{Bytes}(\mathrm{op}_i, T_i)}
  \label{eq:ai}
\end{equation}
where $\mathrm{FLOP}(\mathrm{op}_i, p_i)$ is the
floating-point operation count at precision~$p_i$
(accounting for the $177\times$ throughput scaling
from FP64 to FP4~\cite{ARCH_BW}),
and $\mathrm{Bytes}(\mathrm{op}_i, T_i)$ is the
total data volume for tile shape~$T_i$ (input operands
+ output, counted at the memory tier from which they
are sourced).
\end{definition}

\begin{definition}[Reuse Class
  Assignment~\cite{SEED_3, NV_workloads}]
\label{def:reuse-class}
Each operator receives a reuse class
$r_i \in \{\texttt{streaming},\, \texttt{reuse},\,
\texttt{mixed}\}$ based on its access pattern:
\begin{enumerate}[label=(\roman*)]
  \item \texttt{streaming}: each data element is
    accessed once per tile invocation
    (e.g., $K$, $V$ tiles in Flash
    Attention)~\cite{SEED_3}.
  \item \texttt{reuse}: data elements are accessed
    multiple times across tile invocations
    (e.g., $Q$ tiles resident in
    \SMEM{})~\cite{SEED_3}.
  \item \texttt{mixed}: workload contains operators of
    both types
    (e.g., Flash Attention as a
    whole)~\cite{NV_workloads}.
\end{enumerate}
The reuse class determines which cache model tier
is dominant: L2 for streaming, potentially
L1+\SMEM{} for reuse patterns~\cite{SEED_3}.
\end{definition}


\subsection{HARP Taxonomy Integration}
\label{sec:workload-char:harp}

\begin{definition}[Workload Classification
  Procedure~\cite{NV_workloads, ARCH_BW}]
\label{def:harp-classify}
Classify each target workload along both HARP
axes~\cite{NV_workloads}:
\begin{enumerate}[label=(\roman*)]
  \item \emph{Depth of compute}: determine whether the
    computation is leaf-only (all compute at SM level)
    or hierarchical (compute distributed across
    hierarchy).
    For Blackwell, all workloads targeted by \BUTA{}
    are leaf-only (compute at SM
    level)~\cite{NV_workloads}.
  \item \emph{Location of heterogeneity}: classify the
    SM as intra-node heterogeneous (CUDA cores + tensor
    cores as distinct
    sub-accelerators)~\cite{NV_workloads, ARCH_BW}.
    Operators mapping to different sub-accelerators
    receive different cost entries in
    $\mathbf{c}(x)$.
\end{enumerate}
\end{definition}

\begin{figure}[t]
\centering
\begin{BVerbatim}[fontsize=\scriptsize]
+===========================================================+
|  Workload Characterisation Flow                           |
+===========================================================+
|                                                           |
|  INPUT: Tile program P with operators op_1, ..., op_n     |
|                                                           |
|  STEP 1: Per-Operator Profiling                           |
|  +-----------------------------------------------------+  |
|  | For each op_i:                                       |  |
|  |   Compute AI_i = FLOP(op_i, p_i) / Bytes(op_i, T_i) |  |
|  |   Assign reuse class r_i [SEED_3, NV_workloads]     |  |
|  |   Record precision p_i in P [ARCH_BW]               |  |
|  |   Map to sub-accelerator:                            |  |
|  |     MMA -> TC_5thgen (c_MMA ~ 11.2 cy)              |  |
|  |     softmax -> CUDA cores (c_ALU)                    |  |
|  |     TMA -> TMA unit (c_TMA)                          |  |
|  |     TMEM cp -> tcgen05 (c_cp)                        |  |
|  +--+--------------------------------------------------+  |
|     |                                                     |
|  STEP 2: HARP Classification [NV_workloads]               |
|  +-----------------------------------------------------+  |
|  | Depth of compute: leaf-only (SM-level)               |  |
|  | Heterogeneity: intra-node (CUDA + TC) [ARCH_BW]     |  |
|  | Workload type:                                       |  |
|  |   all op_i same AI class -> homogeneous              |  |
|  |   mix of high-AI + low-AI -> mixed-reuse             |  |
|  | Cross-depth: not applicable for BUTA targets         |  |
|  +--+--------------------------------------------------+  |
|     |                                                     |
|  STEP 3: Cost Vector Conditioning                         |
|  +-----------------------------------------------------+  |
|  | Construct W = {(op_i, AI_i, r_i, p_i) : i in [n]}   |  |
|  | Map W -> c(x) via:                                   |  |
|  |   c(op_i) = calibrated latency [Sec 15]             |  |
|  |   Adjust for precision: sigma(p_i) -> SASS class    |  |
|  |   Weight by AI for OBJ-3 [Part 2]                   |  |
|  | Feed c(x) to solver [Sec 11]                         |  |
|  +-----------------------------------------------------+  |
|                                                           |
|  OUTPUT: Workload descriptor W                            |
|          Conditioned cost vector c(x)                     |
+===========================================================+
\end{BVerbatim}
\caption{Workload characterisation flow.
  Each operator is profiled for arithmetic intensity
  and reuse class~\cite{NV_workloads, SEED_3},
  mapped to a Blackwell
  sub-accelerator~\cite{ARCH_BW}, and classified
  under the HARP
  taxonomy~\cite{NV_workloads}.
  The output conditions the solver cost
  vector~\cite{OPT_PIPE}.}
\label{fig:workload-flow}
\end{figure}


\subsection{Cost Vector Conditioning from
            Workload Descriptor}
\label{sec:workload-char:cost}

\begin{definition}[$\mathcal{W} \to \mathcal{S}$
  Interface
  Realisation~\cite{OPT_PIPE, NV_workloads, ARCH_BW}]
\label{def:w-to-s}
The interface constraint
$\Psi_{\mathcal{W} \to \mathcal{S}}$ (Part~2,
Definition~4.9(iii)) is realised by constructing
the cost vector $\mathbf{c}(x)$ as follows:
\begin{enumerate}[label=(\roman*)]
  \item For each MMA operation: cost
    $c_{\mathrm{MMA}}(p_i, T_i)$ from calibration
    (\cref{def:mma-calib}),
    SASS class $\sigma(p_i)$ from the precision
    lattice~\cite{ARCH_BW}.
  \item For each \tma{} operation: cost
    $c_{\tma{}}(T_i)$ from calibration
    (\cref{def:tma-calib})~\cite{NV_BLOG_TILE}.
  \item For each ALU operation: cost
    $c_{\mathrm{ALU}}$ from standard SM latency;
    reduced-precision epilogues may use tensor-core
    reduction paths \unverified{}.
  \item Apply ZLP normalisation
    (\cref{def:zlp})~\cite{OPT_PIPE} to obtain
    integer costs $\hat{\mathbf{c}}(x)$.
\end{enumerate}
For mixed-reuse workloads~\cite{NV_workloads}, the
cost vector contains heterogeneous entries:
high-$\mathrm{AI}$ MMA operations have cost dominated
by compute latency, while low-$\mathrm{AI}$ epilogue
operations have cost dominated by memory access
latency~\cite{ARCH_BW, NV_workloads}.
\end{definition}

\begin{remark}
The precision lattice introduces a multiplicative
factor: throughput scales up to $177\times$ from FP64
to FP4~\cite{ARCH_BW}, implying that the
\emph{effective} cost per FLOP varies by two orders of
magnitude across precision choices.
The per-operator cost in $\mathbf{c}(x)$ must
therefore be normalised to a common unit (e.g.,
cycles per tile) rather than cycles per
FLOP~\cite{OPT_PIPE, ARCH_BW}.
Block scaling formats (MXFP4, MXFP8) add scale-factor
handling overhead not captured in the base cost model
\unverified{} (gap~G8).
\end{remark}


% ====================================================================
%  SUMMARY
% ====================================================================
\section*{Part~3 Summary}

This part has developed the computational machinery
required to operationalise the \BUTA{} model defined in
Part~2.
\Cref{sec:solver} specified solver encodings (ILP,
SMT, 2CTA extension, ZLP normalisation)
that translate the abstract constraint system
$\mathcal{S}$ into executable solver
inputs~\cite{OPT_PIPE, NV_BLOG_TILE, ARCH_BW}.
\Cref{sec:layout-tool} defined a three-layer layout
verification pipeline spanning
$\Ftwo$~\cite{SEED_1},
\ISL{}~\cite{SEED_2}, and
categorical~\cite{SEED_4} representations.
\Cref{sec:dm-pipeline} operationalised \tma{}
descriptor generation and token DAG construction
with decidable DRF
verification~\cite{NV_BLOG_TILE, ARCH_BW}.
\Cref{sec:cache-pipe} instantiated the reuse-distance
cache traffic model for
Blackwell~\cite{SEED_3, ARCH_BW} and formalised
tile ordering as an optimisation variable.
\Cref{sec:calibration} defined the full calibration
microbenchmark plan with 16~falsifiable predictions
covering \tcgen{} latency, \TMEM{} bandwidth,
\tma{} overhead, L2 characterisation, and
solver validation~\cite{ARCH_BW, OPT_PIPE}.
\Cref{sec:workload-char} specified workload
classification methods using the HARP
taxonomy~\cite{NV_workloads} and precision-dependent
cost conditioning~\cite{ARCH_BW}.

Part~4 will define the evaluation strategy, benchmark
suite, reproducibility plan, and threat analysis that
complete the \BUTA{} proposal.


% ====================================================================
%  SOURCE AUDIT
% ====================================================================
\section*{Source Audit (Part~3)}
\label{sec:audit-p3}

\noindent
All eight golden sources are cited in this part.
\Cref{tab:audit-p3} records usage.

\begin{table}[H]
\centering
\caption{Part~3 source audit: usage and anchors.}
\label{tab:audit-p3}
\footnotesize
\begin{tabular}{@{}p{1.3cm}p{6.0cm}@{}}
\toprule
\textbf{ID} & \textbf{Used for (anchor)} \\
\midrule
ARCH\_BW &
  Cost vector parameterisation
  (\tcgen{}.mma 11.0--11.4\,cy, 4 exec ctx,
  64/48 warps, 148 SMs);
  tier-graph labels (\TMEM{} 256\,KB,
  \SMEM{} 228/128\,KB, L2 ${\sim}$65\,MB);
  \TMEM{} bandwidth \unverified{} (${\sim}$16/8\,TB/s);
  calibration: precision lattice, SASS mappings,
  MMA.2SM TPC bandwidth, 58\% miss-latency reduction;
  cross-platform differences cc\,10.0 vs.\ 12.0;
  AI profiling: 177$\times$ throughput scaling. \\
\addlinespace
OPT\_PIPE &
  ILP encoding for modulo scheduling (\cref{def:ilp-constraints});
  SMT encoding for joint SWP+WS
  ($\mathrm{op}[v,i,t]$, $W(v)$, clause structure);
  ZLP normalisation procedure;
  solver validation: Twill rediscovery of FA schedules;
  solve-time prediction ($< 60$\,s);
  cost-vector conditioning via $\mathcal{W}$. \\
\addlinespace
NV\_BLOG\_ TILE &
  2CTA constraint encoding
  ($\texttt{num\_ctas}{=}2$,
  $\Phi_{\mathrm{2cta}}$);
  \tma{} descriptor types (idesc, sdesc);
  token-ordered memory model (DAG construction rules,
  DRF definition);
  \tma{} vs.\ pointer performance;
  CUDA 13.1 / PTX $>$ 9.0 requirement;
  occupancy hint limitation. \\
\addlinespace
NV\_work\-loads &
  Workload characterisation:
  HARP two-axis taxonomy (depth $\times$ heterogeneity);
  mixed-reuse classification;
  reuse class assignment;
  cost-vector conditioning for heterogeneous operators;
  SM sub-accelerator mapping. \\
\addlinespace
SEED\_1 &
  $\Ftwo$ library interface (compose, convert, rank,
  enumerate);
  layout bit-width dependence on precision;
  linear fragment identification;
  candidate layout enumeration for joint solver;
  verification pipeline Layer~1. \\
\addlinespace
SEED\_2 &
  \ISL{} relation construction (Algorithm~1);
  swizzle as bit-level ISL operation;
  contiguity analysis for \tma{} descriptor generation;
  composition verification via cardinality;
  verification pipeline Layer~2;
  layout size/co-size diagnostics. \\
\addlinespace
SEED\_3 &
  Reuse-distance computation algorithm;
  sawtooth ordering bound;
  67\% L2 miss reduction on GB10 (370M $\to$ 120M
  sectors, falsifiable prediction);
  L1 negligible for streaming;
  L2 hit rate ${\sim} f(N_{\mathrm{SM}})$;
  cache pipeline Stages~1--4;
  ordering optimisation. \\
\addlinespace
SEED\_4 &
  Categorical construction validation (tractability
  check, CUTLASS alignment);
  $\mathbf{Tuple}$/$\mathbf{Nest}$ morphism
  decomposition;
  verification pipeline Layer~3;
  functor $F$ target validation. \\
\bottomrule
\end{tabular}
\end{table}


% ====================================================================
%  BIBLIOGRAPHY
% ====================================================================
\begin{thebibliography}{8}

\bibitem{ARCH_BW}
A.~Jarmusch and S.~Chandrasekaran,
``Microbenchmarking NVIDIA's Blackwell Architecture:
An In-Depth Architectural Analysis,''
\emph{arXiv:2512.02189v1}, Dec.~2025.
\url{https://arxiv.org/html/2512.02189v1}

\bibitem{OPT_PIPE}
K.~Soi, A.~Venkat, T.~Nowicki, and
G.~Biros,
``Optimal Software Pipelining and Warp
Specialization for Tensor Core GPUs,''
\emph{arXiv:2512.18134v1}, Dec.~2025.
\url{https://arxiv.org/html/2512.18134v1}

\bibitem{NV_BLOG_TILE}
NVIDIA,
``Advancing GPU Programming with the CUDA Tile IR
Backend for OpenAI Triton,''
\emph{NVIDIA Developer Blog}, Jan.~2026.
\url{https://developer.nvidia.com/blog/advancing-gpu-programming-with-the-cuda-tile-ir-backend-for-openai-triton/}

\bibitem{NV_workloads}
S.~Garg, M.~Pellauer, and T.~Krishna,
``HARP: A Taxonomy for Heterogeneous and
Hierarchical Processors for Mixed-Reuse Workloads,''
\emph{arXiv:2502.13113}, Feb.~2025.
\url{https://arxiv.org/html/2502.13113}

\bibitem{SEED_1}
J.~Zhou, M.~Lezcano, A.~Goucher, et~al.,
``Linear Layouts: Robust Code Generation of Efficient
Tensor Computation Using $\mathbb{F}_2$,''
\emph{arXiv:2505.23819v3}, Oct.~2025.
\url{https://arxiv.org/html/2505.23819v3}

\bibitem{SEED_2}
S.~Bhaskaracharya, N.~Acharya, M.~Hagedorn,
and S.~Grover,
``Modeling Layout Abstractions Using Integer Set
Relations,''
\emph{arXiv:2511.10374v1}, Nov.~2025.
\url{https://arxiv.org/html/2511.10374v1}

\bibitem{SEED_3}
Z.~Zhu, J.~Pan, and Y.~Ding,
``Sawtooth Wavefront Reordering: Enhanced CuTile
FlashAttention on NVIDIA GB10,''
\emph{arXiv:2601.16032v2}, Jan.~2026.
\url{https://arxiv.org/html/2601.16032v2}

\bibitem{SEED_4}
R.~Carlisle, A.~Shah, H.~Stern, and
D.~VanKoughnett,
``Categorical Foundations for CuTe Layouts,''
\emph{arXiv:2601.05972v1}, Jan.~2026.
\url{https://arxiv.org/pdf/2601.05972v1}

\end{thebibliography}

\end{document}
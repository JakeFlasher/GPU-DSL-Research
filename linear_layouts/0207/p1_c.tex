% ============================================================
%  BUTA — Blackwell Unified Tile Algebra
%  Part 1: Problem Framing, Formal Preliminaries,
%          Notation, and Running Example
% ============================================================
\documentclass[10pt,twocolumn]{article}

% ---- Geometry & Typography ----
\usepackage[margin=0.85in,columnsep=0.25in]{geometry}
\usepackage[T1]{fontenc}
\usepackage{lmodern}
\usepackage{microtype}

% ---- Mathematics ----
\usepackage{amsmath,amssymb,amsthm,mathtools}

% ---- Figures & Tables ----
\usepackage{graphicx,booktabs,array,multirow}
\usepackage{fancyvrb}          % for ASCII diagrams
\usepackage{float}

% ---- Cross-references & Links ----
\usepackage[colorlinks,citecolor=blue!70!black,
            linkcolor=blue!60!black,urlcolor=blue!50!black]{hyperref}
\usepackage[capitalise,noabbrev]{cleveref}

% ---- Misc ----
\usepackage{xcolor}
\usepackage{enumitem}
\setlist{nosep,leftmargin=*}

% ---- Theorem-like environments ----
\theoremstyle{definition}
\newtheorem{definition}{Definition}[section]
\newtheorem{invariant}[definition]{Invariant}
\newtheorem{remark}[definition]{Remark}
\theoremstyle{plain}
\newtheorem{proposition}[definition]{Proposition}
\newtheorem{obligation}[definition]{Proof Obligation}

% ---- Convenience macros ----
\newcommand{\BUTA}{\textsc{Buta}}
\newcommand{\Ftwo}{\mathbb{F}_2}
\newcommand{\ZZ}{\mathbb{Z}}
\newcommand{\NN}{\mathbb{N}}
\newcommand{\ISL}{\textsc{ISL}}
\newcommand{\CuTe}{\textsc{CuTe}}
\newcommand{\TileIR}{Tile\,IR}
\newcommand{\TMEM}{\textsc{TMEM}}
\newcommand{\SMEM}{\textsc{SMEM}}
\newcommand{\tcgen}{\texttt{tcgen05}}
\newcommand{\tma}{\textsc{TMA}}
\newcommand{\II}{\mathit{II}}
\newcommand{\reuse}{d_{\mathrm{reuse}}}
\newcommand{\unverified}{\textsc{[unverified]}}
\newcommand{\inference}{\textsc{[inference]}}

% ---- Title ----
\title{%
  \BUTA{}: Blackwell Unified Tile Algebra\\[4pt]
  \large Part~1 --- Problem Framing, Formal Preliminaries,
         and Running Example}

\author{%
  \textit{[Authors redacted for review]}\\
  Target architecture: NVIDIA Blackwell (B200 cc\,10.0;
  Grace-Blackwell GB10 cc\,12.0)\\
  Toolchain: CUDA ${>}\,13.0$ (prefer 13.1+),
  PTX ${>}\,9.0$ (strict)}

\date{February 2026}

\begin{document}
\maketitle

% ====================================================================
%  SECTION 1 — INTRODUCTION
% ====================================================================
\section{Introduction: The Blackwell Optimisation Challenge}
\label{sec:intro}

The NVIDIA Blackwell GPU generation (B200, compute
capability~10.0; Grace-Blackwell GB10, compute
capability~12.0) introduces four simultaneous
architectural shifts that, taken together, invalidate
the scheduling, layout, and data-movement assumptions
that hold for Hopper (cc\,9.0) and prior
generations~\cite{ARCH_BW}.
This section frames each shift and motivates the need
for a single, compositional formal model---the
\emph{Blackwell Unified Tile Algebra} (\BUTA{})---that
spans layout transformations, pipeline scheduling,
data-movement semantics, and analytic cache modelling
under CUDA~${>}\,13$ and PTX~${>}\,9.0$.

% ------ 1.1 TC Dispatch ------
\subsection{Fifth-Generation Tensor Core Dispatch}
\label{sec:intro:tc}

Blackwell replaces the warp-group-level
\texttt{wgmma} instruction family (128~threads,
Hopper) with the warp-level \tcgen{}\texttt{.mma}
PTX opcode, dispatched by a \emph{single thread}
within a 32-thread warp~\cite{ARCH_BW}.
Microbenchmarks by Jarmusch and
Chandrasekaran~\cite{ARCH_BW} report:

\begin{itemize}
  \item Single-instruction latency of
        $11.0$--$11.4$ cycles,
        \emph{approximately constant} across tile
        sizes and precisions (FP16 inputs/outputs;
        \TMEM{} accumulators).
  \item A $2.9$--$11.6\times$ latency reduction over
        Hopper \texttt{wgmma} for matched tile sizes.
  \item Throughput scaling of up to $177\times$
        between FP64 and FP4, achieved via wider
        datapaths rather than deeper pipelining.
\end{itemize}

\noindent
The shift to single-thread dispatch fundamentally
changes scheduling: each warp may independently issue
MMA operations, exposing fine-grained instruction-level
parallelism but requiring new constraint-based
scheduling strategies~\cite{OPT_PIPE}.
Soi et al.~\cite{OPT_PIPE} demonstrate that an
ILP+SMT formulation for joint software-pipelining (SWP)
and warp specialisation (WS) can rediscover provably
optimal schedules for Flash Attention on both Hopper
and Blackwell, validating a constraint-first approach
for the new dispatch model.

% ------ 1.2 Memory Hierarchy ------
\subsection{Memory Hierarchy Restructuring}
\label{sec:intro:mem}

Blackwell introduces \emph{Tensor Memory} (\TMEM{}),
a 256\,KB per-SM on-chip memory dedicated to tensor
core operands and accumulators, with an internal
organisation of $512$~columns $\times$ $128$~lanes
$\times$ $32$~bits~\cite{ARCH_BW}.
\TMEM{} is \emph{asymmetric}: the tensor core writes
accumulator results \emph{directly} to \TMEM{}; data
must be explicitly copied to the register file (RF) for
epilogue computation via \texttt{tcgen05.cp}.
The new instruction family \texttt{tcgen05.ld},
\texttt{tcgen05.st}, \texttt{tcgen05.cp} is the
\emph{sole} interface to \TMEM{}; legacy instructions
(\texttt{wmma.load}, \texttt{ldmatrix}, etc.) cannot
access \TMEM{}~\cite{ARCH_BW}.

The wider memory hierarchy is parametrised as follows
(notation: cc\,$x$ denotes compute capability~$x$):

\smallskip
\begin{center}
\begin{tabular}{@{}lll@{}}
\toprule
\textbf{Tier} & \textbf{cc\,10.0 (B200)}
              & \textbf{cc\,12.0 (GB10)} \\
\midrule
RF            & 256\,KB/SM  & 256\,KB/SM \\
\TMEM{}       & 256\,KB/SM  & 256\,KB/SM \\
\SMEM{}       & 228\,KB/SM  & 128\,KB/SM \\
L1+Tex+\SMEM{} & 256\,KB/SM (total) & --- \\
L2            & ${\sim}65$\,MB unified
              & ${\sim}65$\,MB unified \\
HBM3e         & 192\,GB, 8\,TB/s
              & Grace-local \\
\bottomrule
\end{tabular}
\end{center}
\smallskip

\noindent
Jarmusch and Chandrasekaran~\cite{ARCH_BW} report a
58\% reduction in cache-miss memory-access latency
(${\approx}420$~cycles vs.\ ${\approx}1000$~cycles on
H200) and \TMEM{} hit rates of 61--82\% in multi-stage
tensor pipelines.
The \SMEM{} capacity difference between cc\,10.0
(228\,KB) and cc\,12.0 (128\,KB) has direct modelling
consequences: tiling constraints and pipeline depth
differ across the two Blackwell sub-families.

Zhu, Pan, and Ding~\cite{SEED_3} demonstrate that for
streaming attention workloads on GB10, the L1 cache
provides \emph{negligible} benefit; L2 is the dominant
cache tier.
Their reuse-distance model correlates L2 hit rate with
active SM count, reducing L2 sector misses by 67\%
(370M$\to$120M sectors) through a sawtooth tile
ordering strategy.

% ------ 1.3 Tile IR ------
\subsection{Tile-Based Programming Abstraction}
\label{sec:intro:tileir}

CUDA~13.1 introduces \emph{CUDA Tile~IR}, an
MLIR-based virtual ISA for tile-based
programs~\cite{NV_BLOG_TILE}.
Tile~IR targets Blackwell exclusively (cc\,10.x and
12.x) and defines a novel memory model:

\begin{itemize}
  \item \textbf{Program-ordered operations}: ordered by
        program text position (traditional semantics).
  \item \textbf{Token-ordered operations}: produce and
        consume abstract \emph{tokens} that establish a
        \emph{partial order} on memory effects.
  \item All memory operations produce results
        immediately; observable effect ordering requires
        explicit token dependencies.
\end{itemize}

\noindent
For dense matrix-multiply workloads, the Tile~IR
parameter \texttt{num\_ctas=2} is
\emph{critical}~\cite{NV_BLOG_TILE}: it enables
2CTA~MMA mode, in which a CTA pair mapped to the same
Texture Processing Cluster (TPC) shares input operands
from \SMEM{}, halving per-SM operand bandwidth
pressure.
The Tensor Memory Access (\tma{}) API is the preferred
data-movement primitive; tensor-of-pointer access
patterns exhibit poor performance~\cite{NV_BLOG_TILE}.

% ------ 1.4 Workload Diversity ------
\subsection{Workload Diversity and Mixed-Reuse
            Patterns}
\label{sec:intro:workloads}

Garg, Pellauer, and Krishna~\cite{NV_workloads}
provide a two-axis taxonomy for heterogeneous
processors:
(i)~\emph{depth of compute} (leaf-only vs.\
hierarchical) and
(ii)~\emph{location of heterogeneity} (homogeneous,
intra-node, cross-node, cross-depth, compound).
Under this classification, the B200 SM is an
\emph{intra-node heterogeneous} processor: CUDA cores
and 5th-generation tensor cores are distinct
sub-accelerators within a single SM, with asymmetric
latencies and data paths~\cite{NV_workloads,ARCH_BW}.

Modern workloads exhibit \emph{mixed reuse}: a single
model contains tensor operators with diverse arithmetic
intensities---from compute-bound MMA tiles to
memory-bound pointwise epilogues.
Formal optimisation must therefore expose the workload
taxonomy as a first-class object so that solver cost
vectors can be conditioned on per-operator
characteristics~\cite{NV_workloads}.

% ------ 1.5 Thesis ------
\subsection{Thesis: The Case for a Unified Formal Model}
\label{sec:intro:thesis}

Recent work has advanced \emph{individual} formal
pillars: $\Ftwo$ binary matrices for linear tensor
layouts~\cite{SEED_1}; integer set (\ISL{})
quasi-affine relations for unified layout
analysis~\cite{SEED_2}; categorical (Tuple/Nest)
foundations for \CuTe{} layout
algebra~\cite{SEED_4}; ILP+SMT constraint
systems for joint SWP+WS
scheduling~\cite{OPT_PIPE}; and reuse-distance
models for L2 cache traffic
prediction~\cite{SEED_3}.
However, no existing framework \emph{composes}
these pillars into a single model that is
simultaneously:

\begin{enumerate}[label=(\roman*)]
  \item \emph{layout-algebraic}---capturing tile
        coordinate transformations across \SMEM{},
        \TMEM{}, and global memory;
  \item \emph{schedule-optimal}---encoding SWP, WS,
        and 2CTA constraints as a solvable optimisation
        problem;
  \item \emph{data-movement-correct}---respecting the
        Tile~IR token-ordered memory model; and
  \item \emph{cache-aware}---predicting L2 sector
        traffic as a function of tile iteration order.
\end{enumerate}

\noindent
\textbf{Core question.}
Can we construct a compositional formal model over
tile-based programs targeting Blackwell that jointly
captures layout transformations (as algebraic
morphisms), pipeline scheduling (as constraint
optimisation), data movement (as token-ordered
descriptor relations), and cache traffic (as analytic
reuse models), such that the composition provably
preserves semantic correctness while enabling automated
optimisation under CUDA~${>}\,13$ and PTX~${>}\,9.0$?

\smallskip\noindent
\BUTA{} answers this question in the affirmative by
defining seven mathematical objects
(\cref{sec:prelim}) whose composition is governed by
layered invariants.
Parts~2--4 of this proposal develop the full model
(\cref{sec:prelim} provides the necessary background),
its solver encodings, and an empirical validation plan.


% ====================================================================
%  SECTION 2 — FORMAL PRELIMINARIES
% ====================================================================
\section{Formal Preliminaries}
\label{sec:prelim}

This section establishes the mathematical machinery
required by \BUTA{}.
Each subsection introduces one formalism, states its
core definitions, and identifies the golden-source
anchor.

% ------ 2.1 F_2 Linear Layouts ------
\subsection{Linear Layouts over $\Ftwo$}
\label{sec:prelim:f2}

Zhou, Lezcano, Goucher et al.~\cite{SEED_1}
observe that a large class of tensor layouts used in
GPU compilers can be represented as binary matrices
over the two-element field~$\Ftwo$.

\begin{definition}[Linear Layout~\cite{SEED_1}]
\label{def:linear-layout}
A \emph{linear layout} is a matrix
$L \in \Ftwo^{m \times n}$ that maps $n$~hardware
address bits to $m$~logical coordinate bits.
Given a hardware index
$\mathbf{b} = (b_0,\dots,b_{n-1})^\top \in \Ftwo^n$,
the logical coordinate is
$\mathbf{c} = L\,\mathbf{b} \in \Ftwo^m$,
where all arithmetic is over~$\Ftwo$.
\end{definition}

\begin{definition}[Layout Composition and
  Conversion~\cite{SEED_1}]
\label{def:f2-compose}
Given layouts $L_1 \in \Ftwo^{m \times n}$ and
$L_2 \in \Ftwo^{p \times m}$, their composition is
$L_2 \circ L_1 = L_2\, L_1 \in \Ftwo^{p \times n}$.
The \emph{layout-to-layout conversion} from~$L_1$
to~$L_2$ is computed as
$L_2 \, L_1^{-1}$ (or pseudoinverse when $L_1$ is
non-square), eliminating the quadratic explosion in
layout-handling code that plagued prior Triton
implementations.
\end{definition}

\begin{remark}
The $\Ftwo$ representation is \emph{restricted to
linear layouts}: non-linear swizzle patterns (e.g.,
XOR-based bank-conflict avoidance) fall outside this
fragment and require the ISL extension of
\cref{sec:prelim:isl}~\cite{SEED_1,SEED_2}.
\end{remark}

% ------ 2.2 ISL Relations ------
\subsection{Integer Set Relations and Quasi-Affine
            Mappings}
\label{sec:prelim:isl}

Bhaskaracharya, Acharya, Hagedorn, and
Grover~\cite{SEED_2} show that both \CuTe{}
stride-based layouts and Triton $\Ftwo$-based
linear layouts can be unified as \emph{quasi-affine
integer set relations} in the \ISL{}
framework~\cite{SEED_2}.

\begin{definition}[\ISL{} Layout Relation~\cite{SEED_2}]
\label{def:isl-relation}
An \ISL{} relation is a map
$R : \ZZ^m \to \ZZ^n$ defined by a conjunction of
quasi-affine inequality and equality constraints.
A \CuTe{} layout $H = (\mathit{shape},\mathit{stride})$
induces a \emph{coordinate mapping}
\[
  M_H(c) \;=\;
    \sum_{i}\!\Bigl(
      \bigl\lfloor c / \sigma_i\bigr\rfloor
      \bmod s_i
    \Bigr) \cdot d_i\,,
\]
where $s_i,d_i,\sigma_i$ denote per-mode shape, stride,
and cumulative stride, respectively.
Algorithm~1 of~\cite{SEED_2} converts any such layout
to an \ISL{} relation.
\end{definition}

\begin{definition}[Tiling and Swizzle~\cite{SEED_2}]
\label{def:isl-tile}
Tiling is a quasi-affine transformation
$\mathcal{C} = \{c \to [c \bmod s_0,\;
\lfloor c/s_0\rfloor \bmod s_1,\;\dots]\}$.
Swizzle is modelled as a bit-level integer set
operation within \ISL{}, capturing non-linear
permutations that exceed the $\Ftwo$ fragment.
\end{definition}

\begin{definition}[Layout Size and Co-size~\cite{SEED_2}]
\label{def:size-cosize}
For layout relation $M_H$:
$|H| = \max_{\mathrm{lex}}\!\operatorname{dom}(M_H)+1$
(\emph{size}); \;
$\|H\| = \max_{\mathrm{lex}}\!\operatorname{ran}(M_H)+1$
(\emph{co-size}).
\end{definition}

% ------ 2.3 Categorical Foundations ------
\subsection{Categorical Foundations for Layout Algebra}
\label{sec:prelim:cat}

Carlisle, Shah, Stern, and
VanKoughnett~\cite{SEED_4} place \CuTe{} layouts on
a categorical footing via two small categories.

\begin{definition}[Categories
  \textbf{Tuple} and \textbf{Nest}~\cite{SEED_4}]
\label{def:tuple-nest}
\begin{enumerate}[label=(\alph*)]
  \item \textbf{Tuple}: objects are tuples
        $(n_1,\dots,n_k)$ with $n_i \in \NN$;
        morphisms are coordinate-level layout maps.
  \item \textbf{Nest}: objects are \emph{nested}
        tuple structures (trees with natural-number
        leaves); morphisms are hierarchical layout
        maps respecting nesting.
\end{enumerate}
The three core \CuTe{} layout operations---composition
($\circ$), logical product ($\otimes$), and logical
division ($/$)---are shown to be
\emph{compatible} with the categorical operations on
Tuple/Nest morphisms~\cite{SEED_4}.
\end{definition}

\begin{definition}[Tractable
  Layout~\cite{SEED_4}]
\label{def:tractable}
A layout is \emph{tractable} if and only if it arises
from the Tuple/Nest categorical construction.
Carlisle et al.~\cite{SEED_4} provide a
\emph{complete characterisation} of which layouts are
tractable, validated against the CUTLASS library.
\end{definition}

% ------ 2.4 Layout Hierarchy ------
\subsection{Layout Formalism Hierarchy}
\label{sec:prelim:hierarchy}

The three layout formalisms organise hierarchically
\inference{}:

\begin{figure}[H]
\centering
\begin{BVerbatim}[fontsize=\scriptsize]
+==========================================+
| Categorical Foundation   [SEED_4]        |
| Categories: Tuple, Nest                  |
| Scope: complete layout characterisation  |
| Ops: compose, logical prod, logical div  |
+---------+--------------------------------+
          | Functor F   [INFERENCE, gap G2]
          v
+==========================================+
| ISL Quasi-Affine Relations [SEED_2]      |
| R : Z^m -> Z^n (affine constraints)      |
| Algorithm 1: CuTe -> ISL conversion      |
| Scope: linear + non-linear swizzle       |
| Cross-system: CuTe + Triton unified      |
+---------+--------------------------------+
          | Embedding phi [INFERENCE, gap G1]
          v
+==========================================+
| F_2 Binary Matrices      [SEED_1]        |
| L in F_2^{m x n}                         |
| Compose: matrix multiply over F_2        |
| Scope: linear layouts only               |
| Cost: O(m*n) per operation               |
+==========================================+
\end{BVerbatim}
\caption{Hierarchical organisation of layout
  formalisms.
  Categorical morphisms (semantic foundation)
  embed into \ISL{} relations (analytical tool),
  which in turn contain $\Ftwo$ matrices
  (efficient computation for the linear sublattice).
  Embedding functors $F$ and $\phi$ are proof
  obligations for \BUTA{} (gaps G1, G2).}
\label{fig:layout-hierarchy}
\end{figure}

\noindent
The embedding $\phi : \Ftwo^{m\times n} \hookrightarrow
\ISL{}\text{-}\mathrm{Rel}$ and the functor
$F : \mathbf{Nest} \to \ISL{}\text{-}\mathrm{Rel}$
are \emph{not} established in the existing
literature---they are proof obligations for \BUTA{}
(gaps G1 and~G2; see Part~4).

% ------ 2.5 Constraint-Based Scheduling ------
\subsection{Constraint-Based Scheduling: ILP and SMT}
\label{sec:prelim:sched}

Soi et al.~\cite{OPT_PIPE} formalise tensor-core
pipeline scheduling as a constraint-satisfaction
problem over a tile-level dependence graph.

\begin{definition}[Dependence Graph~\cite{OPT_PIPE}]
\label{def:dep-graph}
$G = (V, E)$ where $V$ is the set of tile-level
operations (MMA, TMA load, TMA store, ALU, \TMEM{}
copy) and $E$ is the set of directed edges.
Each edge $(u,v) \in E$ carries a \emph{latency}
$d(u,v) \in \NN$ and a \emph{distance}
$\delta(u,v) \in \ZZ$ (iteration offset for
software-pipelined loops).
\end{definition}

\begin{definition}[Modulo Schedule~\cite{OPT_PIPE}]
\label{def:modulo-sched}
A \emph{modulo schedule} is a pair $(M, \II)$ where
$M : V \to \NN$ assigns a time slot to each operation
and $\II \in \NN^+$ is the \emph{initiation interval}
(the number of cycles between successive loop-body
initiations).
The core feasibility constraint for every dependency
$(u,v,d,\delta) \in E$ is:
\begin{equation}
  M(v) - M(u) + \II\cdot\delta \;\geq\; d(u,v)\,.
  \label{eq:dep-constraint}
\end{equation}
\end{definition}

\begin{definition}[Joint SWP+WS via
  ILP+SMT~\cite{OPT_PIPE}]
\label{def:swp-ws}
A three-dimensional boolean array
$\mathrm{op}[v, i, t] \in \{0,1\}$
indicates whether operation $v$ from iteration~$i$ is
scheduled at time~$t$.
In addition, a warp-assignment function
$W : V \to \{0,\dots,N_W{-}1\}$
maps operations to warps.
The ILP determines a lower bound on~$\II$; the SMT
encoding holistically captures the joint SWP+WS
problem.
Cost normalisation as a Zero-One Linear Program (ZLP)
renders absolute cycle-count ratios tractable for the
solver~\cite{OPT_PIPE}.
\end{definition}

\begin{remark}
The Twill system~\cite{OPT_PIPE} validates this
formulation by rediscovering expert-designed Flash
Attention schedules on both Hopper and Blackwell.
The system is evaluated on CUDA~13.0~\cite{OPT_PIPE}.
However, the formulation is restricted to singly-nested
loops; orthogonal optimisations such as memory layout
selection and register allocation are not
handled~\cite{OPT_PIPE}.
\end{remark}

% ------ 2.6 Reuse Distance ------
\subsection{Reuse Distance and Analytic Cache Models}
\label{sec:prelim:cache}

Zhu, Pan, and Ding~\cite{SEED_3} develop an analytic
L2 cache traffic model for tile-based programs on
the Grace-Blackwell GB10.

\begin{definition}[Reuse Distance~\cite{SEED_3}]
\label{def:reuse-dist}
For a cache line $a$, the \emph{reuse distance}
$\reuse(a)$ is the count of \emph{distinct} cache
lines accessed between two consecutive accesses to~$a$
(LRU stack distance).
Under an LRU replacement policy with capacity~$C$:
\[
  a \text{ is a hit} \iff
  \reuse(a) < \frac{C}{\text{sector\_size}}\,.
\]
\end{definition}

\begin{definition}[Sawtooth
  Ordering~\cite{SEED_3}]
\label{def:sawtooth}
The \emph{sawtooth order} alternates the inner-loop
scan direction across iterations:
iteration~$j$ scans
$0 \to N$ if $j$ is even and
$N \to 0$ if $j$ is odd.
For a streaming split-Q Flash Attention dataflow with
persistent CTAs, sawtooth ordering guarantees
$\reuse(a) < D_{\mathrm{data}}$ for a fraction
$\geq (1 - 1/N_{\mathrm{CTA}})$
of sector accesses, where
$D_{\mathrm{data}}$ is the total data
footprint~\cite{SEED_3}.
This yields a 67\% reduction in L2 misses on GB10
(370M $\to$ 120M sectors) and up to 60\% throughput
increase (41 $\to$ 66~TFLOPS causal)~\cite{SEED_3}.
\end{definition}

\begin{remark}
The model is validated \emph{only} for Flash Attention
on GB10 (cc\,12.0, 128\,KB \SMEM{}).
Generalisation to GEMM, convolution, or B200
(228\,KB \SMEM{}) is an \inference{} claim and a gap
(G5) addressed in Part~4.
L1 is negligible for streaming attention patterns on
GB10~\cite{SEED_3}; non-streaming patterns (where L1
may matter) require separate treatment.
\end{remark}

% ------ 2.7 Blackwell Arch Parameters ------
\subsection{Blackwell Architectural Parameters}
\label{sec:prelim:arch}

We summarise the key hardware parameters that
parameterise \BUTA{}, all drawn from
microbenchmarks~\cite{ARCH_BW}.

\begin{figure}[t]
\centering
\begin{BVerbatim}[fontsize=\scriptsize]
+==================================================+
|           Blackwell SM  (cc 10.0 / B200)         |
|--------------------------------------------------|
| [WS-0] [WS-1] [WS-2] [WS-3]  4 Warp Schedulers |
|        up to 64 concurrent warps (cc 10.0)       |
|        up to 48 concurrent warps (cc 12.0)       |
|--------------------------------------------------|
| +------+ +------+ +------+                       |
| |INT32/| |INT32/| | SFU/ |   CUDA Core Clusters  |
| |FP32  | |FP32  | |Trans.|                       |
| +------+ +------+ +------+                       |
|--------------------------------------------------|
| +----------------------------------------------+ |
| | 5th-Gen Tensor Core  (tcgen05 PTX family)    | |
| | Precision: FP4|FP6|FP8|FP16|BF16|TF32|FP64  | |
| | SASS:      QMMA|OMMA|HMMA|DMMA|IMMA         | |
| | Dispatch:  warp-level, single-thread MMA     | |
| | Latency:   11.0--11.4 cycles (approx const)  | |
| | MMA.2SM:   CTA pair on TPC shares operands   | |
| +----------------------------------------------+ |
|--------------------------------------------------|
| RF: 256 KB       | TMEM: 256 KB                  |
| (64KB x 4 parts) | (512 col x 128 lane x 32-bit) |
|--------------------------------------------------|
| SMEM: 228 KB (cc 10.0) / 128 KB (cc 12.0)       |
| L1+Tex+SMEM = 256 KB total (cc 10.0)            |
+==================================================+
   |                         |
   v                         v
+------------------+  +------------------+
|  L2 Cache        |  |  L2 Cache        |
|  ~65 MB unified  |  |  (8 x 8MB slice) |
+------------------+  +------------------+
         |
         v
+------------------+
|  HBM3e           |
|  192 GB, 8 TB/s  |
|  Dual-die NV-HBI |
+------------------+
\end{BVerbatim}
\caption{Blackwell SM micro\-architecture and memory
  hierarchy, parametrised from~\cite{ARCH_BW}.
  B200 has 148~SMs across 8~GPCs (74 per die, dual-die
  via NV-HBI).
  Precision set and SASS mappings support the full
  lattice \{FP4,FP6,FP8,FP16,BF16,TF32,FP64,INT8\}.}
\label{fig:sm-arch}
\end{figure}

\begin{definition}[Blackwell SM
  Resource Bounds~\cite{ARCH_BW}]
\label{def:sm-resource}
For compute capability $x \in \{10.0, 12.0\}$, an
SM~$s$ has resources:
\begin{align}
  \mathcal{R}(s, x) &= \bigl(
    N_{\mathrm{WS}}(x),\;
    C_{\TMEM{}},\;
    C_{\SMEM{}}(x),\;
    C_{\mathrm{RF}},\;
    \tau_{\mathrm{MMA}}
  \bigr) \notag \\
  &= \begin{cases}
    (64,\; 256,\; 228,\; 256,\; 11.2)
      & x = 10.0 \\
    (48,\; 256,\; 128,\; 256,\; 11.2)
      & x = 12.0
  \end{cases} \label{eq:sm-resource}
\end{align}
where $N_{\mathrm{WS}}$ is the maximum concurrent warp
count, $C_{\cdot}$ is capacity in KB, and
$\tau_{\mathrm{MMA}}$ is the representative
\tcgen{}\texttt{.mma} latency in cycles.
The SM contains 4~warp schedulers, each capable of
dispatching to distinct functional
units~\cite{ARCH_BW}.
\end{definition}

\begin{definition}[Precision
  Lattice~\cite{ARCH_BW}]
\label{def:prec-lattice}
The Blackwell precision set is
$\mathcal{P} = \{\texttt{FP64},\, \texttt{TF32},\,
\texttt{BF16},\, \texttt{FP16},\, \texttt{FP8},\,
\texttt{FP6},\, \texttt{FP4},\, \texttt{INT8}\}$,
partially ordered by bit-width refinement.
Each $p \in \mathcal{P}$ maps to a SASS instruction
class $\sigma(p)$:
\[
  \sigma : \mathcal{P} \to
  \{\textsc{DMMA},\,\textsc{HMMA},\,
    \textsc{QMMA},\,\textsc{OMMA},\,\textsc{IMMA}\}.
\]
Throughput scales up to $177\times$ from FP64 to FP4;
FP32 accumulation halves throughput relative to FP16
accumulation~\cite{ARCH_BW}.
FP4 and FP6 are new to Blackwell with no Hopper
datapaths.
\end{definition}

\begin{remark}
\TMEM{} read bandwidth of ${\approx}16$\,TB/s per SM
and write bandwidth of ${\approx}8$\,TB/s per SM are
reported in secondary analysis \unverified{}; these
require independent calibration
(see Part~3 validation plan).
\end{remark}

% ------ 2.8 CUDA Tile IR ------
\subsection{CUDA Tile~IR: Virtual ISA and Memory Model}
\label{sec:prelim:tileir}

NVIDIA's Tile~IR~\cite{NV_BLOG_TILE}
is the compiler target for tile-based programs on
Blackwell, lowered from OpenAI Triton via a
Triton-to-TileIR backend.

\begin{definition}[Tile~IR Memory
  Model~\cite{NV_BLOG_TILE}]
\label{def:tileir-mem}
The Tile~IR memory model partitions operations into
two classes:
\begin{enumerate}[label=(\alph*)]
  \item \emph{Program-ordered}: ordered by textual
        position in the IR program.
  \item \emph{Token-ordered}: each operation may
        consume zero or more input tokens and produce
        zero or more output tokens.
        The \emph{waits-for} relation is the transitive
        closure of token dependencies.
        The \emph{happens-before} relation is
        established by release/acquire pairs on the
        same memory location.
\end{enumerate}
A data race exists within a tile block if two
operations access the same location, at least one
writes, and neither waits-for the other.
\end{definition}

\begin{definition}[\tma{} Descriptor
  Operations~\cite{NV_BLOG_TILE}]
\label{def:tma}
\tma{} provides bulk asynchronous data movement via
typed descriptors.
Each \tma{} operation is characterised by a tuple:
\[
  \mathrm{tma\_op} = (\mathrm{src\_tier},\;
    \mathrm{dst\_tier},\;
    \mathrm{desc},\;
    \tau_{\mathrm{in}},\;
    \tau_{\mathrm{out}})
\]
where $\mathrm{desc} \in \{\texttt{idesc},
\texttt{sdesc}\}$ is the descriptor type and
$\tau_{\mathrm{in}}, \tau_{\mathrm{out}}$ are consumed
and produced tokens, respectively.
\end{definition}

\begin{definition}[2CTA~MMA
  Mode~\cite{NV_BLOG_TILE,ARCH_BW}]
\label{def:2cta}
When \texttt{num\_ctas=2}, two CTAs with adjacent
cluster ranks form a \emph{CTA pair} on a single TPC
(two SMs).
The pair shares input operands~$A$ and~$B$ from
\SMEM{}, reducing per-SM operand loading by
${\sim}50\%$.
The MMA.2SM instruction leverages TPC-internal
bandwidth that exceeds distributed shared memory
(DSMEM) bandwidth~\cite{ARCH_BW}.
This mode is critical for saturating tensor-core
throughput on dense dot
workloads~\cite{NV_BLOG_TILE}.
\end{definition}

\begin{remark}
Tile~IR (CUDA~13.1) does not expose
\texttt{num\_warps} as a tuning
parameter~\cite{NV_BLOG_TILE}, creating a tension
with the warp-level scheduling variables required by
the OPT\_PIPE formulation~\cite{OPT_PIPE}.
An occupancy hint (integer 1--32) is the only
available per-block resource control.
\end{remark}

% ------ 2.9 Workload Taxonomy ------
\subsection{Heterogeneous Processor Taxonomy}
\label{sec:prelim:taxonomy}

\begin{definition}[Two-Axis Workload
  Classification~\cite{NV_workloads}]
\label{def:harp}
The HARP taxonomy classifies processors along two
axes:
\begin{enumerate}[label=(\alph*)]
  \item \emph{Depth of compute}: leaf-only (compute
        at L1 leaves) vs.\ hierarchical (compute
        across hierarchy levels).
  \item \emph{Location of heterogeneity}:
        homogeneous, intra-node, cross-node,
        cross-depth, compound.
\end{enumerate}
A GPU with tensor cores is classified as
\emph{intra-node heterogeneous}; B100 is classified as
\emph{leaf-only}~\cite{NV_workloads}.
\emph{Mixed-reuse workloads} contain tensor operators
with diverse arithmetic intensities, requiring
heterogeneous scheduling
strategies~\cite{NV_workloads}.
\end{definition}

\begin{remark}
\inference{}: Applying the HARP taxonomy to a single
Blackwell SM yields a formal object
$\mathrm{SM} = (\text{CUDA\_cores},\;
\text{TC\_5thgen},\; \TMEM{},\; \SMEM{},\;
\text{RF})$
with distinct per-sub-accelerator latencies, forming
the cost vector consumed by the OPT\_PIPE
solver~\cite{OPT_PIPE, NV_workloads, ARCH_BW}.
The introduction of \TMEM{} challenges the pure
leaf-only classification: \TMEM{}-aware scheduling adds
a data-placement variable absent in prior
models.
\end{remark}


% ====================================================================
%  SECTION 3 — NOTATION AND RUNNING EXAMPLE
% ====================================================================
\section{Notation and Running Example}
\label{sec:notation}

% ------ 3.1 Notation Table ------
\subsection{Symbol Table}
\label{sec:notation:table}

\Cref{tab:notation} collects the principal symbols
used throughout Parts~1--4.

\begin{table*}[t]
\centering
\caption{Principal notation for \BUTA{}.
  Each symbol is anchored to its defining source.}
\label{tab:notation}
\small
\begin{tabular}{@{}clll@{}}
\toprule
\textbf{Symbol} & \textbf{Type} & \textbf{Description}
  & \textbf{Source} \\
\midrule
$L \in \Ftwo^{m \times n}$
  & matrix & Linear layout
  & \cite{SEED_1} \\
$M_H : \ZZ^m \to \ZZ^n$
  & \ISL{} relation & Layout coordinate mapping
  & \cite{SEED_2} \\
$\mathbf{Tuple},\,\mathbf{Nest}$
  & categories & Layout categories
  & \cite{SEED_4} \\
$G = (V, E)$
  & directed graph & Tile-level dependence graph
  & \cite{OPT_PIPE} \\
$M(v)$
  & $\NN$ & Schedule time for operation~$v$
  & \cite{OPT_PIPE} \\
$W(v)$
  & $\{0,\dots,N_W{-}1\}$
  & Warp assignment for operation~$v$
  & \cite{OPT_PIPE} \\
$\II$
  & $\NN^+$ & Initiation interval
  & \cite{OPT_PIPE} \\
$\mathrm{op}[v,i,t]$
  & $\{0,1\}$ & SWP+WS boolean schedule variable
  & \cite{OPT_PIPE} \\
$\reuse(a)$
  & $\NN$ & Reuse distance of cache line~$a$
  & \cite{SEED_3} \\
$\sigma$
  & permutation & Tile iteration order
  & \cite{SEED_3} \\
$\mathcal{H}$
  & labelled graph & Memory tier graph
  & \cite{ARCH_BW} \\
$\mathcal{T}$
  & DAG & Token dependency graph
  & \cite{NV_BLOG_TILE} \\
$\mathcal{P}$
  & poset & Precision lattice
  & \cite{ARCH_BW} \\
$\mathcal{W}$
  & set & Workload descriptor
  & \cite{NV_workloads} \\
$C_{\SMEM{}}(x)$
  & KB & \SMEM{} capacity for cc~$x$
  & \cite{ARCH_BW} \\
$C_{\TMEM{}}$
  & KB & \TMEM{} capacity (256\,KB)
  & \cite{ARCH_BW} \\
$\tau_{\mathrm{MMA}}$
  & cycles & \tcgen{}\texttt{.mma} latency
  & \cite{ARCH_BW} \\
$N_{\mathrm{SM}}$
  & $\NN$ & Active SM count
  & \cite{ARCH_BW,SEED_3} \\
\bottomrule
\end{tabular}
\end{table*}

% ------ 3.2 Running Example ------
\subsection{Running Example: Flash Attention Forward
            on Blackwell}
\label{sec:notation:fa}

We ground the abstract machinery in a concrete
instance: the forward pass of Flash Attention on a
Blackwell B200 (cc\,10.0) under 2CTA~MMA mode.
This workload exercises \emph{all} \BUTA{} layers
simultaneously and has been independently targeted by
multiple golden sources~\cite{OPT_PIPE,SEED_3,
NV_BLOG_TILE, ARCH_BW}.

\paragraph{Problem statement.}
Given query, key, and value matrices
$Q, K, V \in \mathbb{R}^{N \times d}$,
compute
$O = \mathrm{softmax}(QK^\top / \sqrt{d})\, V$.
The computation is tiled into blocks of size
$B_r \times B_c$ (rows of~$Q$ $\times$ columns
of~$K^\top/V$), with $Q$~tiles resident in \SMEM{}
and $K$,$V$ tiles streamed from global
memory~\cite{SEED_3}.

\paragraph{Memory tier mapping.}
\Cref{fig:fa-dataflow} illustrates the data flow.
\tma{} descriptors move $K$,$V$ tiles from HBM to
\SMEM{}~\cite{NV_BLOG_TILE}.
In 2CTA mode, both CTAs of a TPC pair read
shared operands from the aggregate \SMEM{} of the two
SMs~\cite{NV_BLOG_TILE, ARCH_BW}.
The tensor core executes
$S_{ij} = Q_i \cdot K_j^\top$ and
$O_i \mathrel{+}= P_{ij} \cdot V_j$ via
\tcgen{}\texttt{.mma}, writing accumulators to
\TMEM{}~\cite{ARCH_BW}.
Epilogue (online softmax rescaling) requires
\texttt{tcgen05.cp} to move partial results from
\TMEM{} to RF.

\begin{figure}[t]
\centering
\begin{BVerbatim}[fontsize=\scriptsize]
  Outer loop: Q blocks (persistent CTA, round-robin)
  |
  v
+-----+ TMA  +------+  operands  +------+ accum +------+
| HBM |----->| SMEM |----------->|  TC  |------>| TMEM |
|     | K,V  |228 KB|  A,B from  |5thgen|  D    |256 KB|
+-----+      |      |  SMEM      +------+       +--+---+
   ^         +------+                               |
   |            |                       tcgen05.cp   |
   |            |                                    v
   |   Q tile   |                               +------+
   |  (resident) |                               |  RF  |
   |            |                               |(epil.)|
   |            v                               +--+---+
   |   Inner loop: K_j, V_j tiles                  |
   |   Sawtooth order [SEED_3]:                    |
   |     even iter: j = 0..N                       v
   |     odd  iter: j = N..0                   +------+
   |   Reduces reuse distance -> L2 hits       | SMEM |
   |                                           |(write)|
   |   Schedule [OPT_PIPE]:                    +--+---+
   |     ILP -> min II                            |
   |     SMT -> joint SWP+WS                  TMA|store
   |     op[v,i,t] boolean encoding               v
   |                                           +------+
   +<------------------------------------------| HBM  |
          write-back O_i                       +------+
\end{BVerbatim}
\caption{Flash Attention forward-pass data flow on
  Blackwell B200 in 2CTA mode.
  $Q$ tiles are resident in \SMEM{}; $K$,$V$ tiles
  stream via \tma{} with sawtooth
  ordering~\cite{SEED_3}.
  Accumulator $O$ resides in \TMEM{} and is copied to
  RF for softmax rescaling~\cite{ARCH_BW}.
  The schedule is determined by the ILP+SMT constraint
  system~\cite{OPT_PIPE} under 2CTA
  constraints~\cite{NV_BLOG_TILE}.}
\label{fig:fa-dataflow}
\end{figure}

\paragraph{Layout layer.}
Each tile operand requires a layout mapping from
logical coordinates to physical addresses in its
resident memory tier.
For the $Q$ tile in \SMEM{}, a swizzled layout avoids
bank conflicts; this may be a non-linear pattern
outside the $\Ftwo$ fragment, requiring the \ISL{}
representation~\cite{SEED_1, SEED_2}.
The accumulator layout in \TMEM{} is constrained by
the 512-column structure~\cite{ARCH_BW}; its
characterisation as a Tuple/Nest morphism determines
whether the layout is \emph{tractable} in the sense of
\cref{def:tractable}~\cite{SEED_4}.

\paragraph{Scheduling layer.}
The dependence graph $G = (V, E)$ includes:
MMA operations for $S$ and $O$ updates,
\tma{} loads for $K_j$ and $V_j$,
\texttt{tcgen05.cp} for accumulator-to-RF transfer,
and ALU operations for online softmax.
The ILP+SMT system~\cite{OPT_PIPE} minimises $\II$
subject to dependency constraints
(\cref{eq:dep-constraint}), resource bounds
($\leq 4$ execution contexts per
SM~\cite{ARCH_BW,OPT_PIPE}), and the 2CTA pairing
constraint
(\texttt{num\_ctas=2})~\cite{NV_BLOG_TILE}.
Soi et al.~\cite{OPT_PIPE} demonstrate that this
system \emph{rediscovers} expert Flash Attention
schedules on Blackwell.

\paragraph{Data movement layer.}
Token-ordered correctness~\cite{NV_BLOG_TILE}
requires that every \tma{} load of $K_j$,$V_j$
completes (token produced) before the consuming MMA
(token consumed).
The token dependency DAG $\mathcal{T}$ must be acyclic
within each tile block.
Capacity invariants ensure: live tiles in \SMEM{}
$\leq C_{\SMEM{}}(10.0) = 228$\,KB;
live accumulators in \TMEM{}
$\leq C_{\TMEM{}} = 256$\,KB~\cite{ARCH_BW}.

\paragraph{Cache traffic layer.}
The streaming $K$,$V$ access pattern makes L1
negligible~\cite{SEED_3}.
The tile iteration order~$\sigma$ becomes an
optimisation variable: sawtooth ordering reduces
$\reuse(a)$ below $D_{\mathrm{data}}$ for most
accesses, yielding a predicted 67\% L2 miss
reduction~\cite{SEED_3}.
The predicted miss count is
\begin{equation}
  \mathcal{M}(\sigma, N_{\mathrm{SM}}, C_{\mathrm{L2}})
  = \sum_{a \in \mathrm{sectors}}
    \mathbf{1}\!\bigl[\reuse(a, \sigma) >
    C_{\mathrm{L2}} / s\bigr]
  \label{eq:miss-predict}
\end{equation}
where $s = 32$\,bytes (sector size) and
$C_{\mathrm{L2}} \approx 65$\,MB on
B200~\cite{ARCH_BW, SEED_3}.
L2 hit rate correlates with $N_{\mathrm{SM}}$:
increasing active SMs leads to non-decreasing miss
rates for streaming patterns~\cite{SEED_3}.

\paragraph{Workload classification.}
Under the HARP taxonomy~\cite{NV_workloads}, Flash
Attention forward is a mixed-reuse workload: the MMA
tiles (high arithmetic intensity) coexist with the
softmax epilogue (low arithmetic intensity, memory
bound).
The Blackwell SM, classified as intra-node
heterogeneous~\cite{NV_workloads}, dispatches these to
distinct sub-accelerators (tensor core vs.\ CUDA
cores) with asymmetric costs in the scheduling cost
vector~\cite{ARCH_BW}.

\paragraph{Synthesis.}
This example illustrates how the four \BUTA{} layers
interact on a single workload and motivates the
compositional invariant structure developed in Part~2.
Every golden source contributes at least one constraint
or parameter to the running example:
layout algebra (\cite{SEED_1,SEED_2,SEED_4}),
scheduling (\cite{OPT_PIPE}),
data movement (\cite{NV_BLOG_TILE}),
cache modelling (\cite{SEED_3}),
hardware parameterisation (\cite{ARCH_BW}),
and workload classification (\cite{NV_workloads}).


% ====================================================================
%  SOURCE AUDIT
% ====================================================================
\section*{Source Audit (Part~1)}
\label{sec:audit}

\noindent
All eight golden sources are cited in this part.
\Cref{tab:audit} records usage.

\begin{table}[H]
\centering
\caption{Part~1 source audit: usage and anchors.}
\label{tab:audit}
\footnotesize
\begin{tabular}{@{}p{1.3cm}p{6.0cm}@{}}
\toprule
\textbf{ID} & \textbf{Used for (anchor)} \\
\midrule
ARCH\_BW &
  SM topology (148 SMs, 4 WS, 64 warps cc\,10.0);
  \TMEM{} 256\,KB/SM; \SMEM{} 228/128\,KB;
  L2 ${\sim}$65\,MB; \tcgen{}.mma latency 11.0--11.4;
  precision lattice; 58\% cache-miss latency reduction;
  1.56$\times$ mixed-precision throughput vs H200. \\
\addlinespace
OPT\_PIPE &
  Dependence graph $G{=}(V,E)$; ILP for modulo scheduling;
  SMT for joint SWP+WS via $\mathrm{op}[v,i,t]$; ZLP
  normalisation; Twill rediscovery of FA schedules;
  evaluated on CUDA~13.0; singly-nested loop restriction. \\
\addlinespace
NV\_BLOG\_ TILE &
  Tile~IR virtual ISA (CUDA~13.1); unordered memory model
  with token-ordered operations; \texttt{num\_ctas=2}
  for dense MMA; \tma{} preferred over tensor-of-pointer;
  occupancy hint; \texttt{num\_warps} not exposed. \\
\addlinespace
NV\_work\-loads &
  Two-axis taxonomy (depth of compute $\times$
  heterogeneity); B100 leaf-only; GPU+TC intra-node;
  mixed-reuse workload characterisation. \\
\addlinespace
SEED\_1 &
  $\Ftwo$ binary matrix representation of linear layouts;
  composition via matrix multiplication; layout-to-layout
  conversion; quadratic explosion elimination. \\
\addlinespace
SEED\_2 &
  \ISL{} quasi-affine relations; Algorithm~1 (CuTe$\to$ISL);
  tiling as quasi-affine transformation; swizzle as bit-level
  ISL operation; layout size and co-size. \\
\addlinespace
SEED\_3 &
  Reuse distance formalism; sawtooth ordering; 67\% L2 miss
  reduction on GB10; 60\% throughput increase; L1 negligible
  for streaming attention; L2 hit rate $\sim$
  $f(N_{\mathrm{SM}})$. \\
\addlinespace
SEED\_4 &
  Tuple/Nest categories and morphisms; compatibility with
  CuTe operations (composition, logical product, logical
  division); tractable layout characterisation; CUTLASS
  validation. \\
\bottomrule
\end{tabular}
\end{table}


% ====================================================================
%  BIBLIOGRAPHY
% ====================================================================
\begin{thebibliography}{8}

\bibitem{ARCH_BW}
A.~Jarmusch and S.~Chandrasekaran,
``Microbenchmarking NVIDIA's Blackwell Architecture:
An In-Depth Architectural Analysis,''
\emph{arXiv:2512.02189v1}, Dec.~2025.
\url{https://arxiv.org/html/2512.02189v1}

\bibitem{OPT_PIPE}
K.~Soi, A.~Venkat, T.~Nowicki, and
G.~Biros,
``Optimal Software Pipelining and Warp
Specialization for Tensor Core GPUs,''
\emph{arXiv:2512.18134v1}, Dec.~2025.
\url{https://arxiv.org/html/2512.18134v1}

\bibitem{NV_BLOG_TILE}
NVIDIA,
``Advancing GPU Programming with the CUDA Tile IR
Backend for OpenAI Triton,''
\emph{NVIDIA Developer Blog}, Jan.~2026.
\url{https://developer.nvidia.com/blog/advancing-gpu-programming-with-the-cuda-tile-ir-backend-for-openai-triton/}

\bibitem{NV_workloads}
S.~Garg, M.~Pellauer, and T.~Krishna,
``HARP: A Taxonomy for Heterogeneous and
Hierarchical Processors for Mixed-Reuse Workloads,''
\emph{arXiv:2502.13113}, Feb.~2025.
\url{https://arxiv.org/html/2502.13113}

\bibitem{SEED_1}
J.~Zhou, M.~Lezcano, A.~Goucher, et~al.,
``Linear Layouts: Robust Code Generation of Efficient
Tensor Computation Using $\mathbb{F}_2$,''
\emph{arXiv:2505.23819v3}, Oct.~2025.
\url{https://arxiv.org/html/2505.23819v3}

\bibitem{SEED_2}
S.~Bhaskaracharya, N.~Acharya, M.~Hagedorn,
and S.~Grover,
``Modeling Layout Abstractions Using Integer Set
Relations,''
\emph{arXiv:2511.10374v1}, Nov.~2025.
\url{https://arxiv.org/html/2511.10374v1}

\bibitem{SEED_3}
Z.~Zhu, J.~Pan, and Y.~Ding,
``Sawtooth Wavefront Reordering: Enhanced CuTile
FlashAttention on NVIDIA GB10,''
\emph{arXiv:2601.16032v2}, Jan.~2026.
\url{https://arxiv.org/html/2601.16032v2}

\bibitem{SEED_4}
R.~Carlisle, A.~Shah, H.~Stern, and
D.~VanKoughnett,
``Categorical Foundations for CuTe Layouts,''
\emph{arXiv:2601.05972v1}, Jan.~2026.
\url{https://arxiv.org/pdf/2601.05972v1}

\end{thebibliography}

\end{document}
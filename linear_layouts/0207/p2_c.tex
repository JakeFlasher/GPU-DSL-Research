% ============================================================
%  BUTA — Blackwell Unified Tile Algebra
%  Part 2: Model Definition, Semantic Layers,
%          Objective Functions, and Proof Obligations
% ============================================================
\documentclass[10pt,twocolumn]{article}

% ---- Geometry & Typography ----
\usepackage[margin=0.85in,columnsep=0.25in]{geometry}
\usepackage[T1]{fontenc}
\usepackage{lmodern}
\usepackage{microtype}

% ---- Mathematics ----
\usepackage{amsmath,amssymb,amsthm,mathtools}

% ---- Figures & Tables ----
\usepackage{graphicx,booktabs,array,multirow}
\usepackage{fancyvrb}
\usepackage{float}

% ---- Cross-references & Links ----
\usepackage[colorlinks,citecolor=blue!70!black,
            linkcolor=blue!60!black,urlcolor=blue!50!black]{hyperref}
\usepackage[capitalise,noabbrev]{cleveref}

% ---- Misc ----
\usepackage{xcolor}
\usepackage{enumitem}
\setlist{nosep,leftmargin=*}

% ---- Theorem-like environments ----
\theoremstyle{definition}
\newtheorem{definition}{Definition}[section]
\newtheorem{invariant}[definition]{Invariant}
\newtheorem{remark}[definition]{Remark}
\theoremstyle{plain}
\newtheorem{proposition}[definition]{Proposition}
\newtheorem{obligation}[definition]{Proof Obligation}

% ---- Convenience macros (matching Part 1) ----
\newcommand{\BUTA}{\textsc{Buta}}
\newcommand{\Ftwo}{\mathbb{F}_2}
\newcommand{\ZZ}{\mathbb{Z}}
\newcommand{\NN}{\mathbb{N}}
\newcommand{\RR}{\mathbb{R}}
\newcommand{\ISL}{\textsc{ISL}}
\newcommand{\CuTe}{\textsc{CuTe}}
\newcommand{\TileIR}{Tile\,IR}
\newcommand{\TMEM}{\textsc{TMEM}}
\newcommand{\SMEM}{\textsc{SMEM}}
\newcommand{\tcgen}{\texttt{tcgen05}}
\newcommand{\tma}{\textsc{TMA}}
\newcommand{\II}{\mathit{II}}
\newcommand{\reuse}{d_{\mathrm{reuse}}}
\newcommand{\unverified}{\textsc{[unverified]}}
\newcommand{\inference}{\textsc{[inference]}}

% ---- Continue section numbering from Part 1 ----
\setcounter{section}{3}

% ---- Title ----
\title{%
  \BUTA{}: Blackwell Unified Tile Algebra\\[4pt]
  \large Part~2 --- Model Definition, Semantic Layers,\\
         Objective Functions, and Proof Obligations}

\author{%
  \textit{[Authors redacted for review]}\\
  Target architecture: NVIDIA Blackwell (B200 cc\,10.0;
  Grace-Blackwell GB10 cc\,12.0)\\
  Toolchain: CUDA ${>}\,13.0$ (prefer 13.1+),
  PTX ${>}\,9.0$ (strict)}

\date{February 2026}

\begin{document}
\maketitle

% ====================================================================
%  SECTION 4 — BUTA MODEL DEFINITION
% ====================================================================
\section{BUTA Model Definition}
\label{sec:model}

Part~1 established that four simultaneous architectural
shifts in Blackwell---fifth-generation tensor core
dispatch~\cite{ARCH_BW}, memory hierarchy
restructuring with \TMEM{}~\cite{ARCH_BW},
tile-based programming via CUDA
\TileIR{}~\cite{NV_BLOG_TILE}, and mixed-reuse
workload diversity~\cite{NV_workloads}---require
a single compositional formal model spanning layout
algebra, pipeline scheduling, data-movement semantics,
and analytic cache modelling.
This section defines the seven mathematical objects
that constitute \BUTA{} and specifies their
compositional architecture.

\subsection{Model Architecture Overview}
\label{sec:model:arch}

\begin{definition}[\BUTA{} Model Tuple]
\label{def:buta-tuple}
The \emph{Blackwell Unified Tile Algebra} is a
seven-component system:
\[
  \BUTA{} = (\mathcal{L},\;
             \mathcal{H},\;
             \mathcal{S},\;
             \mathcal{D},\;
             \mathcal{C},\;
             \mathcal{W},\;
             \mathcal{P})
\]
where $\mathcal{L}$ is the Layout Morphism Space
(\cref{sec:model:layout}),
$\mathcal{H}$ is the Memory Tier Graph
(\cref{sec:model:tier}),
$\mathcal{S}$ is the Schedule Constraint System
(\cref{sec:model:sched}),
$\mathcal{D}$ is the Data Movement Relation
(\cref{sec:model:datamov}),
$\mathcal{C}$ is the Cache Traffic Function
(\cref{sec:model:cache}),
$\mathcal{W}$ is the Workload Descriptor
(\cref{sec:model:workload}), and
$\mathcal{P}$ is the Precision Lattice
(\cref{sec:model:prec}).
\end{definition}

\noindent
\Cref{fig:buta-arch} illustrates the compositional
architecture.
The seven objects interact through typed interfaces:
$\mathcal{P}$ constrains valid layouts in
$\mathcal{L}$ and determines instruction costs in
$\mathcal{S}$;
$\mathcal{W}$ supplies per-operator cost vectors to
$\mathcal{S}$;
$\mathcal{L}$ determines address-translation costs
consumed by $\mathcal{D}$;
$\mathcal{D}$ is defined over the edges of
$\mathcal{H}$;
$\mathcal{S}$ determines tile iteration ordering which
parameterises $\mathcal{C}$;
and $\mathcal{H}$ supplies capacity bounds to both
$\mathcal{S}$ (resource constraints) and $\mathcal{C}$
(L2 capacity parameter).

\begin{figure}[t]
\centering
\begin{BVerbatim}[fontsize=\scriptsize]
+==========================================================+
|               BUTA Compositional Architecture             |
+==========================================================+
|                                                          |
|  +-----------+  precision   +-----------+                |
|  | Precision |--constrains->| Layout    |                |
|  | Lattice   |              | Morphism  |                |
|  |    P      |--costs----+  | Space  L  |                |
|  +-----------+           |  +-----+-----+                |
|                          |        |                      |
|  +-----------+  cost     |   addr | transforms           |
|  | Workload  |--vector---+       |                      |
|  | Desc.  W  |-----------+       v                      |
|  +-----------+           | +-----+-----+  token    +---------+
|                          +>| Schedule  |--order--->| Data    |
|                            | Constr.   |           | Move-   |
|                            | System S  |<--laten.--| ment  D |
|                            +-----+-----+           +----+----+
|                                  |                      |
|                          tile    | ordering        over |edges
|                                  v                      v
|                            +-----+-----+         +-----+-----+
|                            | Cache     |         | Memory    |
|                            | Traffic   |<--cap.--| Tier      |
|                            | Funct. C  |         | Graph  H  |
|                            +-----------+         +-----------+
|                                                          |
+==========================================================+
| Interface types:                                         |
|   --constrains-->  subset restriction on valid morphisms |
|   --costs-------->  numeric cost vector parameterisation |
|   --order-------->  partial order (token DAG)            |
|   --laten.------->  latency feedback to constraints      |
|   --cap.--------->  capacity bound parameter             |
|   --over edges--->  domain/codomain of relation          |
+==========================================================+
\end{BVerbatim}
\caption{Compositional architecture of the \BUTA{}
  seven-object system.
  Arrows denote typed interfaces between objects.
  The system is parameterised by compute capability
  $x \in \{10.0, 12.0\}$~\cite{ARCH_BW}
  and CUDA version ${\geq}\,13.1$~\cite{NV_BLOG_TILE}.}
\label{fig:buta-arch}
\end{figure}


% ------ 4.2 Layout Morphism Space ------
\subsection{Layout Morphism Space $\mathcal{L}$}
\label{sec:model:layout}

\begin{definition}[Layout Morphism
  Space~\cite{SEED_1,SEED_2,SEED_4}]
\label{def:layout-space}
The \emph{Layout Morphism Space} $\mathcal{L}$ is a
layered category defined by the triple
$\mathcal{L} =
(\mathrm{Obj}_{\mathcal{L}},\;
 \mathrm{Mor}_{\mathcal{L}},\;
 \Lambda)$
where:

\smallskip\noindent
\textbf{Objects.}\;
$\mathrm{Obj}_{\mathcal{L}} =
\{(T, \mu) : T \in \mathbf{TileShapes},\;
\mu \in \mathrm{Tiers}(\mathcal{H})\}$.
A tile shape $T = (n_1, \dots, n_k)$ is a tuple of
positive naturals (an object in $\mathbf{Tuple}$ per
\cite{SEED_4}); $\mu$ is a memory tier annotation
drawn from $\mathcal{H}$
(see \cref{def:tier-graph}).

\smallskip\noindent
\textbf{Morphisms.}\;
A morphism $f : (T_1, \mu_1) \to (T_2, \mu_2)$ is a
coordinate transformation mapping logical indices in
$T_1$ to physical addresses in tier~$\mu_2$.
Morphisms compose: given
$f : (T_1, \mu_1) \to (T_2, \mu_2)$ and
$g : (T_2, \mu_2) \to (T_3, \mu_3)$,
$g \circ f : (T_1, \mu_1) \to (T_3, \mu_3)$.

\smallskip\noindent
\textbf{Layered representation.}\;
$\Lambda = (\mathcal{L}^{\mathrm{Cat}},\,
\mathcal{L}^{\ISL{}},\,
\mathcal{L}^{\Ftwo})$
is a triple of representation layers:
\begin{enumerate}[label=(\roman*)]
  \item $\mathcal{L}^{\mathrm{Cat}}$:
    morphisms expressed as $\mathbf{Nest}$/$\mathbf{Tuple}$
    categorical maps~\cite{SEED_4}.
    Supports composition~($\circ$), logical
    product~($\otimes$), and logical
    division~($/$).
  \item $\mathcal{L}^{\ISL{}}$:
    morphisms expressed as \ISL{} quasi-affine integer
    set relations $R : \ZZ^m \to \ZZ^n$~\cite{SEED_2}.
    Subsumes $\mathcal{L}^{\mathrm{Cat}}$ for all
    representable layouts and extends to non-linear
    swizzle patterns via bit-level ISL
    operations~\cite{SEED_2}.
  \item $\mathcal{L}^{\Ftwo}$:
    morphisms expressed as binary matrices
    $L \in \Ftwo^{m \times n}$~\cite{SEED_1}.
    Restricted to \emph{linear} layouts; composition
    is matrix multiplication over~$\Ftwo$ with cost
    $O(mn)$.
\end{enumerate}
\end{definition}

\noindent
The three layers relate via embedding maps
$\phi$ and functor~$F$ (see \cref{sec:layout-sem}),
which are formal proof obligations for \BUTA{}
(gaps~G1, G2).


% ------ 4.3 Memory Tier Graph ------
\subsection{Memory Tier Graph $\mathcal{H}$}
\label{sec:model:tier}

\begin{definition}[Memory Tier
  Graph~\cite{ARCH_BW,NV_BLOG_TILE}]
\label{def:tier-graph}
The \emph{Memory Tier Graph} is a capacity-labelled
directed multigraph
$\mathcal{H} =
(N,\, E,\, \mathrm{cap},\, \mathrm{lat},\,
 \mathrm{bw},\, \mathrm{dtype})$
parameterised by compute capability
$x \in \{10.0, 12.0\}$:

\smallskip\noindent
\textbf{Nodes.}\;
$N = \{\textsc{RF},\, \TMEM{},\, \SMEM{},\,
\textsc{L1},\, \textsc{L2},\, \textsc{HBM}\}$.

\smallskip\noindent
\textbf{Edges.}\;
$E \subseteq N \times N \times
\{\texttt{tma},\, \tcgen{}\texttt{.ld},\,
\tcgen{}\texttt{.st},\, \tcgen{}\texttt{.cp},\,
\texttt{ld},\, \texttt{st},\, \texttt{mma\_acc}\}$.
Each edge carries a \emph{descriptor type}
$\mathrm{dtype}(e) \in
\{\texttt{idesc}, \texttt{sdesc}, \texttt{none}\}$
following the \tma{} API~\cite{NV_BLOG_TILE}.

\smallskip\noindent
\textbf{Label functions.}\;
\begin{align}
  \mathrm{cap} &: N \to \RR_{\geq 0}
    &&\text{(capacity in KB)} \notag\\
  \mathrm{lat} &: E \to \RR_{> 0}
    &&\text{(latency in cycles)} \notag\\
  \mathrm{bw}  &: E \to \RR_{> 0}
    &&\text{(bandwidth in TB/s)} \notag
\end{align}
Capacity is parameterised by~$x$:
$\mathrm{cap}(\SMEM{}, 10.0) = 228$\,KB,
$\mathrm{cap}(\SMEM{}, 12.0) = 128$\,KB,
$\mathrm{cap}(\TMEM{}) = 256$\,KB (both),
$\mathrm{cap}(\textsc{L2}) \approx 65$\,MB
(unified)~\cite{ARCH_BW}.
\end{definition}

\begin{figure}[t]
\centering
\begin{BVerbatim}[fontsize=\scriptsize]
+========================================================+
|   Memory Tier Graph  H   (B200 cc 10.0 / GB10 cc 12.0)|
+========================================================+
|                                                        |
|  +--------+ tcgen05.cp  +--------+                     |
|  |   RF   |<------------|  TMEM  |                     |
|  | 256 KB |             | 256 KB |                     |
|  +---+----+             +---+----+                     |
|      |                  ^   |                          |
|      |         mma_acc  |   | tcgen05.ld               |
|      |      (TC writes) |   v                          |
|      |            +-----+-------+                      |
|      | explicit   | Tensor Core |                      |
|      | ld/st      |  (tcgen05)  |                      |
|      |            +-----+-------+                      |
|      |                  ^                              |
|      |          A,B     | operands                     |
|      v                  |                              |
|  +--------+  TMA bulk   +--------+   TMA bulk         |
|  | L1/Tex |<------------|  SMEM  |<------+             |
|  |        |  coherent   |228/128 |       |             |
|  +---+----+             +--------+       |             |
|      |                                   |             |
|      v                                   |             |
|  +--------+                          +---+----+        |
|  |   L2   |                          |  HBM   |        |
|  | ~65 MB |<------- cache fill ------|192 GB  |        |
|  | unified|-------> eviction ------->|8 TB/s  |        |
|  +--------+                          +--------+        |
|                                                        |
|  Edge notation:                                        |
|    ----->  = data movement (labeled by operation type) |
|    TMEM is ASYMMETRIC:                                 |
|      TC writes accumulator directly to TMEM            |
|      tcgen05.cp required for TMEM -> RF transfer       |
|      tcgen05.ld/st for explicit TMEM access            |
+========================================================+
\end{BVerbatim}
\caption{Memory Tier Graph $\mathcal{H}$ for
  Blackwell, parameterised from~\cite{ARCH_BW}.
  \TMEM{} is asymmetric: the tensor core writes
  accumulators directly; explicit
  $\tcgen{}\texttt{.cp}$ transfers data to RF for
  epilogue.
  \tma{} provides bulk asynchronous movement via typed
  descriptors~\cite{NV_BLOG_TILE}.
  \SMEM{} capacity depends on compute capability:
  228\,KB (cc\,10.0) vs.\ 128\,KB
  (cc\,12.0)~\cite{ARCH_BW}.}
\label{fig:tier-graph}
\end{figure}

\begin{remark}
\TMEM{} read bandwidth of ${\approx}16$\,TB/s per SM
and write bandwidth of ${\approx}8$\,TB/s per SM are
attributed to secondary analysis \unverified{}; these
values enter $\mathrm{bw}$ labels pending independent
calibration (Part~3)~\cite{ARCH_BW}.
\end{remark}


% ------ 4.4 Schedule Constraint System ------
\subsection{Schedule Constraint System $\mathcal{S}$}
\label{sec:model:sched}

\begin{definition}[Schedule Constraint
  System~\cite{OPT_PIPE,ARCH_BW,NV_BLOG_TILE}]
\label{def:sched-sys}
The \emph{Schedule Constraint System} is a
parameterised constraint-optimisation problem
$\mathcal{S} =
(G,\, \mathbf{X},\, \Phi,\, f_{\mathrm{obj}},\,
 \mathbf{c})$
where:

\smallskip\noindent
\textbf{Dependence graph.}\;
$G = (V, E)$ with $V$ the set of tile-level operations
(MMA, \tma{} load/store, \tcgen{}\texttt{.cp}, ALU)
and $E$ directed dependency edges each carrying latency
$d(u,v)$ and distance $\delta(u,v)$~\cite{OPT_PIPE}.

\smallskip\noindent
\textbf{Variables.}\;
$\mathbf{X} =
\{\mathrm{op}[v,i,t],\; W(v),\; \II\}$
where $\mathrm{op}[v,i,t] \in \{0,1\}$ encodes
whether operation~$v$ from iteration~$i$ is scheduled
at time~$t$;
$W : V \to \{0,\dots,N_W{-}1\}$ assigns warps;
$\II \in \NN^+$ is the initiation
interval~\cite{OPT_PIPE}.

\smallskip\noindent
\textbf{Constraint set.}\;
$\Phi = \Phi_{\mathrm{dep}} \wedge
\Phi_{\mathrm{res}} \wedge
\Phi_{\mathrm{fu}} \wedge
\Phi_{\mathrm{2cta}}$ (see~\cref{sec:sched-sem}
for full specification).

\smallskip\noindent
\textbf{Objective.}\;
$f_{\mathrm{obj}} = \min\, \II$
subject to $\Phi$~\cite{OPT_PIPE}.

\smallskip\noindent
\textbf{Cost vector.}\;
$\mathbf{c} : V \times \mathcal{P} \times
\mathbf{TileShapes} \to \RR_{>0}$
maps each operation type, precision, and tile shape
to a latency cost, parameterised by Blackwell
microbenchmark data~\cite{ARCH_BW}.
Cost normalisation as a Zero-One Linear Program (ZLP)
renders ratios tractable for the
solver~\cite{OPT_PIPE}.
\end{definition}


% ------ 4.5 Data Movement Relation ------
\subsection{Data Movement Relation $\mathcal{D}$}
\label{sec:model:datamov}

\begin{definition}[Data Movement
  Relation~\cite{NV_BLOG_TILE,ARCH_BW}]
\label{def:datamov-rel}
The \emph{Data Movement Relation} is a set of typed
data-flow operations over $\mathcal{H}$:
\[
  \mathcal{D} \subseteq
  N \times N \times \mathrm{Desc} \times
  \mathrm{Token}^{*} \times \mathrm{Token}^{*}
\]
where each element
$\delta = (\mu_s,\, \mu_d,\, \mathrm{desc},\,
\bar{\tau}_{\mathrm{in}},\, \bar{\tau}_{\mathrm{out}})$
represents a data transfer from source tier~$\mu_s$ to
destination tier~$\mu_d$ via descriptor type
$\mathrm{desc} \in
\{\texttt{idesc}, \texttt{sdesc}, \texttt{none}\}$,
consuming input token sequence
$\bar{\tau}_{\mathrm{in}}$ and producing output token
sequence $\bar{\tau}_{\mathrm{out}}$~\cite{NV_BLOG_TILE}.

The induced \emph{token dependency DAG}
$\mathcal{T} = (\mathrm{Ops},\, \mathrm{TokenEdges})$
has an edge $(u, v)$ whenever some output token of~$u$
is consumed as an input token of~$v$.
The \emph{waits-for} relation is the transitive
closure of $\mathrm{TokenEdges}$~\cite{NV_BLOG_TILE}.
\end{definition}


% ------ 4.6 Cache Traffic Function ------
\subsection{Cache Traffic Function $\mathcal{C}$}
\label{sec:model:cache}

\begin{definition}[Cache Traffic
  Function~\cite{SEED_3,ARCH_BW}]
\label{def:cache-func}
The \emph{Cache Traffic Function} is a map
\[
  \mathcal{C} :
  \mathrm{Perm}(\mathrm{Tiles}) \times \NN \times
  \RR_{>0} \;\longrightarrow\; \NN
\]
defined by
$\mathcal{C}(\sigma, N_{\mathrm{SM}}, C_{\mathrm{L2}})
= \sum_{a \in \mathrm{sectors}}
\mathbf{1}[\reuse(a, \sigma) >
C_{\mathrm{L2}} / s]$,
where $\sigma$ is a tile iteration permutation,
$N_{\mathrm{SM}}$ is the active SM count,
$C_{\mathrm{L2}} \approx 65$\,MB is the L2 cache
capacity~\cite{ARCH_BW},
$s = 32$\,bytes is the sector size, and
$\reuse(a, \sigma)$ is the reuse distance of cache
line~$a$ under ordering~$\sigma$~\cite{SEED_3}.
For streaming tile patterns, L1 is
negligible~\cite{SEED_3}; L2 is the dominant cache
tier.
\end{definition}

\begin{remark}[\inference{}]
The cache traffic function carries a $(\min, +)$
tropical semiring structure:
the additive operation sums per-tile-block miss
counts (total misses decompose over independent
sub-programs), and the multiplicative operation
selects the ordering~$\sigma^{*}$ that minimises
total misses.
This semiring structure is inferred from the
composition of miss-count aggregation (additive,
\cite{SEED_3}) with reuse-distance minimisation
(order-dependent, \cite{SEED_3}); it is not
established in the golden sources.
\end{remark}


% ------ 4.7 Workload Descriptor ------
\subsection{Workload Descriptor $\mathcal{W}$}
\label{sec:model:workload}

\begin{definition}[Workload
  Descriptor~\cite{NV_workloads,ARCH_BW}]
\label{def:workload}
A \emph{Workload Descriptor} is a labelled set of
tile-level operators:
\[
  \mathcal{W} =
  \bigl\{(\mathrm{op}_i,\;
         \mathrm{AI}_i,\;
         r_i,\;
         p_i)
  : i \in [n]\bigr\}
\]
where $\mathrm{op}_i$ is the operation type
(MMA, TMA, ALU, etc.),
$\mathrm{AI}_i \in \RR_{>0}$ is the arithmetic
intensity (FLOP/byte),
$r_i \in \{\texttt{streaming},\, \texttt{reuse},\,
\texttt{mixed}\}$ is the reuse class,
and $p_i \in \mathcal{P}$ is the operating
precision.
The two-axis HARP taxonomy~\cite{NV_workloads}
classifies the overall workload along
\emph{depth of compute} (leaf-only vs.\ hierarchical)
and \emph{location of heterogeneity}
(homogeneous, intra-node, cross-node, cross-depth,
compound).
Per-operator arithmetic intensity is determined by
precision-dependent throughput
from~\cite{ARCH_BW}.
\end{definition}


% ------ 4.8 Precision Lattice ------
\subsection{Precision Lattice $\mathcal{P}$}
\label{sec:model:prec}

\begin{definition}[Precision
  Lattice~\cite{ARCH_BW,NV_BLOG_TILE}]
\label{def:prec-lattice-full}
The \emph{Precision Lattice} is a triple
$\mathcal{P} = (P,\, \leq_b,\, \sigma)$ where:

\smallskip\noindent
$P = \{\texttt{FP64},\, \texttt{TF32},\,
\texttt{BF16},\, \texttt{FP16},\, \texttt{FP8},\,
\texttt{FP6},\, \texttt{FP4},\,
\texttt{INT8}\}$~\cite{ARCH_BW}.

\smallskip\noindent
$\leq_b$ is a partial order by bit-width refinement:
$p_1 \leq_b p_2$ iff $\mathrm{bits}(p_1)
\leq \mathrm{bits}(p_2)$.

\smallskip\noindent
$\sigma : P \to
\{\textsc{DMMA},\, \textsc{HMMA},\,
\textsc{QMMA},\, \textsc{OMMA},\,
\textsc{IMMA}\}$
maps each precision to its SASS instruction
class~\cite{ARCH_BW}.
\TileIR{} types include precision-specific
encodings (\texttt{fp8e4m3fn},
\texttt{fp8e5m2}, \texttt{tf32},
etc.)~\cite{NV_BLOG_TILE}.

\smallskip\noindent
Throughput scales up to $177\times$ from FP64 to FP4;
FP32 accumulation halves throughput relative to FP16
accumulation~\cite{ARCH_BW}.
FP4 and FP6 are new to Blackwell with no Hopper
datapaths~\cite{ARCH_BW}.
\end{definition}


% ------ 4.9 Compositional Interface Constraints ------
\subsection{Compositional Interface Constraints}
\label{sec:model:compose}

The seven objects interact via \emph{interface
constraints} that ensure compositional consistency.
We enumerate the principal interfaces that govern how
optimisation in one layer constrains or parameterises
another.

\begin{definition}[Interface
  Constraint Set $\Psi$]
\label{def:interface}
The \emph{interface constraint set} is the conjunction:
\begin{align}
  \Psi \;=\;\;
    &\Psi_{\mathcal{P} \to \mathcal{L}}
    \;\wedge\;
    \Psi_{\mathcal{P} \to \mathcal{S}}
    \;\wedge\;
    \Psi_{\mathcal{W} \to \mathcal{S}}
    \;\wedge\;
    \Psi_{\mathcal{L} \to \mathcal{D}} \notag\\
    \wedge\;\;
    &\Psi_{\mathcal{H} \to \mathcal{D}}
    \;\wedge\;
    \Psi_{\mathcal{H} \to \mathcal{S}}
    \;\wedge\;
    \Psi_{\mathcal{H} \to \mathcal{C}}
    \;\wedge\;
    \Psi_{\mathcal{S} \to \mathcal{C}} \notag\\
    \wedge\;\;
    &\Psi_{\mathcal{D} \to \mathcal{S}}
    \label{eq:interface}
\end{align}
where each $\Psi_{A \to B}$ denotes a set of
constraints that the output of object~$A$ imposes on
object~$B$.
Specifically:
\begin{enumerate}[label=(\roman*)]
  \item $\Psi_{\mathcal{P} \to \mathcal{L}}$:
    precision~$p$ constrains valid tile shapes
    and layout bit-widths in
    $\mathcal{L}$~\cite{ARCH_BW, SEED_1}.
  \item $\Psi_{\mathcal{P} \to \mathcal{S}}$:
    precision determines \tcgen{}\texttt{.mma}
    instruction class and throughput in
    $\mathbf{c}$~\cite{ARCH_BW}.
  \item $\Psi_{\mathcal{W} \to \mathcal{S}}$:
    per-operator AI and reuse class condition the
    cost vector for the
    solver~\cite{NV_workloads, OPT_PIPE}.
  \item $\Psi_{\mathcal{L} \to \mathcal{D}}$:
    layout morphisms determine addressing for
    each \tma{} descriptor
    operation~\cite{SEED_2, NV_BLOG_TILE}.
  \item $\Psi_{\mathcal{H} \to \mathcal{D}}$:
    $\mathcal{D}$ operates over edges
    of~$\mathcal{H}$;
    edge bandwidth/latency labels constrain
    achievable throughput~\cite{ARCH_BW}.
  \item $\Psi_{\mathcal{H} \to \mathcal{S}}$:
    memory tier capacities impose resource
    bounds in $\Phi_{\mathrm{res}}$
    (e.g., live tiles in \SMEM{}
    $\leq C_{\SMEM{}}(x)$)~\cite{ARCH_BW}.
  \item $\Psi_{\mathcal{H} \to \mathcal{C}}$:
    $C_{\mathrm{L2}}$ parameterises the miss
    prediction threshold~\cite{ARCH_BW, SEED_3}.
  \item $\Psi_{\mathcal{S} \to \mathcal{C}}$:
    the schedule determines tile iteration
    ordering~$\sigma$, which is the primary input
    to $\mathcal{C}$~\cite{SEED_3}.
  \item $\Psi_{\mathcal{D} \to \mathcal{S}}$:
    token-ordered data movement latencies feed
    back as dependency edge weights in~$G$%
    ~\cite{OPT_PIPE, NV_BLOG_TILE}.
\end{enumerate}
\end{definition}


% ====================================================================
%  SECTION 5 — LAYOUT SEMANTICS LAYER
% ====================================================================
\section{Layout Semantics Layer}
\label{sec:layout-sem}

This section formalises the three-level representation
hierarchy within $\mathcal{L}$, defines the embedding
maps between levels, and states the layout invariants
that \BUTA{} must preserve.

\subsection{Three-Level Hierarchy and Embedding Maps}
\label{sec:layout-sem:hierarchy}

The three representation layers of $\mathcal{L}$ are
ordered by generality (ascending) and computational
efficiency (descending).
We define the embedding maps that connect them.

\begin{definition}[$\Ftwo$-to-\ISL{}
  Embedding $\phi$~\inference{}]
\label{def:phi-embed}
The map
$\phi : \Ftwo^{m \times n} \to
\mathrm{ISL\text{-}Rel}(\ZZ^n, \ZZ^m)$
is defined by
\[
  \phi(L) \;=\;
  \bigl\{
    [b_0, \dots, b_{n-1}] \to
    \bigl[\textstyle\sum_{j} L_{0j}\, b_j
      \!\!\mod 2,\;
    \dots,\;
    \textstyle\sum_{j} L_{(m-1)j}\, b_j
      \!\!\mod 2
    \bigr]
  \bigr\}
\]
with domain restricted to
$b_j \in \{0, 1\}$~\cite{SEED_1, SEED_2}.
\end{definition}

\begin{obligation}[$\phi$ Correctness (Gap~G1)]
\label{po:phi}
Prove:
(a)~$\phi$ is \emph{injective}:
distinct $\Ftwo$ matrices yield distinct \ISL{}
relations on the binary domain;
(b)~$\phi$ \emph{preserves composition}:
$\phi(L_2 \cdot L_1) =
\phi(L_2) \circ_{\ISL{}} \phi(L_1)$
for all composable pairs~\cite{SEED_1, SEED_2}.
Planned technique: constructive proof via \ISL{}
relation composition semantics; automated verification
on small layout instances using \ISL{} library.
\end{obligation}

\begin{definition}[Categorical-to-\ISL{}
  Functor $F$~\inference{}]
\label{def:F-functor}
The functor $F : \mathbf{Nest} \to
\mathrm{ISL\text{-}Rel}$ is defined:

\smallskip\noindent
\emph{On objects:}\;
$F\bigl((n_1,\dots,n_k)\bigr) =
\{[x_1,\dots,x_k] : 0 \leq x_i < n_i\}$.

\smallskip\noindent
\emph{On morphisms:}\;
each $\mathbf{Nest}$ morphism~$f$ defining a
coordinate transformation is mapped to its \ISL{}
relation representation via Algorithm~1
of~\cite{SEED_2}, which converts
$(\mathrm{shape}, \mathrm{stride})$ pairs to
quasi-affine \ISL{} relations.
\end{definition}

\begin{obligation}[$F$ Compatibility (Gap~G2)]
\label{po:F}
Prove:
(a)~$F$ \emph{preserves composition}:
$F(g \circ f) =
F(g) \circ_{\ISL{}} F(f)$;
(b)~$F$ preserves \emph{logical product}:
$F(f \otimes g) =
F(f) \times_{\ISL{}} F(g)$;
(c)~$F$ preserves \emph{logical division}:
$F(f / g) = F(f) /_{\ISL{}} F(g)$;
(d)~the image of~$F$ \emph{coincides} with the set of
tractable layouts characterised
in~\cite{SEED_4}~\cite{SEED_2, SEED_4}.
Planned technique: category-theoretic proof with
Python implementation extending the SEED\_4
codebase~\cite{SEED_4} to emit \ISL{} relations;
cross-validation with Algorithm~1
of~\cite{SEED_2}.
\end{obligation}

\begin{figure}[t]
\centering
\begin{BVerbatim}[fontsize=\scriptsize]
+===========================================================+
|  Layout Semantics: Commutative Diagram of Embeddings      |
+===========================================================+
|                                                           |
|                     Nest                                  |
|                    /    \                                  |
|             morph /      \ morph                          |
|                  v        v                               |
|       (T1,mu1) ----f----> (T2,mu2)                       |
|                                                           |
|       Functor F [PO-2, gap G2]                            |
|          |                                                |
|          v                                                |
|                    ISL-Rel                                 |
|                   /       \                               |
|          R_f     /         \  R_g                          |
|                 v           v                             |
|       ISL(T1) ----R_f----> ISL(T2)                        |
|                                                           |
|       Embedding phi [PO-1, gap G1]                        |
|          ^                                                |
|          |                                                |
|                     F_2                                    |
|                   /      \                                |
|             L1   /        \  L2                            |
|                 v          v                              |
|       F2^n ----L2*L1^-1--> F2^m                           |
|                                                           |
|  Required: F and phi make the diagram COMMUTE             |
|    phi(L2 * L1) = phi(L2) o phi(L1)      [PO-1(b)]       |
|    F(g o f)     = F(g) o F(f)            [PO-2(a)]       |
|    phi(L) in Image(F) for linear L       [PO-2(d)]       |
+===========================================================+
\end{BVerbatim}
\caption{Commutative diagram for layout semantics.
  The categorical layer
  ($\mathbf{Nest}$~\cite{SEED_4}) maps via functor~$F$
  to the \ISL{} analytical
  layer~\cite{SEED_2}; the $\Ftwo$ efficient
  layer~\cite{SEED_1} embeds via~$\phi$.
  Commutativity is a proof obligation for
  \BUTA{}.}
\label{fig:layout-commute}
\end{figure}

\subsection{Layout Invariants}
\label{sec:layout-sem:inv}

\begin{invariant}[INV-L1: Bijectivity]
\label{inv:L1}
Every layout morphism
$f \in \mathrm{Mor}_{\mathcal{L}}$ is
\emph{injective} on its defined domain:
for all $\mathbf{c}_1 \neq \mathbf{c}_2 \in
\mathrm{dom}(f)$,
$f(\mathbf{c}_1) \neq f(\mathbf{c}_2)$.
This ensures that distinct logical coordinates map to
distinct physical addresses, preventing data
corruption~\cite{SEED_1, SEED_4}.
\end{invariant}

\begin{invariant}[INV-L2: Composition Correctness]
\label{inv:L2}
For any two representation levels
$\ell_1, \ell_2 \in
\{\mathrm{Cat}, \ISL{}, \Ftwo\}$
and composable morphisms $f, g$ representable at both
levels, the composition at~$\ell_1$ agrees with the
composition at~$\ell_2$ after applying the appropriate
embedding.
Concretely, for $\Ftwo$ and \ISL{}:
\[
  \phi(L_g \cdot L_f) \;=\;
  \phi(L_g) \circ_{\ISL{}} \phi(L_f)
\]
and for $\mathbf{Nest}$ and \ISL{}:
\[
  F(g \circ f) \;=\;
  F(g) \circ_{\ISL{}} F(f)
\]
where $L_f, L_g$ are the $\Ftwo$ matrices of $f, g$
respectively~\cite{SEED_1, SEED_2, SEED_4}.
\end{invariant}

\begin{invariant}[INV-L3: Swizzle Well-Formedness]
\label{inv:L3}
Every bit-level swizzle operation modelled as an \ISL{}
integer set operation~\cite{SEED_2} produces a
\emph{valid permutation} on memory offsets within the
target tier.
That is, for swizzle relation
$S_w : \ZZ^n \to \ZZ^n$,
$S_w$ is bijective on
$\{0, \dots, |\mathrm{tier}| - 1\}$.
Non-linear swizzles lie outside $\mathcal{L}^{\Ftwo}$
and require $\mathcal{L}^{\ISL{}}$
representation~\cite{SEED_1, SEED_2}.
\end{invariant}

\begin{invariant}[INV-L4: Linear Fragment Closure]
\label{inv:L4}
The sub-category $\mathcal{L}^{\Ftwo}$ is
\emph{closed} under composition and layout-to-layout
conversion:
if $L_1, L_2 \in \Ftwo^{m \times n}$ are linear
layouts, then $L_2 \cdot L_1^{-1}$
(or pseudoinverse) is again a linear
layout~\cite{SEED_1}.
This guarantees that the efficient $\Ftwo$
computational path remains available throughout a
chain of linear layout transformations.
\end{invariant}


% ====================================================================
%  SECTION 6 — DATA MOVEMENT SEMANTICS LAYER
% ====================================================================
\section{Data Movement Semantics Layer}
\label{sec:datamov-sem}

The data movement layer governs how tile data flows
between memory tiers in $\mathcal{H}$, with correctness
enforced by the \TileIR{} token ordering
model~\cite{NV_BLOG_TILE} and capacity constraints
from Blackwell hardware parameters~\cite{ARCH_BW}.

\subsection{\tma{} Descriptor Relations over
            $\mathcal{H}$}
\label{sec:datamov-sem:tma}

\begin{definition}[\tma{} Operation Typing]
\label{def:tma-typing}
Each \tma{} bulk data-movement operation in
$\mathcal{D}$ is typed by a tuple:
\[
  \delta_{\tma{}} =
  (\mu_s,\, \mu_d,\, \mathrm{desc},\,
   \bar{\tau}_{\mathrm{in}},\,
   \bar{\tau}_{\mathrm{out}},\,
   \ell)
\]
where $(\mu_s, \mu_d) \in E(\mathcal{H})$ is an edge
of the tier graph,
$\mathrm{desc} \in \{\texttt{idesc}, \texttt{sdesc}\}$
is the descriptor type~\cite{NV_BLOG_TILE},
$\bar{\tau}_{\mathrm{in}}, \bar{\tau}_{\mathrm{out}}$
are token sequences, and $\ell \in
\mathrm{Mor}_{\mathcal{L}}$ is the layout morphism
governing address translation for the transfer.
\tma{} descriptors are \emph{strictly preferred} over
tensor-of-pointer access patterns, which exhibit poor
performance on
Blackwell~\cite{NV_BLOG_TILE}.
\end{definition}

\begin{definition}[\TMEM{} Operations]
\label{def:tmem-ops}
\TMEM{}-specific data movement uses three
\tcgen{}-family operations~\cite{ARCH_BW}:
\begin{enumerate}[label=(\alph*)]
  \item $\tcgen{}\texttt{.ld}$:
    explicit read from \TMEM{} to register file.
  \item $\tcgen{}\texttt{.st}$:
    explicit write from register file to \TMEM{}.
  \item $\tcgen{}\texttt{.cp}$:
    copy from \TMEM{} to RF, required for
    epilogue after MMA writes accumulators directly
    to \TMEM{}.
\end{enumerate}
In addition, the tensor core writes accumulators
directly to \TMEM{} via
$\texttt{mma\_acc}$ edges, with no explicit
instruction---the \tcgen{}\texttt{.mma} output
operand~$D$ resides in \TMEM{}~\cite{ARCH_BW}.
\TMEM{} is organised as
$512 \times 128 \times 32$\,bits
(columns $\times$ lanes $\times$ bit-width)
with reported hit rates of 61--82\% in multi-stage
tensor pipelines~\cite{ARCH_BW}.
\end{definition}


\subsection{\TMEM{} Lifecycle State Machine}
\label{sec:datamov-sem:lifecycle}

\begin{definition}[\TMEM{} Lifecycle~\cite{ARCH_BW}]
\label{def:tmem-lifecycle}
The \TMEM{} lifecycle for a given allocation region is
modelled as a finite-state machine
$\mathcal{F}_{\TMEM{}} =
(Q,\, q_0,\, \Sigma,\, \delta_F)$
with states, initial state, input alphabet, and
transition function:
\begin{align}
  Q &= \{\textsc{Free},\, \textsc{Alloc},\,
         \textsc{Compute},\,
         \textsc{Ready},\,
         \textsc{Consumed}\} \notag\\
  q_0 &= \textsc{Free} \notag\\
  \Sigma &= \{\texttt{alloc},\, \texttt{mma},\,
              \texttt{mma\_done},\, \texttt{cp},\,
              \texttt{dealloc},\,
              \texttt{relinquish}\} \notag
\end{align}
\end{definition}

\begin{figure}[t]
\centering
\begin{BVerbatim}[fontsize=\scriptsize]
+===========================================================+
|  TMEM Lifecycle State Machine  F_TMEM                     |
+===========================================================+
|                                                           |
|         tcgen05.alloc                                     |
|  +------+------------>+---------+                         |
|  | FREE |             |ALLOCATED|                         |
|  +--^---+             +----+----+                         |
|     |                      |                              |
|     |             tcgen05.mma (TC writes D -> TMEM)       |
|     |                      |                              |
|     |                      v                              |
|     |                 +---------+                         |
|     |  tcgen05.       |COMPUTING|  MMA in progress;       |
|     |  dealloc        |         |  TMEM region locked     |
|     |                 +----+----+                         |
|     |                      |                              |
|     |              mma_done (accumulator written)         |
|     |                      |                              |
|     |                      v                              |
|     |                 +---------+                         |
|     +<----------------|  READY  |  Accumulator readable;  |
|     |  dealloc        |         |  tcgen05.cp available   |
|     |                 +----+----+                         |
|     |                      |                              |
|     |              tcgen05.cp (TMEM -> RF)                |
|     |                      |                              |
|     |                      v                              |
|     |                 +---------+                         |
|     +<----------------| CONSUMED|  Data in RF;            |
|        dealloc /      |         |  TMEM may be freed      |
|        relinquish     +---------+                         |
|                                                           |
|  Constraint: sum of live TMEM columns <= 512 per SM       |
|  MMA.2SM: CTA pair shares TMEM region across TPC          |
+===========================================================+
\end{BVerbatim}
\caption{\TMEM{} lifecycle state machine.
  Transitions are triggered by \tcgen{}-family
  operations~\cite{ARCH_BW}.
  The \textsc{Computing} state locks the \TMEM{} region
  until the MMA completes.
  The \texttt{relinquish\_alloc\_permit} operation
  releases the allocation lock for CTA
  queuing~\cite{ARCH_BW}.
  \TMEM{} allocation details beyond this model are
  partially documented \unverified{} (gap~G3).}
\label{fig:tmem-lifecycle}
\end{figure}


\subsection{Token Ordering and Correctness}
\label{sec:datamov-sem:tokens}

\begin{definition}[Token Partial
  Order~\cite{NV_BLOG_TILE}]
\label{def:token-po}
The \TileIR{} memory model partitions operations into
\emph{program-ordered} (sequenced by textual position)
and \emph{token-ordered} (sequenced by token
dependencies)~\cite{NV_BLOG_TILE}.
Define the \emph{token partial order}
$\preceq_{\mathcal{T}}$ on the operation set
$\mathrm{Ops}$ as the reflexive transitive closure of
$\mathrm{TokenEdges}$ in $\mathcal{T}$:
\[
  u \preceq_{\mathcal{T}} v
  \;\iff\;
  \exists\, \text{path } u \to \cdots \to v
  \text{ in } \mathcal{T}\,.
\]
The \emph{happens-before} relation
$u \xrightarrow{\mathrm{hb}} v$ is established by
release/acquire pairs on the same memory
location~\cite{NV_BLOG_TILE}.
\end{definition}

\begin{definition}[Data-Race
  Freedom~\cite{NV_BLOG_TILE}]
\label{def:drf}
A tile block is \emph{data-race free} iff for every
pair of operations $(a, b)$ such that
$a$ and $b$ access the same memory location and at
least one writes:
\[
  a \preceq_{\mathcal{T}} b
  \;\;\lor\;\;
  b \preceq_{\mathcal{T}} a\,.
\]
Data races are possible within a single tile block if
token ordering is
absent~\cite{NV_BLOG_TILE}.
\end{definition}


\subsection{Data Movement Invariants}
\label{sec:datamov-sem:inv}

\begin{invariant}[INV-D1: Data Reachability]
\label{inv:D1}
Every tile datum consumed by a compute operation has a
\emph{path} from \textsc{HBM} to the compute unit
through $\mathcal{H}$:
for every operand~$o$ of every $v \in V$,
there exists a sequence of edges
$e_1, \dots, e_k \in E(\mathcal{H})$
such that
$\mu_s(e_1) = \textsc{HBM}$ and
$\mu_d(e_k) = \mathrm{tier}(v.\mathrm{operand})$%
~\cite{ARCH_BW, NV_BLOG_TILE}.
For tensor-core MMA: operands~$A$ from \SMEM{}
(or \TMEM{} in TS mode); operands~$B$ from
\SMEM{}; accumulator~$D$ to
\TMEM{}~\cite{ARCH_BW}.
\end{invariant}

\begin{invariant}[INV-D2: Token DAG Acyclicity]
\label{inv:D2}
The token dependency graph
$\mathcal{T} = (\mathrm{Ops}, \mathrm{TokenEdges})$
is a \emph{directed acyclic graph} within each tile
block.
This ensures the token partial order
$\preceq_{\mathcal{T}}$ is well-defined and prevents
circular waits~\cite{NV_BLOG_TILE}.
\end{invariant}

\begin{invariant}[INV-D3: Tier Capacity]
\label{inv:D3}
At every point in the execution, the sum of
\emph{live} tile allocations at each memory tier does
not exceed the tier capacity:
\begin{align}
  \forall\, t,\;\forall\, \mu \in N(\mathcal{H}):\quad
  &\sum_{\substack{a \in \mathrm{allocs}(t) \\
  \mathrm{tier}(a) = \mu}}
  \mathrm{size}(a)
  \;\leq\; \mathrm{cap}(\mu, x) \notag
\end{align}
In particular:
\TMEM{} live columns $\leq 512$
(i.e., $\leq 256$\,KB)~\cite{ARCH_BW};
\SMEM{} live tiles $\leq C_{\SMEM{}}(x)$
($228$\,KB for cc\,10.0,
$128$\,KB for cc\,12.0)~\cite{ARCH_BW}.
\end{invariant}

\begin{invariant}[INV-D4: \TMEM{} Lifecycle Ordering]
\label{inv:D4}
Every \TMEM{} allocation follows the state machine
$\mathcal{F}_{\TMEM{}}$
(\cref{def:tmem-lifecycle}):
allocation precedes use, deallocation follows last
read, and
$\texttt{relinquish\_alloc\_permit}$ releases the
allocation lock before kernel exit or CTA
re-queuing~\cite{ARCH_BW}.
No \TMEM{} region may be in state \textsc{Computing}
when a $\tcgen{}\texttt{.cp}$ or
$\tcgen{}\texttt{.ld}$ targets it.
\end{invariant}


% ====================================================================
%  SECTION 7 — SCHEDULING SEMANTICS LAYER
% ====================================================================
\section{Scheduling Semantics Layer}
\label{sec:sched-sem}

The scheduling layer instantiates the constraint
system $\mathcal{S}$ with Blackwell-specific
parameters and extends the
OPT\_PIPE~\cite{OPT_PIPE} formulation to handle
2CTA MMA mode~\cite{NV_BLOG_TILE}.

\subsection{Blackwell-Parameterised ILP+SMT
            Formulation}
\label{sec:sched-sem:ilp}

The ILP determines a lower bound on the initiation
interval $\II$, and the SMT encoding holistically
captures the joint SWP+WS
problem~\cite{OPT_PIPE}.
The Blackwell parameterisation instantiates the
abstract cost vector $\mathbf{c}$ of
\cref{def:sched-sys} with measured hardware
latencies.

\begin{definition}[Blackwell Cost Vector]
\label{def:cost-vector}
The cost vector for a Blackwell SM of compute
capability~$x$ is:
\begin{align}
  \mathbf{c}(x) &= \bigl(
    c_{\mathrm{MMA}}(p, T),\;
    c_{\mathrm{TMA}}^{\mathrm{ld}}(T),\;
    c_{\mathrm{TMA}}^{\mathrm{st}}(T),\; \notag\\
    &\qquad c_{\mathrm{ALU}},\;
    c_{\TMEM{}}^{\mathrm{cp}}
  \bigr) \label{eq:cost-vector}
\end{align}
where $p \in \mathcal{P}$ is precision,
$T$ is tile shape, and each component is a latency in
cycles.
Known anchors from~\cite{ARCH_BW}:
\begin{enumerate}[label=(\alph*)]
  \item $c_{\mathrm{MMA}}(p, T) \approx 11.0$--$11.4$
    cycles, approximately constant across tile sizes and
    precisions for FP16 inputs~\cite{ARCH_BW}.
  \item The SM has 4~warp schedulers, supporting up to
    $N_{\mathrm{WS}}(10.0) = 64$ and
    $N_{\mathrm{WS}}(12.0) = 48$ concurrent
    warps~\cite{ARCH_BW}.
  \item Execution contexts: $\leq 4$ per SM, bounding
    the maximum number of in-flight software pipeline
    stages~\cite{ARCH_BW, OPT_PIPE}.
\end{enumerate}
ZLP normalisation (per \cite{OPT_PIPE}) converts
absolute cycles to minimal positive integer ratios for
solver tractability.
\end{definition}

\begin{remark}
A complete cost table covering all precisions
($\mathcal{P}$), tile sizes, and operation types
is not available from the golden sources.
\tcgen{}\texttt{.mma} latency constancy
across precisions beyond FP16 is \unverified{} for
FP4, FP6, and FP64 (gap~G9).
TMA bulk load/store latencies for specific tile sizes
are not published and require independent
calibration (Part~3).
\end{remark}


\subsection{Constraint Decomposition}
\label{sec:sched-sem:constraints}

The full constraint set
$\Phi = \Phi_{\mathrm{dep}} \wedge
\Phi_{\mathrm{res}} \wedge
\Phi_{\mathrm{fu}} \wedge
\Phi_{\mathrm{2cta}}$
is specified below.

\begin{definition}[Dependency Constraints
  $\Phi_{\mathrm{dep}}$~\cite{OPT_PIPE}]
\label{def:phi-dep}
For every dependency edge $(u, v, d, \delta) \in E$:
\begin{equation}
  M(v) - M(u) + \II \cdot \delta
  \;\geq\; d(u,v)
  \label{eq:dep-constraint-p2}
\end{equation}
where $d(u,v)$ is the latency of edge $(u,v)$ drawn
from $\mathbf{c}(x)$ and $\delta(u,v)$ is the
iteration distance~\cite{OPT_PIPE}.
In the 3-D boolean encoding, this becomes:
for all $v, i, t$ with $\mathrm{op}[v,i,t] = 1$ and
predecessor $u$ at time $t'$:
$t - t' + \II \cdot \delta \geq d$~\cite{OPT_PIPE}.
\end{definition}

\begin{definition}[Resource Constraints
  $\Phi_{\mathrm{res}}$~\cite{ARCH_BW,OPT_PIPE}]
\label{def:phi-res}
At every time step~$t$, the number of active execution
contexts does not exceed the SM limit:
\begin{equation}
  \sum_{v \in V}
  \mathbf{1}\bigl[\exists\, i :
    \mathrm{op}[v,i,t] = 1\bigr]
  \;\leq\; 4
  \label{eq:resource}
\end{equation}
Additionally, memory tier capacities from
$\mathcal{H}$ apply:
live \SMEM{} allocations $\leq C_{\SMEM{}}(x)$;
live \TMEM{} allocations $\leq 256$\,KB.
The concurrent warp count is bounded by
$N_{\mathrm{WS}}(x)$~\cite{ARCH_BW}.
\end{definition}

\begin{definition}[Functional Unit Exclusivity
  $\Phi_{\mathrm{fu}}$~\cite{OPT_PIPE}]
\label{def:phi-fu}
At most one operation per functional unit type per
cycle:
for each unit class
$u_c \in \{\text{TC},\, \tma{},\, \text{ALU}\}$
and time step $t$:
\begin{equation}
  \sum_{\substack{v \in V \\ \mathrm{unit}(v) = u_c}}
  \sum_{i}
  \mathrm{op}[v,i,t]
  \;\leq\; 1
  \label{eq:fu-excl}
\end{equation}
where $\mathrm{unit}(v)$ maps each operation to its
functional unit class~\cite{OPT_PIPE}.
\end{definition}


\subsection{2CTA MMA Constraint}
\label{sec:sched-sem:2cta}

\begin{definition}[2CTA Constraint
  $\Phi_{\mathrm{2cta}}$~\cite{NV_BLOG_TILE,ARCH_BW}]
\label{def:phi-2cta}
When $\texttt{num\_ctas} = 2$, a CTA pair
$(\mathrm{CTA}_0, \mathrm{CTA}_1)$ mapped to the same
TPC (two adjacent SMs) shares input operands from the
aggregate \SMEM{} of both SMs~\cite{NV_BLOG_TILE}.
This introduces a \emph{CTA-pair assignment variable}:
\[
  \mathrm{cta}(v) \in \{0, 1\}
  \quad \forall\, v \in V
\]
and the following constraints:
\begin{enumerate}[label=(\alph*)]
  \item \emph{MMA pairing}: for every MMA operation
    $v$, the operand-$A$ load and operand-$B$ load
    must jointly span both CTAs:
    $\mathrm{cta}(v.A_{\mathrm{load}}) \cup
    \mathrm{cta}(v.B_{\mathrm{load}})
    = \{0, 1\}$~\cite{NV_BLOG_TILE}.
  \item \emph{\SMEM{} sharing}: total \SMEM{} per TPC
    $= C_{\SMEM{}}^{(0)} + C_{\SMEM{}}^{(1)}$,
    with shared operand region counted once.
    The MMA.2SM instruction leverages TPC-internal
    bandwidth~\cite{ARCH_BW}.
  \item \emph{Symmetry}: $\mathrm{CTA}_0$ and
    $\mathrm{CTA}_1$ are interchangeable;
    the solver may break symmetry by fixing
    $\mathrm{cta}(v_0) = 0$ for one
    operation~$v_0$.
\end{enumerate}
\end{definition}

\begin{remark}
2CTA mode is critical for dense dot
workloads~\cite{NV_BLOG_TILE}.
OPT\_PIPE~\cite{OPT_PIPE} does not explicitly model
inter-CTA coupling; extending the SMT formulation with
$\Phi_{\mathrm{2cta}}$ is a formal contribution of
\BUTA{} (gap~G6).
Inter-CTA synchronisation latency within a TPC is not
well characterised and is \unverified{} pending
microbenchmark calibration.
\end{remark}

\begin{figure}[t]
\centering
\begin{BVerbatim}[fontsize=\scriptsize]
+===========================================================+
|  Schedule Constraint Architecture for Blackwell           |
+===========================================================+
|                                                           |
|  Dependence Graph G = (V, E)                              |
|  +------+  d,delta  +------+  d,delta  +------+          |
|  | TMA  |---------->| MMA  |---------->| TMEM |          |
|  | load |           | S,O  |           |  cp  |          |
|  +------+           +------+           +------+          |
|      |                  ^                  |              |
|      |     tokens       |                  |              |
|      +---> [NV_BLOG_TILE] <---------+-----+              |
|                                                           |
|  Solver Pipeline:                                         |
|  +----------+   II_lb   +----------+   (M,W,II)          |
|  |   ILP    |---------->|   SMT    |----------> Schedule  |
|  | min II   |           | SWP+WS  |                      |
|  |          |           | op[v,i,t]|                      |
|  +----+-----+           +----+-----+                      |
|       |                      |                            |
|       v                      v                            |
|  Cost vector c(x)      Constraint set Phi                 |
|  from ARCH_BW:          = Phi_dep [OPT_PIPE]              |
|  - MMA: 11.0-11.4 cyc   AND Phi_res [ARCH_BW]            |
|  - WS: 64/48 warps      AND Phi_fu  [OPT_PIPE]           |
|  - Exec ctx: <=4        AND Phi_2cta [NV_BLOG_TILE]      |
+===========================================================+
\end{BVerbatim}
\caption{Schedule constraint architecture.
  The ILP lower-bounds $\II$; the SMT encodes joint
  SWP+WS~\cite{OPT_PIPE} with Blackwell cost
  vector~\cite{ARCH_BW} and 2CTA
  constraint~\cite{NV_BLOG_TILE}.
  Token ordering from $\mathcal{D}$ injects additional
  dependency edges into~$G$ (gap~G7).}
\label{fig:sched-arch}
\end{figure}


\subsection{Scheduling Invariants}
\label{sec:sched-sem:inv}

\begin{invariant}[INV-S1: Resource Feasibility]
\label{inv:S1}
Active execution contexts
$\leq 4$ per SM at all times~\cite{ARCH_BW, OPT_PIPE}.
Concurrent warps
$\leq N_{\mathrm{WS}}(x)$~\cite{ARCH_BW}.
Memory tier capacities respected per
INV-D3~(\cref{inv:D3}).
\end{invariant}

\begin{invariant}[INV-S2: No Data Hazards]
\label{inv:S2}
For every dependency $(u, v, d, \delta) \in E$:
$M(v) - M(u) + \II \cdot \delta \geq d(u,v)$%
~\cite{OPT_PIPE}.
This ensures that no operation reads data before its
producer has completed.
\end{invariant}

\begin{invariant}[INV-S3: Functional Unit Exclusivity]
\label{inv:S3}
At most one operation per TC, per \tma{} unit, and per
ALU cluster per cycle~\cite{OPT_PIPE}.
This models the physical constraint that Blackwell SMs
have one 5th-gen tensor core per
SM~\cite{ARCH_BW}.
\end{invariant}

\begin{invariant}[INV-S4: 2CTA Pairing]
\label{inv:S4}
When $\texttt{num\_ctas} = 2$: both CTAs of the TPC
pair are co-scheduled; shared \SMEM{} operand
regions are allocated consistently; MMA.2SM operations
reference operands from both SMs' \SMEM{}
stores~\cite{NV_BLOG_TILE, ARCH_BW}.
\end{invariant}

\begin{invariant}[INV-S5: Liveness]
\label{inv:S5}
All iterations of the software-pipelined loop
eventually complete; no deadlock arises in the
producer-consumer pipeline.
The modulo schedule $(M, \II)$ with period $\II$
guarantees progress: iteration $i+1$ initiates
exactly $\II$ cycles after
iteration~$i$~\cite{OPT_PIPE}.
\end{invariant}

\begin{remark}
\TileIR{} (CUDA~13.1) does not expose
$\texttt{num\_warps}$ as a tuning parameter;
only an occupancy hint (integer 1--32)
is available~\cite{NV_BLOG_TILE}.
This creates a tension with $W(v)$ in $\mathcal{S}$,
which assigns specific warps to
operations~\cite{OPT_PIPE}.
\BUTA{} treats $W(v)$ as a \emph{logical}
assignment: the compiler maps logical warp indices to
physical warps subject to the occupancy
hint \inference{}.
\end{remark}


% ====================================================================
%  SECTION 8 — CACHE TRAFFIC SEMANTICS LAYER
% ====================================================================
\section{Cache Traffic Semantics Layer}
\label{sec:cache-sem}

The cache traffic layer predicts L2 sector-level misses
as a function of tile iteration order, building on the
reuse-distance formalism of Zhu, Pan, and
Ding~\cite{SEED_3} and hardware parameters
from~\cite{ARCH_BW}.

\subsection{Parameterised Reuse-Distance Model}
\label{sec:cache-sem:model}

\begin{definition}[Parameterised Cache Traffic Model]
\label{def:cache-model-param}
For a tile program with iteration space $\mathcal{I}$,
tile ordering permutation
$\sigma : \mathcal{I} \to \mathcal{I}$,
active SM count $N_{\mathrm{SM}}$,
and L2 capacity $C_{\mathrm{L2}}$,
the predicted L2 sector miss count is:
\begin{equation}
  \mathcal{C}(\sigma, N_{\mathrm{SM}}, C_{\mathrm{L2}})
  = \sum_{a \in \mathrm{sectors}}
    \mathbf{1}\!\Bigl[
      \reuse(a, \sigma) >
      \frac{C_{\mathrm{L2}}}{s}
    \Bigr]
  \label{eq:miss-model}
\end{equation}
where $s = 32$\,bytes is the sector size and
$\reuse(a, \sigma)$ is the \emph{reuse distance}:
the count of distinct cache lines accessed between
consecutive accesses to line~$a$ under
ordering~$\sigma$~\cite{SEED_3}.
For Blackwell:
$C_{\mathrm{L2}} \approx 65$\,MB (unified across
SMs)~\cite{ARCH_BW}.
\end{definition}

\begin{definition}[Sawtooth Ordering Bound
  ~\cite{SEED_3}]
\label{def:sawtooth-bound}
For a streaming split-$Q$ dataflow with persistent
CTAs and $N_{\mathrm{CTA}}$ active CTAs, the
\emph{sawtooth order} (alternating inner-loop scan
direction) guarantees:
\[
  \reuse(a, \sigma_{\mathrm{saw}}) < D_{\mathrm{data}}
  \;\;\text{for fraction}\;\;
  \geq \bigl(1 - 1/N_{\mathrm{CTA}}\bigr)
\]
of sector accesses, where $D_{\mathrm{data}}$ is the
total data footprint~\cite{SEED_3}.
This achieves a 67\% L2 miss reduction
(370M\,$\to$\,120M sectors) and up to 60\% throughput
increase (41\,$\to$\,66~TFLOPS causal) on
GB10~\cite{SEED_3}.
\end{definition}

\begin{remark}
The reuse-distance model has been validated
\emph{only} for Flash Attention on GB10
(cc\,12.0, 128\,KB \SMEM{})~\cite{SEED_3}.
Generalisation to GEMM, convolution, or other
workloads on B200 (228\,KB \SMEM{},
cc\,10.0) is an \inference{} claim (gap~G5).
Furthermore, L1 is negligible for streaming attention
patterns~\cite{SEED_3}; non-streaming patterns where
L1 matters require a separate model extension.
\end{remark}


\subsection{Tropical Semiring Structure}
\label{sec:cache-sem:semiring}

\begin{proposition}[\inference{}]
\label{prop:tropical}
The cache traffic function $\mathcal{C}$ over tile
program compositions admits a $(\min, +)$ tropical
semiring structure.
Let $P = P_1 ; P_2$ be the sequential composition of
two tile programs with independent data footprints.
Then:
\begin{enumerate}[label=(\roman*)]
  \item \emph{Additive decomposition}: the total
    miss count decomposes as
    $\mathcal{C}(P, \sigma) =
    \mathcal{C}(P_1, \sigma|_{P_1}) +
    \mathcal{C}(P_2, \sigma|_{P_2})$.
  \item \emph{Optimal selection}:
    $\mathcal{C}^{*}(P) =
    \min_{\sigma}\, \mathcal{C}(P, \sigma)$.
\end{enumerate}
The semiring $(\NN \cup \{\infty\},\, \min,\, +)$
governs optimisation: total cost is additive over
sequential stages, and the optimal ordering minimises
the sum.
\end{proposition}

\begin{remark}
This semiring structure is inferred from the
compositional properties of
$\reuse$~\cite{SEED_3} and does not appear in the
golden sources.
Its validity depends on the assumption of independent
data footprints between composed tile programs; shared
data between stages may break additive
decomposition.
\end{remark}


\subsection{Cache Traffic Invariants}
\label{sec:cache-sem:inv}

\begin{invariant}[INV-T1: Conservation]
\label{inv:T1}
Total sector accesses decompose exactly into hits and
misses:
$|\mathrm{accesses}| =
|\mathrm{hits}| + |\mathrm{misses}|$.
The model $\mathcal{C}$ predicts $|\mathrm{misses}|$;
$|\mathrm{hits}| =
|\mathrm{accesses}| -
\mathcal{C}(\sigma, N_{\mathrm{SM}},
C_{\mathrm{L2}})$~\cite{SEED_3}.
\end{invariant}

\begin{invariant}[INV-T2: Monotonicity]
\label{inv:T2}
For streaming tile patterns, increasing active SM
count leads to non-decreasing miss rate:
if $N_{\mathrm{SM}}' > N_{\mathrm{SM}}$ then
$\mathcal{C}(\sigma, N_{\mathrm{SM}}',
C_{\mathrm{L2}})
\geq
\mathcal{C}(\sigma, N_{\mathrm{SM}},
C_{\mathrm{L2}})$~\cite{SEED_3}.
More active SMs increase L2 contention and reduce
effective per-SM cache capacity.
L2 hit rate correlates with
$N_{\mathrm{SM}}$~\cite{SEED_3}.
\end{invariant}

\begin{invariant}[INV-T3: Reuse Distance Bound]
\label{inv:T3}
Sawtooth ordering achieves
$\reuse(a, \sigma_{\mathrm{saw}}) < D_{\mathrm{data}}$
for a fraction
$\geq (1 - 1/N_{\mathrm{CTA}})$
of sector accesses~\cite{SEED_3}.
This provides a \emph{worst-case} bound on miss rate
under sawtooth ordering, independent of L2 cache
capacity.
\end{invariant}

\begin{invariant}[INV-T4: Capacity Ceiling]
\label{inv:T4}
If the total working set fits in L2
($D_{\mathrm{data}} < C_{\mathrm{L2}}$),
then the miss rate approaches zero for streaming
access patterns:
$\mathcal{C}(\sigma, N_{\mathrm{SM}},
C_{\mathrm{L2}}) \to 0$
as $D_{\mathrm{data}} / C_{\mathrm{L2}} \to 0$%
~\cite{SEED_3, ARCH_BW}.
Blackwell's ${\sim}65$\,MB L2~\cite{ARCH_BW}
combined with the 58\% cache-miss latency
reduction~\cite{ARCH_BW} makes this ceiling
achievable for moderate problem sizes.
\end{invariant}


% ====================================================================
%  SECTION 9 — WORKLOAD CLASSIFICATION AND PRECISION
% ====================================================================
\section{Workload Classification within the Model}
\label{sec:workload-class}

\subsection{HARP Taxonomy Applied to Blackwell SM}
\label{sec:workload-class:harp}

The HARP two-axis taxonomy~\cite{NV_workloads}
classifies the Blackwell SM as follows:

\begin{definition}[Blackwell SM
  Classification~\cite{NV_workloads,ARCH_BW}]
\label{def:sm-class}
Under the HARP taxonomy:
\begin{enumerate}[label=(\alph*)]
  \item \emph{Depth of compute}: the B200 GPU is
    leaf-only (compute at SM level, no hierarchical
    compute)~\cite{NV_workloads}.
  \item \emph{Heterogeneity}: the SM is
    \emph{intra-node heterogeneous}---CUDA cores and
    5th-generation tensor cores are distinct
    sub-accelerators with asymmetric latencies and data
    paths~\cite{NV_workloads, ARCH_BW}.
\end{enumerate}
\end{definition}

\begin{definition}[SM Sub-Accelerator
  Decomposition \inference{}]
\label{def:sm-decomp}
Applying the HARP taxonomy at the SM level yields a
formal object:
\[
  \mathrm{SM}(x) = \bigl(
    \underbrace{\text{CUDA\_cores}}_{\text{ALU}},\;
    \underbrace{\text{TC\_5thgen}}_{\text{MMA}},\;
    \underbrace{\TMEM{}}_{\text{near-compute}},\;
    \underbrace{\SMEM{}(x)}_{\text{shared}},\;
    \underbrace{\text{RF}}_{\text{private}}
  \bigr)
\]
with per-component latencies forming the cost vector
$\mathbf{c}(x)$ consumed by
$\mathcal{S}$~\cite{ARCH_BW, OPT_PIPE}.
\end{definition}

\begin{remark}
The introduction of \TMEM{} as a near-compute memory
tier challenges the pure leaf-only classification:
\TMEM{}-aware scheduling adds a data-placement
variable absent in prior models~\cite{ARCH_BW}.
B200 dual-die topology (148~SMs, 74 per die via
NV-HBI~\cite{ARCH_BW}) and explicit \TMEM{}
are not classified in the HARP taxonomy as
published~\cite{NV_workloads} \inference{}.
\end{remark}

\begin{figure}[t]
\centering
\begin{BVerbatim}[fontsize=\scriptsize]
+===========================================================+
|  SM Sub-Accelerator Model (HARP + ARCH_BW)                |
+===========================================================+
|                                                           |
|  Workload Descriptor W                                    |
|  +----------+----------+---------+---------+              |
|  | Operator | AI       | Reuse   | Prec.   |              |
|  +----------+----------+---------+---------+              |
|  | MMA S    | high     | reuse   | FP16    | --> TC       |
|  | MMA O    | high     | reuse   | FP16    | --> TC       |
|  | softmax  | low      | stream  | FP32    | --> CUDA     |
|  | TMA Kj   | N/A      | stream  | FP16    | --> TMA unit |
|  | TMA Vj   | N/A      | stream  | FP16    | --> TMA unit |
|  | TMEM cp  | N/A      | N/A     | FP32    | --> tcgen05  |
|  +----------+----------+---------+---------+              |
|       |                                                   |
|       v  cost vector c(x) parameterised per operator      |
|                                                           |
|  SM(10.0) = (CUDA_cores, TC_5thgen, TMEM, SMEM, RF)      |
|              |            |          |      |      |      |
|         c_ALU=?    c_MMA=11.2  UNVERIFIED  228KB 256KB   |
|                    cycles       bw values                 |
|                                                           |
|  Classification [NV_workloads]:                           |
|    Depth of compute: leaf-only                            |
|    Heterogeneity:    intra-node (CUDA + TC)               |
|    Workload type:    mixed-reuse (high-AI MMA + low-AI    |
|                      epilogue)                            |
+===========================================================+
\end{BVerbatim}
\caption{Workload classification applied to Flash
  Attention forward on Blackwell.
  Each operator in $\mathcal{W}$ is labelled with
  arithmetic intensity, reuse class, and
  precision~\cite{NV_workloads, ARCH_BW}.
  The SM sub-accelerator model routes operators to
  physical units with distinct costs.
  Mixed-reuse workloads~\cite{NV_workloads} contain
  both compute-bound (MMA) and memory-bound (softmax)
  operators.}
\label{fig:workload-class}
\end{figure}


\subsection{Precision-Dependent Cost Conditioning}
\label{sec:workload-class:prec}

The precision lattice $\mathcal{P}$
(\cref{def:prec-lattice-full}) conditions the
scheduling cost vector on a per-operator basis.
Different precisions induce different SASS
instructions~\cite{ARCH_BW}, potentially different
\SMEM{} layout requirements~\cite{SEED_1}, and
different throughput
scalings~\cite{ARCH_BW}.

\begin{definition}[Precision-Parameterised
  Layout Space \inference{} (Gap~G8)]
\label{def:layout-prec}
For each precision $p \in \mathcal{P}$, the valid
layout space is:
\[
  \mathcal{L}(p) =
  \bigl\{f \in \mathrm{Mor}_{\mathcal{L}} :
  \mathrm{bits}(f) = \mathrm{bits}(p),\;
  f \text{ satisfies alignment}(p)
  \bigr\}
\]
where $\mathrm{alignment}(p)$ encodes MMA tile shape
requirements
$\mathrm{MMA\_M}(p) \times \mathrm{MMA\_N}(p) \times
\mathrm{MMA\_K}(p)$ for the hardware~\cite{ARCH_BW}.
The $\Ftwo$ matrix dimension changes with element
bit-width: wider elements require fewer address bits
for the same tile shape~\cite{SEED_1}.
\end{definition}

\begin{remark}
FP4/FP6 layout requirements and block scaling formats
(MXFP4/MXFP8) are not fully documented in the golden
sources \unverified{}.
Empirical testing with \tcgen{} microbenchmarks is
required to characterise precision-specific swizzle
patterns and determine which fall outside the $\Ftwo$
fragment (gap~G8)~\cite{ARCH_BW, SEED_1}.
\end{remark}


% ====================================================================
%  SECTION 10 — OBJECTIVE FUNCTIONS AND PROOF OBLIGATIONS
% ====================================================================
\section{Objective Functions and Proof Obligations}
\label{sec:obj-proofs}

This section states the three optimisation objectives
that \BUTA{} targets and the six proof obligations
required to establish model soundness.

\subsection{Objective Functions}
\label{sec:obj-proofs:obj}

\begin{definition}[OBJ-1: Minimise Initiation
  Interval~\cite{OPT_PIPE}]
\label{def:obj1}
\begin{equation}
  \min\; \II
  \quad \text{subject to}\quad
  \Phi_{\mathrm{dep}} \wedge
  \Phi_{\mathrm{res}} \wedge
  \Phi_{\mathrm{fu}} \wedge
  \Phi_{\mathrm{2cta}}
  \label{eq:obj1}
\end{equation}
This is the primary scheduling objective:
minimise the steady-state cycle count per loop
iteration.
The ILP determines a lower bound; the SMT holistically
solves the joint SWP+WS
problem~\cite{OPT_PIPE}.
Cost vector $\mathbf{c}(x)$ is parameterised
by~\cite{ARCH_BW}.
\end{definition}

\begin{definition}[OBJ-2: Minimise L2 Cache
  Misses~\cite{SEED_3,ARCH_BW}]
\label{def:obj2}
\begin{equation}
  \min_{\sigma \in \mathrm{Perm}(\mathrm{Tiles})}\;
  \mathcal{C}(\sigma, N_{\mathrm{SM}}, C_{\mathrm{L2}})
  \quad \text{s.t.}\quad
  \text{INV-T1} \wedge \cdots \wedge \text{INV-T4}
  \label{eq:obj2}
\end{equation}
subject additionally to layout constraints
INV-L1\,--\,L4 (tile ordering must be compatible with
the chosen layout morphisms).
The sawtooth ordering~\cite{SEED_3} provides a known
feasible solution;
OBJ-2 seeks to establish whether it is optimal or
whether better orderings exist for specific workloads.
\end{definition}

\begin{definition}[OBJ-3: Composite
  Cost \inference{}]
\label{def:obj3}
\begin{equation}
  \min\;
  \alpha \cdot \II
  \;+\;
  \beta \cdot \mathcal{C}(\sigma, N_{\mathrm{SM}},
    C_{\mathrm{L2}})
  \;+\;
  \gamma \cdot \mathrm{DM}_{\mathrm{vol}}
  \label{eq:obj3}
\end{equation}
subject to all invariants
(INV-L1\,--\,L4, INV-D1\,--\,D4,
INV-S1\,--\,S5, INV-T1\,--\,T4)
and interface constraints~$\Psi$
(\cref{def:interface}),
where
$\mathrm{DM}_{\mathrm{vol}} = \sum_{e \in
\mathcal{D}} \mathrm{size}(e)$
is the total data movement volume across
$\mathcal{H}$ edges, and
$\alpha, \beta, \gamma > 0$ are calibration
weights~\cite{OPT_PIPE, SEED_3, ARCH_BW}.
\end{definition}

\begin{remark}
OBJ-3 is a \emph{multi-objective} formulation.
The weights $\alpha, \beta, \gamma$ must be calibrated
empirically: $\alpha$ in cycles per initiation,
$\beta$ in cycles per L2 miss (related to the 58\%
cache-miss latency reduction on
Blackwell~\cite{ARCH_BW}), and $\gamma$ in cycles per
byte of data movement.
OBJ-3 does not appear in any golden source and is
an \inference{} contribution of \BUTA{}.
\end{remark}

\begin{figure}[t]
\centering
\begin{BVerbatim}[fontsize=\scriptsize]
+===========================================================+
|  Objective Function Architecture                          |
+===========================================================+
|                                                           |
|  OBJ-1: min II             OBJ-2: min C(sigma)           |
|  [OPT_PIPE + ARCH_BW]     [SEED_3 + ARCH_BW]             |
|         \                      /                          |
|          \    OBJ-3: min      /                           |
|           \  alpha*II +      /                            |
|            \ beta*C(sigma)+ /                             |
|             \ gamma*DM_vol /                              |
|              \            /                               |
|               v          v                                |
|        +---------------------+                            |
|        | Joint Optimisation  |                            |
|        +-----+-----+--------+                            |
|              |     |                                      |
|     Subject to:    |                                      |
|  INV-L1..L4  |  INV-D1..D4                               |
|  INV-S1..S5  |  INV-T1..T4                               |
|  Interface   |  constraints Psi                           |
|              v                                            |
|  +---------------------------------------------------+   |
|  | Calibration: alpha, beta, gamma from               |   |
|  | hardware counters on B200/GB10 [ARCH_BW, SEED_3]   |   |
|  +---------------------------------------------------+   |
+===========================================================+
\end{BVerbatim}
\caption{Objective function architecture.
  OBJ-1 and OBJ-2 are anchored in golden
  sources~\cite{OPT_PIPE, SEED_3};
  OBJ-3 is a composite \inference{} objective
  combining scheduling, cache, and data-movement
  costs under all model invariants.}
\label{fig:obj-arch}
\end{figure}


\subsection{Proof Obligations}
\label{sec:obj-proofs:po}

We state six proof obligations required for \BUTA{}
model soundness.
Proof Obligations~1--5 each address a specific
inter-layer correctness property;
Proof Obligation~6 is the end-to-end composition
theorem.

\begin{obligation}[PO-1: $\Ftwo$-to-\ISL{}
  Embedding Correctness (Gap~G1)]
\label{po:1}
\emph{Statement.}\;
The embedding $\phi$ (\cref{def:phi-embed}) is
injective and preserves composition over the binary
domain.

\smallskip\noindent
\emph{Planned technique.}\;
Constructive proof: define
$\phi(L)$ as an \ISL{} relation with binary domain
constraints ($b_j \in \{0,1\}$); show injectivity by
demonstrating that distinct $\Ftwo$ matrices yield
distinct \ISL{} relations;
prove composition preservation via \ISL{} relation
composition semantics matching $\Ftwo$ matrix
multiplication~\cite{SEED_1, SEED_2}.
Automated checking via \ISL{} \texttt{compute\_flow}
for small layout sizes.

\smallskip\noindent
\emph{Falsifiable prediction.}\;
All $\Ftwo$ layouts in the Triton test suite
have verified \ISL{} equivalents~\cite{SEED_1};
composition preservation holds for all layout pairs in
the CUTLASS test suite~\cite{SEED_4}.
\end{obligation}

\begin{obligation}[PO-2: Categorical-to-\ISL{}
  Compatibility (Gap~G2)]
\label{po:2}
\emph{Statement.}\;
The functor $F$ (\cref{def:F-functor}) preserves
composition, logical product, and logical division;
its image coincides with the tractable layout
set~\cite{SEED_4}.

\smallskip\noindent
\emph{Planned technique.}\;
Category-theoretic proof with Python
implementation extending the SEED\_4
codebase~\cite{SEED_4} to emit \ISL{}
relations; cross-validation with Algorithm~1
of~\cite{SEED_2}.
Verify that the image of $F$ covers all \CuTe{}
layouts convertible by
Algorithm~1~\cite{SEED_2}.

\smallskip\noindent
\emph{Falsifiable prediction.}\;
Functor $F$ defined and tested on all CUTLASS test
layouts from~\cite{SEED_4}; image of $F$ covers all
\CuTe{} layouts handled by Algorithm~1
of~\cite{SEED_2}.
\end{obligation}

\begin{obligation}[PO-3: Schedule Feasibility
  Soundness (Gap~G6)]
\label{po:3}
\emph{Statement.}\;
Any schedule $(M, W, \II)$ satisfying the SMT
constraints $\Phi$ is \emph{realisable} on a
Blackwell SM:
resource counts are respected, dependency latencies
are met, and 2CTA pairing is valid.

\smallskip\noindent
\emph{Planned technique.}\;
Given a satisfying assignment from the SMT solver,
verify:
(a)~resource counts against
$\mathcal{R}(s, x)$~\cite{ARCH_BW};
(b)~dependency latencies against \tcgen{}\texttt{.mma}
and \tma{} measured
costs~\cite{ARCH_BW};
(c)~2CTA pairing constraints match TPC topology
(\texttt{num\_ctas=2})~\cite{NV_BLOG_TILE}.
Reference: Twill~\cite{OPT_PIPE} rediscovery of
expert Flash Attention schedules serves as
cross-validation.

\smallskip\noindent
\emph{Falsifiable prediction.}\;
Solver with calibrated Blackwell cost vector produces
schedules matching or exceeding expert FA3/FA4
schedules~\cite{OPT_PIPE};
Blackwell parameterisation yields distinct optimal
schedules vs.\ Hopper~\cite{OPT_PIPE}.
\end{obligation}

\begin{obligation}[PO-4: Cache Model
  Fidelity (Gap~G5)]
\label{po:4}
\emph{Statement.}\;
Predicted L2 miss count
$\mathcal{C}(\sigma, N_{\mathrm{SM}}, C_{\mathrm{L2}})$
matches measured hardware counter values within a
bounded relative error.

\smallskip\noindent
\emph{Planned technique.}\;
Derive predicted L2 miss count from reuse-distance
model~\cite{SEED_3}; compare against hardware
performance counters (via Nsight Compute) on
GB10~\cite{SEED_3} and
B200~\cite{ARCH_BW};
bound prediction error statistically with confidence
interval.
Reproduce the SEED\_3 67\% miss reduction
result~\cite{SEED_3} as a baseline.

\smallskip\noindent
\emph{Falsifiable prediction.}\;
Predicted miss count within 15\% of measured for
sawtooth-ordered Flash Attention on
GB10~\cite{SEED_3};
model correctly predicts when sawtooth ordering helps
vs.\ hurts.
\end{obligation}

\begin{obligation}[PO-5: Memory Ordering
  Correctness (Gap~G7)]
\label{po:5}
\emph{Statement.}\;
The \TileIR{} token ordering
$\preceq_{\mathcal{T}}$~(\cref{def:token-po})
is respected by any schedule $(M, W, \II)$ produced
by $\mathcal{S}$;
well-formed token DAGs are data-race free.

\smallskip\noindent
\emph{Planned technique.}\;
Formalise token dependencies as additional edges in
the dependence graph $G$: for each token edge
$(u, v) \in \mathcal{T}$, add dependency
$M(v) \geq M(u) + \mathrm{lat}(u)$
to~$\Phi_{\mathrm{dep}}$~\cite{NV_BLOG_TILE,
OPT_PIPE}.
Data-race freedom (\cref{def:drf}) is checked as a
decidable property of the token DAG:
for all conflicting pairs, verify
ordering~\cite{NV_BLOG_TILE}.

\smallskip\noindent
\emph{Falsifiable prediction.}\;
Token-augmented solver produces correct schedules
for all Triton-to-\TileIR{} test programs;
no data races detected in solver output for
well-formed programs.
\end{obligation}

\begin{obligation}[PO-6: End-to-End Semantic
  Preservation \inference{}]
\label{po:6}
\emph{Statement.}\;
The composition of layout selection
($\mathcal{L}$), data movement ($\mathcal{D}$),
scheduling ($\mathcal{S}$), and cache ordering
($\mathcal{C}$) preserves the operational semantics of
the original tile program: the output tensor computed
under the optimised configuration is
\emph{bitwise identical} to the output under a
reference sequential execution.

\smallskip\noindent
\emph{Planned technique.}\;
Layer simulation relations:
define a simulation relation
$\sim_{\ell}$ for each semantic layer~$\ell$
such that:
\begin{enumerate}[label=(\roman*)]
  \item $\mathcal{L}$: layout morphisms preserve
    data values (INV-L1 bijectivity);
  \item $\mathcal{D}$: token DAG acyclicity
    (INV-D2) and capacity (INV-D3) ensure
    well-ordered data delivery;
  \item $\mathcal{S}$: schedule respects
    dependencies (INV-S2) and resources (INV-S1);
  \item $\mathcal{C}$: cache traffic affects only
    performance, not correctness
    (conservation, INV-T1).
\end{enumerate}
Compose: $\sim \;= \;\sim_{\mathcal{L}} \circ
\sim_{\mathcal{D}} \circ \sim_{\mathcal{S}} \circ
\sim_{\mathcal{C}}$ is a bisimulation between
optimised and reference
executions~\cite{SEED_1, SEED_2, SEED_4,
OPT_PIPE, NV_BLOG_TILE, ARCH_BW}.

\smallskip\noindent
\emph{Falsifiable prediction.}\;
Optimised Flash Attention produces bitwise-identical
output to reference implementation for all test
inputs on both B200 and GB10.
\end{obligation}


\subsection{Proof Obligation Dependency Structure}
\label{sec:obj-proofs:deps}

\begin{figure}[t]
\centering
\begin{BVerbatim}[fontsize=\scriptsize]
+===========================================================+
|  Proof Obligation Dependency Graph                        |
+===========================================================+
|                                                           |
|  PO-1: F2->ISL        PO-2: Cat->ISL                     |
|  [SEED_1, SEED_2]     [SEED_4, SEED_2]                   |
|       \                   /                               |
|        \                 /                                |
|         v               v                                 |
|  +------+---+---+------+                                 |
|  |  Layout Layer Sound  |                                 |
|  |  (INV-L1..L4)        |                                 |
|  +----------+-----------+                                 |
|             |                                             |
|             v                                             |
|  PO-3: Schedule Sound    PO-5: Token Order                |
|  [OPT_PIPE, ARCH_BW,    [NV_BLOG_TILE,                   |
|   NV_BLOG_TILE]           OPT_PIPE]                       |
|       \                      /                            |
|        \                    /                             |
|         v                  v                              |
|     PO-4: Cache Fidelity                                  |
|     [SEED_3, ARCH_BW]                                     |
|              \                                            |
|               v                                           |
|  +---------------------------+                            |
|  |  PO-6: End-to-End        |                             |
|  |  Semantic Preservation    |                             |
|  |  [ALL SOURCES]            |                             |
|  |  Requires: PO-1..PO-5    |                             |
|  +---------------------------+                            |
|                                                           |
|  Legend:                                                   |
|    ---->  =  "depends on"                                 |
|    PO-6 is the capstone: requires all prior POs           |
+===========================================================+
\end{BVerbatim}
\caption{Dependency graph among proof obligations.
  PO-1 and PO-2 establish layout-layer soundness;
  PO-3 and PO-5 establish scheduling and memory
  ordering soundness;
  PO-4 establishes cache model fidelity;
  PO-6 (end-to-end) requires all five.
  Each PO is anchored to specific golden
  sources.}
\label{fig:po-deps}
\end{figure}

\noindent
\Cref{fig:po-deps} illustrates the dependency
structure.
PO-6 is the \emph{capstone} obligation: it composes
the soundness guarantees of PO-1--PO-5 into a
single end-to-end statement.
The dependency structure reveals that layout soundness
(PO-1, PO-2) is a prerequisite for all downstream
obligations, since layout correctness underpins the
data movement relation $\mathcal{D}$ and the
scheduling cost vector in $\mathcal{S}$.

\subsection{Summary of Model Guarantees}
\label{sec:obj-proofs:summary}

\Cref{tab:guarantee-summary} links each semantic layer
to its invariants, objective functions, and proof
obligations.

\begin{table}[t]
\centering
\caption{Model guarantee matrix: layers, invariants,
  objectives, and proof obligations.}
\label{tab:guarantee-summary}
\footnotesize
\begin{tabular}{@{}llll@{}}
\toprule
\textbf{Layer} & \textbf{Invariants}
  & \textbf{Obj.} & \textbf{POs} \\
\midrule
Layout $\mathcal{L}$
  & INV-L1..L4 & --- & PO-1, PO-2 \\
Data Mvmt $\mathcal{D}$
  & INV-D1..D4 & (OBJ-3) & PO-5 \\
Schedule $\mathcal{S}$
  & INV-S1..S5 & OBJ-1 & PO-3 \\
Cache $\mathcal{C}$
  & INV-T1..T4 & OBJ-2 & PO-4 \\
Composite
  & All & OBJ-3 & PO-6 \\
\bottomrule
\end{tabular}
\end{table}


% ====================================================================
%  SOURCE AUDIT
% ====================================================================
\section*{Source Audit (Part~2)}
\label{sec:audit-p2}

\noindent
All eight golden sources are cited in this part.
\Cref{tab:audit-p2} records usage.

\begin{table}[H]
\centering
\caption{Part~2 source audit: usage and anchors.}
\label{tab:audit-p2}
\footnotesize
\begin{tabular}{@{}p{1.3cm}p{6.0cm}@{}}
\toprule
\textbf{ID} & \textbf{Used for (anchor)} \\
\midrule
ARCH\_BW &
  Memory tier graph parameterisation (\TMEM{} 256\,KB,
  \SMEM{} 228/128\,KB, L2 ${\sim}$65\,MB);
  SM resource bounds (148 SMs, 4 WS, 64/48 warps);
  \tcgen{}.mma latency 11.0--11.4 cycles;
  precision lattice and SASS mappings;
  58\% cache-miss latency reduction;
  MMA.2SM TPC-internal bandwidth;
  cost vector parameterisation for $\mathcal{S}$. \\
\addlinespace
OPT\_PIPE &
  Schedule constraint system $\mathcal{S}$ definition
  ($G{=}(V,E)$, $\mathrm{op}[v,i,t]$, $W(v)$, $\II$);
  ILP for modulo scheduling; SMT for joint SWP+WS;
  ZLP normalisation; dependency constraint formulation;
  functional unit exclusivity;
  Twill cross-validation of PO-3. \\
\addlinespace
NV\_BLOG\_ TILE &
  Data movement relation $\mathcal{D}$ (TMA descriptors,
  token-ordered memory model, waits-for relation,
  happens-before);
  2CTA MMA constraint ($\Phi_{\mathrm{2cta}}$,
  $\texttt{num\_ctas}{=}2$);
  \TileIR{} precision types;
  occupancy hint limitation (\texttt{num\_warps}
  not exposed);
  data-race freedom definition. \\
\addlinespace
NV\_work\-loads &
  Workload descriptor $\mathcal{W}$ (two-axis HARP
  taxonomy: depth of compute $\times$ heterogeneity);
  B100 leaf-only classification;
  GPU+TC intra-node heterogeneous;
  mixed-reuse workload characterisation;
  SM sub-accelerator decomposition. \\
\addlinespace
SEED\_1 &
  $\Ftwo$ representation layer of $\mathcal{L}$
  (binary matrices, composition, layout-to-layout
  conversion); linear fragment closure (INV-L4);
  precision-dependent layout bit-width;
  embedding $\phi$ source definition. \\
\addlinespace
SEED\_2 &
  \ISL{} representation layer of $\mathcal{L}$
  (quasi-affine relations, Algorithm~1 CuTe$\to$ISL);
  swizzle as bit-level ISL operation (INV-L3);
  tiling as quasi-affine transformation;
  cross-system layout unification;
  functor $F$ target definition. \\
\addlinespace
SEED\_3 &
  Cache traffic function $\mathcal{C}$ (reuse-distance
  formalism, sawtooth ordering bound, 67\% L2 miss
  reduction on GB10, 60\% throughput increase);
  L1 negligible for streaming attention;
  L2 hit rate ${\sim}f(N_{\mathrm{SM}})$;
  INV-T1..T4 anchoring; PO-4 validation basis. \\
\addlinespace
SEED\_4 &
  Categorical representation layer of $\mathcal{L}$
  ($\mathbf{Tuple}$/$\mathbf{Nest}$ categories,
  morphisms, compatibility with CuTe operations);
  tractable layout characterisation;
  CUTLASS validation; functor $F$ source definition;
  PO-2 anchoring. \\
\bottomrule
\end{tabular}
\end{table}


% ====================================================================
%  BIBLIOGRAPHY
% ====================================================================
\begin{thebibliography}{8}

\bibitem{ARCH_BW}
A.~Jarmusch and S.~Chandrasekaran,
``Microbenchmarking NVIDIA's Blackwell Architecture:
An In-Depth Architectural Analysis,''
\emph{arXiv:2512.02189v1}, Dec.~2025.
\url{https://arxiv.org/html/2512.02189v1}

\bibitem{OPT_PIPE}
K.~Soi, A.~Venkat, T.~Nowicki, and
G.~Biros,
``Optimal Software Pipelining and Warp
Specialization for Tensor Core GPUs,''
\emph{arXiv:2512.18134v1}, Dec.~2025.
\url{https://arxiv.org/html/2512.18134v1}

\bibitem{NV_BLOG_TILE}
NVIDIA,
``Advancing GPU Programming with the CUDA Tile IR
Backend for OpenAI Triton,''
\emph{NVIDIA Developer Blog}, Jan.~2026.
\url{https://developer.nvidia.com/blog/advancing-gpu-programming-with-the-cuda-tile-ir-backend-for-openai-triton/}

\bibitem{NV_workloads}
S.~Garg, M.~Pellauer, and T.~Krishna,
``HARP: A Taxonomy for Heterogeneous and
Hierarchical Processors for Mixed-Reuse Workloads,''
\emph{arXiv:2502.13113}, Feb.~2025.
\url{https://arxiv.org/html/2502.13113}

\bibitem{SEED_1}
J.~Zhou, M.~Lezcano, A.~Goucher, et~al.,
``Linear Layouts: Robust Code Generation of Efficient
Tensor Computation Using $\mathbb{F}_2$,''
\emph{arXiv:2505.23819v3}, Oct.~2025.
\url{https://arxiv.org/html/2505.23819v3}

\bibitem{SEED_2}
S.~Bhaskaracharya, N.~Acharya, M.~Hagedorn,
and S.~Grover,
``Modeling Layout Abstractions Using Integer Set
Relations,''
\emph{arXiv:2511.10374v1}, Nov.~2025.
\url{https://arxiv.org/html/2511.10374v1}

\bibitem{SEED_3}
Z.~Zhu, J.~Pan, and Y.~Ding,
``Sawtooth Wavefront Reordering: Enhanced CuTile
FlashAttention on NVIDIA GB10,''
\emph{arXiv:2601.16032v2}, Jan.~2026.
\url{https://arxiv.org/html/2601.16032v2}

\bibitem{SEED_4}
R.~Carlisle, A.~Shah, H.~Stern, and
D.~VanKoughnett,
``Categorical Foundations for CuTe Layouts,''
\emph{arXiv:2601.05972v1}, Jan.~2026.
\url{https://arxiv.org/pdf/2601.05972v1}

\end{thebibliography}

\end{document}